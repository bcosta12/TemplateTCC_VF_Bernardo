\section{Pré-processamento}

\subsection{Base de Dados}
\label{subsec:base_de_dados}
 Com o objetivo de alimentar o modelo proposto foram utilizados dois tipos de dados: o do valor do preço do Bitcoin, fornecidos pela Coinbase \cite{Coinbase}, e o do valor do Google Trends, fornecidos pelo próprio Google. Com esses dados foi possível criar os dados que treinaram e testaram o modelo proposto.
 
Para comparar os resultados obtidos com o o resultados apresentados no trabalho \cite{wilson_yelowitz_2014}, foram utilizados dados no do dia 19 de Agosto de 2013 até 19 de Julho de 2016, que é o mesmo intervalo utilizado pelo trabalho \cite{wilson_yelowitz_2014}.
 
 Os gráficos dos dados utilizados, tanto do valor do preço do Bitcoin como do valor do Google Trends, podem ser vistos na Figura \ref{fg:comparacao_btc}
 
 \subsection{Normalização}
 \label{subsec:normalizacao}
A normalização dos dados é uma técnica utilizada antes dos dados serem inseridos para a Rede Neural. Essa técnica consiste em subtrair cada amostra original pela média das amostras e dividir pelo desvio padrão das amostras. O objetivo dessa técnica é suavizar os dados, ou seja, remover picos, ou anomalias, dos dados \cite{quackenbush2002microarray}

A equação utilizada para a normalização pode ser vista na Eq. \ref{fun:normalizacao}, onde: $x$ é o valor original; $\overline{m}$ é média da coluna referente a amostra; e $\sigma$ é o desvio padrão da coluna referente a amostra.

\begin{equation}
    \label{fun:normalizacao}
 \mathbf{X'} = (\mathbf{X}-\overline{m})/\sigma
\end{equation}

Para exemplificar o que a técnica de normalização faz com o dados, os gráficos dos dados normalizados, tanto do valor do preço do Bitcoin como do valor do Google Trends, podem ser vistos na Figura \ref{fg:comparacao_btc_n}.

\subsection{PCA}

O PCA, do inglês \textit{Principal Component Analysis}, é um algoritmo que tem como objetivo principal reduzir a dimensão dos dados, de modo a manter suas informações, aliviando assim a necessidade de poder de processamento para computar a mesma informação. De modo mais genérico, o PCA permite reduzir a dimensão de uma base de dados de $d$ para dimensão $d'$, onde $0<d'<=d$ \cite{Jolliffe}

Para realizar a redução na dimensão dos dados mantendo a essência da informação, o PCA busca relacionar os pontos da base de dados de dimensão $d$ com um espaço de dimensão $d'$, representando os dados no espaço menor que o original. Por exemplo: considere uma base de dados de dimensão igual a $2$ que contém diversos pontos $P_n$. Para reduzir a dimensão desses dados para $1$, o algoritmo do PCA traça a reta (que tem uma dimensão igual a $1$, ou seja, menor que a dimensão original) que minimiza a distância de cada um dos pontos $P_n$ até essa reta. Traçada essa reta, é preciso mapear cada um dos pontos $P_n$ na reta, para isso, o Algoritmo marca na reta o ponto que tiver a menor distância para cada ponto $P_n$, que será a sua nova representação. O término desse processo, os dados estarão representados em uma base de dados com uma dimensão a menos que a base de dados original. O mesmo processo é realizado para reduzir uma base de dados de $3$ para $2$ dimensões, o espaço de $3$ dimensões é mapeado em um plano de duas dimensões. Generalizando, o algoritmo do PCA mapeia pontos de uma base de dados de dimensão $d$ para dimensão $d'$, onde $0<d'<=d$, calculando a menor distância entre os pontos originais e o espaço com dimensão inferior, sucessivamente até a dimensão $d'$ requerida. 

O algoritmo PCA se baseia na premissa de que os dados estão normalizados, então é preciso executar a normalização apresentada na seção \ref{subsec:normalizacao}. Feito isso, é possível aplicar uma série de operações matemáticas com os dados, apresentadas no algoritmo do PCA. A prova matemática desse algoritmo para o cálculo do PCA foge ao escopo deste trabalho, mas pode ser encontrada em \cite{Jolliffe}.

O Algoritmo abaixo apresenta os passos para o cálculo do PCA, recebendo como entrada a base de dados original, $\mathbf{X}$, e a dimensão desejada, $d'$. O Algoritmo executa os seguintes passos:  
\begin{enumerate}
    \item calcula-se a dimensão de $\mathbf{X}$, $d$;
    \item calcula-se a matriz de covariância, $\Sigma$;
    \item calcula-se o autovalor, $mathbf{U}$, de $\Sigma$;
    \item seleciona-se as $d'$ primeiras colunas da $mathbf{U}$, que chamaremos de $\mathbf{U_{reduzido}}$;
    \item calcula-se a resposta, $z$, que é a multiplicação de $\mathbf{U_{reduzido}}^T $ e a base original, $\mathbf{X}$.
\end{enumerate}

Como é provado matematicamente em \cite{Jolliffe}, com essas operações obtém-se o PCA. 

\begin{center}
 \begin{algorithm}[H]
   \SetAlgoLined
   \Entrada{$\mathbf{X}, d'$} 
   \Saida{$z$}
   \Inicio{
   $d = dim(\mathbf{X})$\\
   $\Sigma = \frac{1}{d} *  \mathbf{X}^T * \mathbf{X}$\\
   $ \mathbf{U} = autovalor(\Sigma)$\\
   $  \mathbf{U_{reduzido}} =  \mathbf{U}[:,1:d']$\\
   $  \mathbf{z} = \mathbf{U_{reduzido}}^T * \mathbf{X}$\\
   
   }
   \Retorna{$\mathbf{z}$}
   \label{alg:PCA1}
   \caption{\textsc{Algoritmo do PCA }}
 \end{algorithm}
\end{center}

Para ilustrar o funcionamento do PCA, vamos considerar os dados referentes ao valor do preço do Bitcoin e do Google Trends, apresentados na seção \ref{subsec:base_de_dados}. Essas bases de dados serão concatenadas para formar um gráfico com 3 dimensões. Logo, os dados consistem em uma lista de valores, onde a primeira coluna corresponde a data, a segunda ao valor do preço do Bitcoin e a terceira ao valor do Google Trends. Um gráfico em duas dimensões exibindo as duas listas de valores é apresentada na Figura \ref{gr:2d}. 

Com os dados anteriores, foi executado o algoritmo do PCA com $c=2$, reduzindo a dimensão dos dados para 1. A Figura \ref{gr:1d} apresenta o gráfico resultante do PCA.

Na próxima seção serão apresentados os algoritmos e a metodologia utilizada no trabalho. 

\newpage
 %%%%%%%%%%%%%%%%%%%%%%%%%%%%%%%%%%%%%%%
\begin{center}
	\begin{figure}[H]
    \caption{Comparação dos gráficos do valor do Bitcoin e do \textit{Google Trends} no mesmo período utilizado no trabalho \cite{wilson_yelowitz_2014}}\label{fg:comparacao_btc_n}

	\begin{subfigure}{}
		
		\begin{tikzpicture}

		\begin{axis}[
		title={Valor do preço do Bitcoin},
		    date coordinates in=x,
		    xticklabel={\day-\month-\year},
		    x tick label style={rotate=45,anchor=north east},
		    date ZERO=2013-08-19,
		    xmin=2013-08-19, xmax=2016-07-19,
		    ymin=0,ymax=1102,
		    height=8cm,
			width=14cm,
			grid=major,
		    xlabel={Data (dia-mês-ano)},
		    ylabel={Valor em dólares},
		]
		\addlegendentry{Preço BTC}
		\addplot [blue, mark=*] table [x=date, y=value, col sep=comma] {price.csv};
		\end{axis}
		
		\end{tikzpicture}
        \label{gr:btc}
	\end{subfigure}
	%sub fig 2
	\begin{subfigure}{}
		\begin{tikzpicture}

		\begin{axis}[
			title={Valor \textit{Google Trends} do Bitcoin}, 
		    date coordinates in=x,
		    xticklabel={\day-\month-\year},
		    x tick label style={rotate=45,anchor=north east},
		    xmin=2013-08-19, xmax=2016-07-19,
		    ymin=0,ymax=249,
		    height=8cm,
			width=14cm,
			grid=major,
		    xlabel={Data (dia-mês-ano)},
		    ylabel={Valor \%},  
		]
		\addlegendentry{Trends BTC}
		\addplot [red, mark=*] table [x=date, y=bitcoin, col sep=comma] {gt.csv};
		\end{axis}
		\end{tikzpicture}
    	\label{gr:bitcoin_gtrends_mesmoperiodo}
	\end{subfigure}
	\begin{center}
	    Fonte: O autor (2018)
	\end{center}
	\end{figure}
\end{center}
%%%%%%%%%%%%%%%%%%%%%%%%%%%%%%%%%%%%%%%
\newpage
 %%%%%%%%%%%%%%%%%%%%%%%%%%%%%%%%%%%%%%%
\begin{center}
	\begin{figure}[H]
    \caption{Comparação dos gráficos do valor do Bitcoin e do \textit{Google Trends} normalizados, no mesmo período utilizado no trabalho \cite{wilson_yelowitz_2014}}\label{fg:comparacao_btc_normalizado}

	\begin{subfigure}{}
		
		\begin{tikzpicture}

		\begin{axis}[
		title={Valor do preço do Bitcoin Normalizados},
		    date coordinates in=x,
		    xticklabel={\day-\month-\year},
		    x tick label style={rotate=45,anchor=north east},
		    date ZERO=2013-08-19,
		    xmin=2013-08-19, xmax=2016-07-19,
		    ymin=-2,ymax=3.7,
		    height=8cm,
			width=14cm,
			grid=major,
		    xlabel={Data (dia-mês-ano)},
		    ylabel={Valor em dólares},
		]
		\addlegendentry{Preço BTC}
		\addplot [blue, mark=*] table [x=date, y=value, col sep=comma] {price_n.csv};
		\end{axis}
		\end{tikzpicture}
        \label{gr:btc_normalizado}
	\end{subfigure}
	%sub fig 2
	\begin{subfigure}{}
		\begin{tikzpicture}

		\begin{axis}[
			title={Valor do \textit{Google Trends}  do termor Bitcoin Normalizado}, 
		    date coordinates in=x,
		    xticklabel={\day-\month-\year},
		    x tick label style={rotate=45,anchor=north east},
		    xmin=2013-08-19, xmax=2016-07-19,
		    ymin=-2,ymax=8.4,
		    height=8cm,
			width=14cm,
			grid=major,
		    xlabel={Data (dia-mês-ano)},
		    ylabel={Valor \%},  
		]
		\addlegendentry{Trends BTC}
		\addplot [red, mark=*] table [x=date, y=bitcoin, col sep=comma] {gt_n.csv};
		\end{axis}
		\end{tikzpicture}
    	\label{gr:btc_gtrends_n}
	\end{subfigure}
	\begin{center}
	    Fonte: O autor (2018)
	\end{center}
	\end{figure}
\end{center}
%%%%%%%%%%%%%%%%%%%%%%%%%%%%%%%%%%%%%%%
\newpage
 %%%%%%%%%%%%%%%%%%%%%%%%%%%%%%%%%%%%%%%
\begin{center}
	\begin{figure}[H]
    \caption{Comparação dos gráficos do valor do Bitcoin e do \textit{Google Trends} normalizados concatenados, com o resultado dos mesmos dados após o PCA com $n=2$, no mesmo período utilizado no trabalho \cite{wilson_yelowitz_2014}}\label{fg:comparacao_btc}

	\begin{subfigure}{}
		
		\begin{tikzpicture}

		\begin{axis}[
		title={Valor do Bitcoin e do \textit{Google Trends} do termo Bitcoin Normalizados, sobrepostos},
		    date coordinates in=x,
		    xticklabel={\day-\month-\year},
		    x tick label style={rotate=45,anchor=north east},
		    date ZERO=2013-08-19,
		    xmin=2013-08-19, xmax=2016-07-19,
		    ymin=-2,ymax=8,
		    height=8cm,
			width=14cm,
			grid=major,
		    xlabel={Data (dia-mês-ano)},
		    ylabel={Valor em dólares},
		]
		\addlegendentry{Trends BTC}
		\addplot [blue, mark=*] table [x=date, y=gtrends, col sep=comma] {2d.csv};
		\addlegendentry{Preço BTC}
		\addplot [red, mark=*] table [x=date, y=bitcoin, col sep=comma] {2d.csv};
		\end{axis}
		\end{tikzpicture}
        \label{gr:2d}
	\end{subfigure}
	%sub fig 2
	\begin{subfigure}{}
		\begin{tikzpicture}

		\begin{axis}[
			title={Valor do Bitcoin e do \textit{Google Trends} fo Bitcoin Normalizados após aplicação do PCA}, 
		    date coordinates in=x,
		    xticklabel={\day-\month-\year},
		    x tick label style={rotate=45,anchor=north east},
		    xmin=2013-08-19, xmax=2016-07-19,
		    ymin=-350,ymax=700,
		    height=8cm,
			width=14cm,
			grid=major,
		    xlabel={Data (dia-mês-ano)},
		    ylabel={Valor},  
		]
		\addlegendentry{PCA com BTC e Trends}
		\addplot [purple, mark=*] table [x=date, y=mix, col sep=comma] {1d.csv};
		\end{axis}
		\end{tikzpicture}
    	\label{gr:1d}
	\end{subfigure}
	\begin{center}
	    Fonte: O autor (2018)
	\end{center}
	\end{figure}
\end{center}
%%%%%%%%%%%%%%%%%%%%%%%%%%%%%%%%%%%%%%%