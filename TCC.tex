\documentclass[12pt,twoside,final]{article}

\usepackage[portuguese,brazil]{babel}
%\usepackage[latin1]{inputenc}
\usepackage[utf8]{inputenc}


%\usepackage{indentfirst, natbib}
\usepackage{indentfirst}
\usepackage{caption}
\RequirePackage{etex}
\usepackage{hyperref,ae}
% \usepackage{harvard}
\usepackage{pslatex}  %fonte times new roman
\usepackage{amssymb,fancyhdr,fancybox,epsfig,psfrag,amsmath,tabularx}
\usepackage[paperwidth=8.5in,paperheight=11in,hmargin={25mm,20mm},vmargin={20mm,20mm}]{geometry} %tamanho letter
\usepackage{graphicx, url}
\usepackage{subfigure,tocloft}

\usepackage{float}


\usepackage[utf8]{inputenc}
\usepackage{mathtools}
\usepackage{siunitx}
\usepackage{pgfplots}
\usepackage{etoolbox}
\usepgfplotslibrary{dateplot}
\pgfplotsset{compat=1.14}
\usepackage{filecontents}
\usepackage{pgfplotstable}
\pgfplotsset{table/search path={csv}}
    
\usepackage{subcaption}

%\setlength{\cftsubsecnumwidth}{0pt}
%\renewcommand{\cftsubsecaftersnumb}{\hspace{1.5em}}

\usepackage{pslatex}  %fonte times new roman
\usepackage{amssymb,fancyhdr,fancybox,epsfig,psfrag,amsmath,tabularx}
\usepackage[paperwidth=8.5in,paperheight=11in,hmargin={25mm,20mm},vmargin={20mm,20mm}]{geometry} %tamanho letter


\usepackage{caption}
%\captionsetup{labelsep=endash}
%\DeclareCaptionLabelSeparator*{cap_travessao}{--}

\usepackage{multirow}
\usepackage[lined,ruled,vlined,linesnumbered,portuguese]{algorithm2e}
\usepackage{ulem,soul}
\usepackage[dvips]{lscape} %rotacionar
\usepackage{setspace}
\usepackage{color}
\usepackage{wrapfig, graphics,multirow}
\usepackage{array}
\usepackage{pdfpages}
\usepackage{enumitem,comment}  
%\usepackage[titletoc,toc,page]{appendix}
\usepackage[titletoc,title]{appendix}

%\usepackage[title]{appendix}
%\newcommand{\source}[1]{\caption*{Source: {#1}}}
%\renewcommand{\appendixtocname}{Ap\^endices}
%\renewcommand{\appendixpagename}{Ap\^endices}

%Center T�tulo do S�mario
%\renewcommand{\cfttoctitlefont}{\normalfont\Large\bfseries\MakeUppercase}
\renewcommand{\cfttoctitlefont}{\hspace*{\fill}\large\bfseries\MakeUppercase}
\renewcommand{\cftaftertoctitle}{\hspace*{\fill}}
%Center T�tulo do 
\renewcommand{\cftlottitlefont}{\hspace*{\fill}\large  \bfseries}
\renewcommand{\cftafterlottitle}{\hspace*{\fill}}
%Center T�tulo da Lista de Figuras 
\renewcommand{\cftloftitlefont}{\hspace*{\fill}\large \bfseries}
\renewcommand{\cftafterloftitle}{\hspace*{\fill}}


%\renewcommand{\listfigurename}{LIST OF FIGURES}
%\renewcommand{\listtablename}{LIST OF TABLES}

%\renewcommand\listtablename{LISTA DE TABELA}



\definecolor{lightgray}{gray}{0.8}
\definecolor{mediumgray}{gray}{0.75}

\usepackage[color]{showkeys}
\definecolor{refkey}{rgb}{0.39,0.58,1}
\definecolor{labeled}{rgb}{1,0,0}
\usepackage[Lenny]{fncychap}
\setlength{\headheight}{15pt}
\setlength{\parindent}{1.2cm}
\usepackage[num,abnt-repeated-author-omit=yes,abnt-emphasize = bf]{abntex2cite}
%=========================================== Headers =========================================

% \renewcommand{\chaptermark}[1]{\markboth{\chaptername\ \thechapter. \ #1}{ }}
\renewcommand{\sectionmark}[1]{\markright{\thesection. \ #1}}
\fancyhead{}
\fancyfoot{}
\fancyhead[RE,RO]{\thepage}
\renewcommand{\headrulewidth}{0pt}
%\fancyhead[RE]{\nouppercase{\leftmark}}
%\fancyhead[LO]{\nouppercase{\rightmark}}
%==============================================================================================

\hyphenation{pro-fis-si-o-nais}

% \input{macros}
\newcolumntype{P}[1]{>{\centering\arraybackslash}p{#1}}


\renewcommand{\listtablename}{LISTA TABELAS} 
\renewcommand{\listfigurename}{LISTA  FIGURAS}

%\defbibheading{bibliography}{\centering The New Name}

%|||||||||| Pacotes usando ||||||||||||||||||||||||||\



%%%% LISTINGS


%\usepackage{minted}

\usepackage{caption}
\usepackage[newfloat]{minted}

\usemintedstyle{vs}
\captionsetup[listing]{position=top}

%%%%%%%%Algoritmos
\usepackage[portuguese, ruled, linesnumbered]{algorithm2e}
\usepackage{smartdiagram}
\usesmartdiagramlibrary{additions}

\usepackage{tikz, tkz-base, tkz-fct, times}
\usetikzlibrary{shapes,backgrounds}
% Suppose we have three circles or ellipses or whatever. Let us define
% commands for their paths since we will need them repeatedly in the 
\usetikzlibrary{mindmap,backgrounds}

\usepackage{pgfplots}


%%%%%%%%%%%%%%%%%%%%%%%%%%%%%%%% Fluxograma
\usetikzlibrary{arrows.meta}
\tikzset{%
  >={Latex[width=2mm,length=2mm]},
  % Specifications for style of nodes:
            base/.style = {rectangle, rounded corners, draw=black,
                           minimum width=4cm, minimum height=1cm,
                           text centered, font=\sffamily},
  activityStarts/.style = {base, fill=blue!30},
       startstop/.style = {base, fill=red!30},
    activityRuns/.style = {base, fill=green!30},
         process/.style = {base, minimum width=2.5cm, fill=orange!15,
                           font=\ttfamily},
}

%\usetikzlibrary{shapes.geometric, arrows}
\usetikzlibrary{arrows, positioning, shapes.geometric}
\tikzstyle{startstop} = [rectangle, rounded corners, minimum width=3cm, minimum height=1cm,text centered, draw=black, fill=red!30]
\tikzstyle{io} = [trapezium, trapezium left angle=70, trapezium right angle=110, minimum width=3cm, minimum height=1cm, text centered, draw=black, fill=blue!30]
\tikzstyle{process} = [rectangle, minimum width=3cm, minimum height=1cm, text centered, draw=black, fill=orange!30]
\tikzstyle{decision} = [diamond, minimum width=1cm, minimum height=1cm,aspect=1.5, text centered, draw=black, fill=green!30]
\tikzstyle{arrow} = [thick,->,>=stealth]


%%%%%%%%%%%%%%%%%%%%%%%%%%%%%%%%%%

%%%%%%Definindo variaveis para o MLP
\usetikzlibrary{positioning}

\tikzstyle{inputNode}=[draw,circle,minimum size=10pt,inner sep=0pt]
\tikzstyle{stateTransition}=[-stealth, thick]
\tikzstyle{stateTransitionDashed}=[dashed, -stealth, thick]

%%%%%%%%%%%%%%%%%%%%%%%%%%%%%%%%%%%%%%%%%%%%%%%%%%%%%%%%%%%%%%%%%%%%%%%%%%
%||||||||||||||||||||||||||||||||||||||||||||||||||||

\usepackage{color, colortbl}
\newcommand{\dbh}[1]{\textcolor{blue}{#1}}
\newcommand{\bac}[1]{\textcolor{red}{#1}}

\begin{document}

\setlength{\cftsubsecindent}{0em}
\setlength{\cftsubsubsecindent}{0em}


\thispagestyle{empty}


%PPC em portugu�s
%================================================================================================
%================================= PRIMEIRA FOLHA INTERNA  ======================================
%================================================================================================
\begin{figure}
\center
\includegraphics[height=0.15\textwidth]{Figs/logoCefetCampusPetropolis.jpg} 
\end{figure}


\vspace*{0.8cm}

\begin{center}
{\large \bf CENTRO FEDERAL DE EDUCAÇÃO TECNOLÓGICA} \vspace{1mm} \\
{\large \bf CELSO SUCKOW DA FONSECA - CEFET/RJ \textit{CAMPUS} PETRÓPOLIS} \vspace{1mm} \\
{\large \bf CURSO: BACHARELADO EM ENGENHARIA DE COMPUTAÇÃO}\\

\vspace*{5cm}
{\large \bf PREVISÃO NEURAL DE TENDÊNCIAS DE VALORES FUTUROS DO BITCOIN}\\
\end{center}
\vspace{4cm}
\hfill
%\begin{minipage}%{0.45\linewidth}
	\begin{flushright}
	Bernardo Botelho Antunes da Costa
	\end{flushright}
%\end{minipage}


\vspace*{3.3cm}
\begin{center}
{\bf PETRÓPOLIS \\ 2018}\\
\end{center}

%--------------------------------------------------------------------------
%--------------------------------------------------------------------------
\newpage
\pagestyle{empty}

\begin{center}
{\large \bf CENTRO FEDERAL DE EDUCAÇÃO TECNOLÓGICA} \vspace{1mm} \\
{\large \bf CELSO SUCKOW DA FONSECA - CEFET/RJ \textit{CAMPUS} PETRÓPOLIS} \vspace{1mm} \\
{\large \bf CURSO: BACHARELADO EM ENGENHARIA DE COMPUTAÇÃO}

\vspace*{3cm}
\normalsize{\large \bf PREVISÃO NEURAL DE TENDÊNCIAS DE VALORES FUTUROS DO BITCOIN}\\
\end{center}
\vspace{1.5cm}
\hfill
%\begin{minipage}%{0.45\linewidth}
	\begin{flushright}
	Bernardo Botelho Antunes da Costa
	\end{flushright}
%\end{minipage}
\vspace*{1.5cm}
\begin{flushright}
	\begin{minipage}{0.5\textwidth}
		{\normalsize
		Trabalho de Conclusão de Curso apresentado ao  
	 CEFET/RJ -{\it campus} Petrópolis, como parte dos requisitos para obtenção do título de Bacharel em Engenharia de Computação.}
	\end{minipage}\\[1.5cm]
\end{flushright}
\vspace{1.5cm}
\hfill
%\begin{minipage}%{0.45\linewidth}
\begin{flushright}
Orientador: Prof. Diego Barreto Haddad, D.Sc.
\end{flushright}




\vspace*{3.3cm}
\begin{center}
{\bf PETRÓPOLIS \\ 2018}\\
\end{center}



%--------------------------------------------------------------------------
%--------------------------------------------------------------------------
\newpage
\begin{minipage}{0.9\textwidth}

{\normalsize 
\begin{center}
Autorizo(amos) a reprodução e divulgação total ou parcial deste trabalho, por qualquer meio
eletrônico ou convencional, para fins de estudo e pesquisa, desde que citada a fonte.
\end{center}
\vspace*{0.5cm}
}
\end{minipage}
\vspace*{5cm}
\begin{center}
\includegraphics[height=0.4\textwidth]{Figs/biblioteca_engcomp.jpeg} 
\end{center}






%--------------------------------------------------------------------------
%--------------------------------------------------------------------------

\newpage
\newcommand{\HRule}{\rule{0.6\linewidth}{0.5mm}}
\pagestyle{empty}

{\center % Center everything on the page


\begin{figure}
\center
\includegraphics[height=0.13\textwidth]{Figs/logoCefetCampusPetropolis.jpg} 
\end{figure}

\begin{center}
{\large \bf CENTRO FEDERAL DE EDUCAÇÃO TECNOLÓGICA} \vspace{1mm} \\
{\large \bf CELSO SUCKOW DA FONSECA - CEFET/RJ \textit{CAMPUS} PETRÓPOLIS} \vspace{1mm} \\
{\large \bf CURSO: BACHARELADO EM ENGENHARIA DE COMPUTAÇÃO}\\
\vspace*{1.2cm}
{\large  FOLHA DE APROVAÇÃO}

\vspace*{1.3cm}
{\large \bf  Previsão Neural de Tendência de Valores Futuros do Bitcoin}\\
\end{center}
\vspace{0.5cm}
\hfill
%\begin{minipage}%{0.45\linewidth}
\begin{flushright}
    Bernardo Botelho Antunes da Costa
	\end{flushright}
%\end{minipage}
\vspace*{0.5cm}
\begin{flushright}
	\begin{minipage}{0.5\textwidth}
		{\normalsize
		Trabalho de Conclusão de Curso apresentado ao  
	 CEFET/RJ -{ {\it campus} Petrópolis}, como parte dos requisitos para obtenção do título de Bacharel em Engenharia de Computação.}
	\end{minipage}\\[0.5cm]
\end{flushright}
\vspace{0.5cm}
\hfill
%\begin{minipage}%{0.45\linewidth}
\begin{flushright}
Orientador: Prof. Diego Barreto Haddad
\end{flushright}

\begin{minipage}{0.9\textwidth}
	\begin{flushleft}
	Aprovado por:
	\end{flushleft}
\end{minipage}\\[1cm]

\center
\HRule \\
Prof. Diego Barreto Haddad, D.Sc. (Orientador) \\[0.4cm]
\HRule \\
Prof. Luís Domingues Tomé Jardim Tarrataca, D.Sc.\\[0.4cm]
\HRule \\
Profa. Laura Silva de Assis, D.Sc.  \\[1.5cm]


\begin{center}
{Agosto de 2018}
\end{center}


}



\newpage


% Dedicat�ria
%\begin{center}
%\textbf{\large DEDICATÓRIA}
%\end{center}
%      \vspace{0.5cm}
%
%Opcional.
%\newpage



% Agradecimento
\begin{center}
\textbf{\large  AGRADECIMENTO}
\end{center}
      \vspace{0.5cm}

Dedico este trabalho ao povo brasileiro que contribuiu de forma significativa a minha formação e estada nesta Universidade. Este projeto é uma pequena forma de retribuir o investimento a mim depositados, de forma compulsória, com a exígua esperança de fomentar o desenvolvimento econômico e tecnológico da nação.



\newpage


% Resumo
\begin{center}
\textbf{\large RESUMO}
\end{center}
      \vspace{0.5cm}
      
 Este trabalho apresenta uma estudo de previsão da série temporal do Bitcoin usando Redes Neurais, utilizando para isso, os dados do valor do preço do Bitcoin e os valores de tendência de pesquisa, que são encontrados no \textit{Google Trends}. Para isso foram utilizados os algoritmos do PCA para redução da dimensão da base de dados, reduzindo a necessidade de poder de processamento; O \textit{Backpropagation} para o calculo dos pesos da Rede; O \textit{K-fold} para escolha da arquitetura da rede a ser usada no modelo. Com o objetivo de reduzir a variância e aumentar a acurácia do sistema, também foi foi utilizada a técnica do \textit{ensemble}.

 Para comparar os resultados obtidos com os resultados apresentados  em trabalhos anteriores, foram utilizados dados no do dia 19 de Agosto de 2013 até 19 de Julho de 2016. 
 
 Como resultado, a abordagem deste trabalho obteve acurácia 53,99\%, utilizando os dados do valor do preço do Bitcoin e o valor do \textit{Google Trends} do termo Bitcoin, enquanto o máximo obtido em trabalhos anteriores foi de 52,78\%, utilizando apenas os dados do valor do preço do Bitcoin, o que significa um aumento de 1,21\%.
 
 Entretanto, o acurácia ainda está muito abaixo de uma valor no qual seria possível obter lucros superiores ao mercado. Tal dificuldade pode ser explicada pela teoria de mercado eficiente, que diz que não seria possível obter lucros acima da média do mercado sem levar em conta a sorte ou o uso de informações privilegiadas.


\begin{flushleft}
{\bf Palavras-chaves:} Previsão, Bitcoin, \textit{Google Trends}, Redes neurais.
\end{flushleft}

\newpage

\begin{center}
\textbf{\large ABSTRACT}
\end{center}
\vspace{0.5cm}

This work presents a prediction study of the Bitcoin time series using Neural Networks, utilizing Bitcoin price values and the trend values of web search, which are enables in Google Trends. For this, we use the  PCA algorithms to reduce the size of the dataset, reducing the demand for processing. We use Backpropagation to compute the network weight. In order to select the best network architecture, we use the K-fold algorithm.  Finally, we use the ensemble technique to reduce variance and increase the accuracy of the system.

 With the objective to compare the results that we obtain with the previous studies results, we use the dataset from August 19, 2013, until July 19, 2016.
 
 As the result of this study, our approach presented by this paper got an accuracy of 53.99\%, applying dataset from Bitcoin's price and  Bitcoin's Google Trends, while the maximum achieved in previous works was 52.78\%, using only Bitcoin's price dataset, which means that we increase the accuracy to 1.21\%.
 
 However, the accuracy is still no good enough to obtain greater profits than to the market. We attribute this difficulty to the efficient market theory whereby should be impossible to outperform the overall market, unless though luck or some insider information.


\begin{flushleft}
{\bf Key-words:} Forecasting, Bitcoin,\textit{Google Trends}, Neural Networks.
\end{flushleft}

\newpage

%=============================== lista de tabelas e figuras ==========================
{\thispagestyle{empty}
\renewcommand{\listfigurename}{LISTA DE FIGURAS}

\listoffigures}
\newpage

{\thispagestyle{empty}
\renewcommand{\listtablename}{LISTA DE TABELAS}
\listoftables}
\newpage

\begin{center}
\textbf{\large LISTA DE SIGLAS}
\end{center}
\vspace{0.5cm}
\singlespacing
\noindent
\begin{tabular}{l c p{.85\linewidth}}
CEFET & - & Centro Federal de Educação Tecnológica \\
TCC & - & Trabalho de Conclusão de Curso \\
PCA & - & Principal Component Analysis \\
AML & - & Amazon Machine Learning \\
ReLU & - & Rectified Linear Unit \\
MSE & - & Mean Square Error \\
BART & - & Bayesian Additive Regression Trees \\
DHC & - & District Heating and Cooling systems \\
ARIMA & - & Autoregressive Integrated Moving Average \\
DAEPF & - & Day-Ahead Electricity Price Forecasting \\
LSTM & - & Long Short Term Memory \\
BTC & - & Bitcoin \\
ETH & - & Ethereum \\
LTC & - & Litecoin \\
XRP & - & Ripple \\
BCH & - & Bitcoin Cash \\
API & - & Application Programming Interface \\
RNN & - & Rede Neurais Recorrentes \\
LSTM & - & Long Short-Term Memory \\
RMS & - & Root Mean Square \\
AWS & - & Amazon Web Services \\
EEM & - & European Energy Market \\
IEEE & - & Instituto de Engenheiros Eletricistas e Eletrônicos \\
ISSN & - & International Standard Serial Number \\
IGARSS & - & International Geoscience and Remote Sensing Symposium  \\
ISBI & - & International Symposium on Biomedical Imaging \\
IPTA & - & International Conference on Image Processing Theory, Tools and Applications \\
EHB & - & E-Health and Bioengineering Conference \\
EECS & - & European Conference on Electrical Engineering and Computer Science \\
ICSG & - & International Istanbul Smart Grids and Cities Congress and Fair \\
ISGT & - & Innovative Smart Grid Technologies \\
ICRIS & - & International Conference on Robots Intelligent System \\
CSUR & - & ACM Computing Surveys \\
\end{tabular}

\onehalfspacing

\newpage

%=============================== sum�rio =============================================

\tableofcontents

\newpage
%caso o sum�rio acabe em uma p�gina �mpar, o verso ser� em branco
%\null \vfill
%\newpage

%Agora muda para o estilo fancy
\pagestyle{fancy}
\pagenumbering{arabic}
\setcounter{page}{12}
\onehalfspacing

\newpage
\section{Introduçao}

 \subsection{Séries Temporais}
 \label{sec:SeriesTemporais}
Séries temporais são definidas como ``qualquer conjunto de observações ordenadas no tempo'' \cite{morettin2006analise}. Elas são observadas em problemas envolvendo diversas áreas de conhecimento, tais como: meteorologia \cite{7982030}, processamento de imagens \cite{7729869,7164182, 6723283}, processamento de vídeos \cite{6469509}, marca d'água digital\footnote{\textit{Digital watermarking} - ou, em português, marca d'água digital é uma técnica esteganográfica de ocultação de informação.} \cite{7024611}, agricultura \cite{6723610}, medicina \cite{6707296} economia \cite{4810671}, entre outros.
 
 \subsection{Tipos de Séries Temporais}
 Existem séries temporais \textit{discretas} e \textit{contínuas}. As discretas costumam ser obtidas através de amostragem de uma série contínua a intervalos de tempo regulares \cite{morettin2006analise,hyndman2018forecasting}. Por exemplo, para analisar a série do valor da temperatura de uma cidade ao longo de um ano, será preciso amostrá-la a intervalos, obtendo uma lista de valores. Esse processo converte uma série contínua em uma série discreta.

Os estudos de séries temporais estão geralmente relacionados a dois domínios: o domínio do tempo e o domínio da frequência. Ambos os enfoques estão interessados em construir modelos para séries. Os modelos no domínio do tempo em geral são paramétricos, ou seja, têm um número finito de parâmetros \cite{conover1981rank}. Já os modelos no domínio da frequência são classificados como não-paramétricos \cite{hollander1999nonparametric}. Em termos mais rigorosos, modelos paramétricos são aqueles que assumem um conjunto finito de parâmetros que cumpre estimar, o que limita a complexidade do modelo mesmo quando a informação contida nos dados é arbitrariamente grande. Já os modelos não paramétricos assumem que a distribuição dos dados depende de um conjunto de parâmetros de dimensionalidade infinita. Estes modelos capturam mais informação dos dados à medida que a disponibilidade de informação aumenta, o que os torna mais flexíveis \cite{ChenNeural2001}.
 
\subsection{Notação}
Este trabalho interpretará uma série temporal como um vetor, cujos elementos consistem de amostras no tempo $t$. Admitindo um total de $r$ elementos, podemos definir o vetor $\boldsymbol{z}(t) \in \mathbb{R}^n$ que contém uma série temporal como
\begin{equation}
\boldsymbol{z}(t) \triangleq \begin{bmatrix}z(t_1) & z(t_2) & \ldots & z(t_n)\end{bmatrix}.
\end{equation}
%\dbh{Prefiro a seguinte notação: vetores em negrito e minúsculas, escalares em fonte normal e matrizes em negrito e em maiúsculas}.
 
 \subsection{Estacionariedade}
 
 Uma série temporal é dita estacionária de primeira ordem quando se desenvolve no tempo aleatoriamente ao redor de uma média constante. A maioria dos procedimentos de análise estatística de séries temporais supõe que estas sejam estacionárias. Caso tal hipótese não se confirme, será necessário efetuar alguma transformação nos dados originais de sorte a realçar a estacionariedade da série transformada \cite{grenander1957statistical}.

 A transformação mais comum consiste em tomar diferenças sucessivas da série original. Esta técnica é conhecida como diferenciação. Seu objetivo consiste em ajudar a estabilizar a média de uma série temporal, removendo as alterações bruscas no nível de uma série temporal e, portanto, eliminando (ou reduzindo) a tendência e a sazonalidade \cite{morettin2006analise, hyndman2018forecasting}.
 

 
 A primeira diferença é definida por:
\begin{equation} 
\delta_1(t) \triangleq z(t) - z(t - 1),
\end{equation}
para $t \in \{2,\ldots,r\}$. A sequência obtida pela segunda diferença pode ser descrita como
\begin{equation}
\delta_2(t) = \delta_1\{\delta_1(t)\} = \delta_1[z(t) - z(t - 1)],
\end{equation}
ou, equivalentemente:
\begin{equation}
\delta_2(t) \triangleq z(t) - 2z(t - 1) + z(t - 2).
\end{equation}
De modo geral, a $n$-ésima diferença de $\boldsymbol{z}(t)$ será:
\begin{equation}
\delta_n\boldsymbol{z}(t) = \delta_1[\delta_{n-1}\boldsymbol{z}(t)].
\end{equation}
Em situações normais, será necessário efetuar uma ou duas diferenças para que a série se torne estacionária \cite{morettin2006analise}.

\subsection{Objetivo da análise de séries temporais}

O processamento de séries temporais, em geral, é empregado para realizarmos procedimentos de previsão, eliminação de ruído, predição e análise de valores \cite{morettin2006analise, hyndman2018forecasting}. Nesse escopo, os principais objetivos da análise de séries temporais são:

\begin{itemize}
\item investigar o mecanismo gerador da série; por exemplo, desejamos entender como os valores de uma determinada série foram gerados;
\item verificação de comportamento; neste caso desejamos identificar a existência de tendências, ciclos e variações sazonais;
\item procurar periodicidades relevantes nos dados;
\item previsão (foco deste trabalho); por exemplo, desejamos usar os valores anteriores para prever qual o próximo valor da série temporal.
\end{itemize}

\subsection{Previsão em Séries Temporais}

O problema de previsão é um dos principais objetivos da análise de séries temporais. Prever o(s) próximo(s) valor(es) de uma série utilizando valores pregressos é um problema que pode ser usado em diversas aplicações, sendo amplamente explorado na literatura \cite{weigend2018time}. Tal técnica é de grande interesse nas mais distintas áreas, com resultados que auxiliam o processo de tomada de decisão \cite{morettin2006analise, hyndman2018forecasting}. Neste trabalho estaremos interessados no problema de previsão de valores de uma cripto-moeda, o Bitcoin \cite{nakamoto2008bitcoin}, cujo problema será explicado nas próximas subseções.

O problema de previsão de séries temporais pode ser descrito como encontrar o valor da série temporal $z(t)$ no instante $t + h$, onde $t$ é o instante de origem e $h$ o \textit{horizonte}. Por exemplo, considere que a série temporal $z(t)$ representa os valores do real frente ao dólar e que suas amostras estão espaçadas no período de um dia. Se quisermos prever o valor do próximo dia do Real frente ao Dólar, teremos $h = 1$ e calcularemos $z(t + 1)$ . Se quisermos prever o valor do real frente ao Dólar na próxima semana, teremos $h = 7$ e estimaremos $z(t + 7)$; e assim por diante. 

Por questão de simplicidade nas notações, iremos denotar a estimativa de $z(t+h)$ como $\hat{z}(h)$. Por exemplo, se a série temporal $z(t)$ representa os valores do real frente ao dólar, para representar o valor desta série na próxima semana (ou seja, $z(t + 7)$, escreveremos simplesmente $\hat{z}(7)$. 

Levando em conta essas notações, o erro quadrático médio\footnote{MSE, do inglês \emph{mean-squared error}.} da previsão pode ser descrito como:
\begin{equation}
\text{MSE} \triangleq \mathbb{E}\left\{\left[z(t + h) - \hat{z}(t+h)\right]^2\right\},
\end{equation}
onde $\mathbb{E}[\cdot]$ é o operador de esperança estatística.

\subsection{Aplicações de previsões de séries temporais}
O estado da arte no tratamento de séries temporais, como foi dito na Seção \ref{sec:SeriesTemporais}, contempla diversas áreas do conhecimento e, consequentemente, diversos conhecimentos específicos de cada área são empregados como suporte ao modelo de previsão proposto. 

O trabalho \cite{8412670}, por exemplo, efetua uma previsão para o pico de carga em circuitos de distribuição. Para fazer previsão da carga de pico são utilizadas árvores de regressão aditiva Bayesiana (BART). O método é comparado com o método de regressão linear múltipla e o método de regressão vetorial de suporte com base na área residencial com dados de carga pública do utilitário do Texas.

Já o trabalho \cite{8412036} implementa uma rede neural para previsão de curto prazo da produção de energia em uma usina fotovoltaica. A estrutura dessas redes neurais é otimizada com um algoritmo genético que seleciona os valores para os principais parâmetros da rede neural e as variáveis usadas como entradas. Essas variáveis de entrada são selecionadas entre um conjunto de variáveis que inclui variáveis meteorológicas, cronológicas, astronômicas e previstas relacionadas à localização da usina.

O trabalho \cite{8413116} explora o problema de otimização do ajuste hierárquico de carga, onde a carga elétrica é organizada hierarquicamente com base na geografia, o que requer previsões hierárquicas que abrangem todos os níveis das operações do sistema de energia. Uma maneira trivial de implementar previsões hierárquicas consiste em gerar previsões de carga em cada nível de forma independente. No entanto, essas previsões geradas de forma independente podem não satisfazer estruturas hierárquicas, ou seja, a soma das previsões de nível inferior não pode ser adicionado exatamente nas previsões de nível superior. Para resolver esse o problema, o trabalho \cite{8413116} apresenta um modelo de programação quadrática (QP) para ajustar otimamente as previsões de carga geradas independentemente em cada nível de uma hierarquia.

As tecnologias para \textit{Smart Grids} também podem ser amparadas mediante o concurso de modelos de previsão de séries temporais. Classicamente, um modelo de \emph{Smart Grid} permite que diferentes fontes de energia participem da rede \cite{8408964}. Além disso, a demanda de carga também varia continuamente, por isso a previsão de carga é de importância crucial para a operação, manutenção e planejamento adequados do sistema de energia elétrica. A referência \cite{8408964} apresenta uma abordagem para esse problema, utilizando Sistemas Adaptativos de Inferência Neuro-difusa.

O trabalho \cite{8410563} investiga como a previsão dos efeitos de ressonância pode impactar um pneu, onde a intensidade de vibração do pneu está relacionada com a taxa de atrito com dependência de longo alcance. As características dinâmicas dos pneus permitem prever a tendência dos sinais de vibração.

Aprendizado de máquina e algoritmos de regressão são usados no trabalho \cite{8403331} para prever cargas de aquecimento e resfriamento em uma rede \textit{district heating and cooling systems}. Os dados são modelados para prever a carga de consumo de aquecimento e resfriamento em uma rede DHC.

O trabalho \cite{8408773} analisa uma série de algoritmos de inteligência artificial, como redes neurais, para prever a velocidade do vento, o que é importante para a indústria no planejamento da operação de sistemas de energia renovável.

A previsão de preços de eletricidade é de grande importância, especialmente na construção do mercado de energia em todo o mundo. O trabalho \cite{8403372} propôs uma abordagem de previsão combinada paralela com o modelo ARIMA \cite{box1970time}, que consiste em ajustar modelos auto-regressivos integrados de médias móveis, redes neurais, entre outros, com base na classificação sazonal para previsão de preço de eletricidade (DAEPF) no dia seguinte. 

A previsão de séries temporais associadas a preço de ações também é um problema recorrente na previsão de séries temporais, explorado por exemplo em \cite{8410278, 7310722}. Eis um problema clássico para a previsão de séries temporais, onde em geral é utilizado o valor do preço de uma ação para estimar quanto será o preço futuro do ativo ou ainda classificar se o ativo irá se valorizar ou não. Abordagens com redes neurais são comuns e foram aplicadas em \cite{8410278}. Já o trabalho \cite{7310722} utiliza algoritmos tradicionais de clusterização para a previsão.

\subsection{Série Temporal: O Bitcoin}
 
 O Bitcoin \cite{nakamoto2008bitcoin} é uma versão \emph{peer-to-peer} de dinheiro eletrônico que permite pagamentos \emph{online} que são enviados diretamente de uma parte para outra sem necessidade de intermédio de uma instituição financeira. 
 
 Para realizar a validação das transações é necessário o emprego de assinaturas digitais que evitam que o mesmo montante de Bitcoins seja enviado para duas pessoas ao mesmo tempo, fenômeno conhecido como \textit{double-spending}. Para  resolver o problema de \textit{ double-spending}, o Bitcoin usa uma rede \emph{peer-to-peer} que salva o horário das transações, calculando o seu HASH (atualmente consoante o padrão Double-SHA256 \cite{bitcoinwikihashcash}), e salvando em uma estrutura de dados que hoje é conhecida como \textit{Blockchain}.
 
O Blockchain, como o próprio nome diz, é uma corrente de blocos, na qual o bloco atual é ligado ao bloco anterior por meio do cálculo de HASH do bloco anterior. 

Para validar o HASH de cada bloco, o Bitcoin usa um \textit{proof-of-work} que consiste em calcular o Double-HASH-256 de cada bloco. Ou seja, o nó deverá calcular qual frase gerará o HASH recebido.

Para executar a rede do Bitcoin os seguintes passos são necessários:
\begin{enumerate}
\item Novas transações são enviadas em \textit{broadcast} para todos os nós da rede;
\item Cada nó transforma a nova transação em um bloco;
\item Cada nó trabalha achando a \textit{proof-of-work} daquele bloco;
\item Quando o nó achar o \textit{proof-of-work}, o resultado é enviado em \textit{broadcast} para todos os nós dentro do novo bloco;
\item Os nós aceitam o bloco apenas se todas as transações nele forem válidas e não foram feitas ainda;
\item Os nós mostram que aceitaram o novo bloco criando o próximo bloco da sua ``corrente'', usando o HASH do bloco que foi aceito.
\end{enumerate}

Nesse trabalho empregaremos a Série Temporal histórica do Bitcoin. O Bitcoin começou a ser negociado em meados de 2009 sem valor comercial, passou a barreira de US\$ 1,00 no começo de 2011, chegando a incríveis US\$ 19.990,00 no final de 2017.

O foco deste trabalho consiste na análise e predição das séries temporais do valor do Bitcoin, possivelmente correlacionadas com a série de tendência de busca do termo ``Bitcoin'' nos mecanismos de busca da internet. Trabalhos anteriores mostraram a forte relação entre essas duas séries. O trabalho \cite{matta2015bitcoin} chegou a mostrar uma  correlação linear cruzada de 64\% entre a série do \textit{Google Trends} e a série do valor do preço do Bitcoin. Além disso, o trabalho apresentado em \cite{wilson_yelowitz_2014} também realizou análises de correlação cruzada que sugeriram uma forte dependência estatística entre as duas séries.

Usando também essa relação, \cite{mcnally2016predicting} obteve acurácia de 52,78\% na previsão do valor do Bitcoin, com o algoritmo \textit{Long Short Term Memory} (LSTM), utilizando dados do valor do preço de fechamento do Bitcoin entre 19 de Agosto de 2013 e 19 de Julho de 2016. Embora esta taxa pareça baixa, cabe ressaltar que a obtenção de uma taxa de acerto ligeiramente acima de 50\% pode, ao longo do tempo, implicar grande retorno financeiro. Esta acurácia de 52,78\% também explicita o quão desafiadora é a tarefa enfocada neste trabalho.


\subsection{Tema}

O tema do trabalho é o desenvolvimento de um algoritmo de inteligência artificial para auxiliar na tomada de decisão de compra ou de venda do Bitcoin, a partir de dados da série temporal do histórico do preço do Bitcoin \cite{nakamoto2008bitcoin} e da quantidade de vezes que o termo foi buscado no Google, utilizando a ferramenta \textit{Google Trends}. O problema a ser resolvido consiste em prever se o preço futuro do Bitcoin irá subir ou descer em determinado espaço de tempo.

O algoritmo proposto, baseado numa estrutura neural treinada pelo concurso do algoritmo \emph{Backpropagation} \cite{hecht1992theory}, pretende superar a acurácia obtida por \cite{mcnally2016predicting}. Cumpre ressaltar que tal algoritmo apresenta a capacidade de treinar as camadas intermediárias da rede neural, calculando o gradiente que será usado para ajustar os pesos das camadas escondidas. Tal procedimento é indicado para identificação de relações não lineares, como as que a série histórica do Bitcoin provavelmente apresenta. Afinal de contas, há diversos fatores que concorrem para aumentar ou reduzir a cotação desta cripto-moeda e seria demasiado ingênuo supor que todos estes fatores possam ser descritos por operadores puramente lineares. 


\subsection{Delimitação}

O objeto de estudo é a Série Temporal do histórico do preço do Bitcoin e como a tendência das buscas do termo "Bitcoin", capturada pelo \textit{Google Trends}, pode treinar uma rede neural classificadora para que ela determine se o valor do preço Bitcoin irá subir ou diminuir.

Existem dezenas de cripto-moedas no mercado atualmente. Esse trabalho focará, em primeiro momento, apenas no Bitcoin, pois busca explorar a ligação que o valor do Bitcoin tem com as tendências de busca do termo "Bitcoin", como mostrado em \cite{matta2015bitcoin}. 

As Figuras \ref{fg:comparacao_bitcoin_12m}, \ref{fg:comparacao_ethereum_12m}, \ref{fg:comparacao_litecoin_12m}, \ref{fg:comparacao_ripple_12m}, \ref{fg:comparacao_bcash_12m} apresentam a comparação entre a série do preço das cripto-moedas Bitcoin \cite{nakamoto2008bitcoin}, Ethereum \cite{Ethereum}, Litecoin \cite{Litecoin}, Ripple \cite{Ripple} e Bitcoin Cash \cite{BitcoinCash}, e os dados do \textit{Google Trends} referentes ao nome das cripto-moedas, nos últimos 12 meses, com frequência de 7 dias, onde o valor é medido em Dólares e o valor do \textit{Google Trends} é medido em percentagem, ou seja, 0\% se no período a busca relativa ao termo foi mínima e 100\% se foi máxima. Essas são as 5 das cripto-moedas mais relevantes da atualidade. 
Analisando os dois gráficos de cada cripto-moeda é possível perceber como os dois gráficos apresentam um formato parecido, especialmente no Bitcoin, que é a cripto-moeda mais famosa. Essa semelhança mostra uma possível relação entre os dois valores, relação esta que será explorada nesse trabalho, por meio das estatísticas oriundas do \textit{Google Trends} para prever o valor da cripto-moeda. É possível perceber que os outros dois gráficos das cripto-moedas restantes não têm uma correlação tão forte como a apresentada pela série do Bitcoin (o que coincide com \cite{matta2015bitcoin}). O Bitcoin, por ser a cripto-moeda mais famosa, também é a que tem mais relação com a quantidade de buscas das pessoas, levando a crer que as outras cripto-moedas são negociadas por nichos mais específicos de compradores, ou seja, a pessoa que vai comprar uma cripto-moeda mais específica, como a Ripple, é uma pessoa com mais conhecimento técnico, não precisando buscar tanto o termo antes de comprar. Já o Bitcoin, como é mais conhecido, oferece diversos meios de compra, de forma mais simples, o que favorece a compra das pessoas mais leigas, colaborando com a premissa de que a quantidade de vezes que o termo ``Bitcoin'' é buscado tem relação com o valor do Bitcoin. Porém, deve ser deixado claro que o modelo proposto pode ser posteriormente usado com qualquer cripto-moeda.

%%%%%%%%%%%%%%%%%%%%%%%%%%%%%%%%%%%%%%%
\begin{center}
	\begin{figure}[H]


	\begin{subfigure}{}
		
		\begin{tikzpicture}

		\begin{axis}[
		    date coordinates in=x,
		    xticklabel={\day-\month-\year},
		    x tick label style={rotate=45,anchor=north east},
		    date ZERO=2017-05-27,
		    xmin=2017-05-27, xmax=2018-05-20,
		    ymin=1000,ymax=21000,
		    height=8cm,
			width=14cm,
			grid=major,
		    xlabel={Data (dia-mês-ano)},
		    ylabel={Valor em dólares},
		]
		\addlegendentry{Preço BTC}
		\addplot [blue, mark=*] table [x=date, y=value, col sep=comma] {bitcoin_coindesk_12m.csv};
		\end{axis}
		\end{tikzpicture}
        \caption{Valor do Bitcoin.}\label{gr:bitcoin_coindesk_12m}
	\end{subfigure}
	%sub fig 2
	\begin{subfigure}{}
		\begin{tikzpicture}

		\begin{axis}[
			title={Valor Bitcoin no Google Trends}, 
		    date coordinates in=x,
		    xticklabel={\day-\month-\year},
		    x tick label style={rotate=45,anchor=north east},
		    xmin=2017-05-27, xmax=2018-05-20,
		    ymin=0,ymax=100,
		    height=8cm,
			width=14cm,
			grid=major,
		    xlabel={Data (dia-mês-ano)},
		    ylabel={Valor \%},  
		]
		\addlegendentry{Trends BTC}
		\addplot [red, mark=*] table [x=date, y=value, col sep=comma] {bitcoin_gtrends_12m.csv};
		\end{axis}
		\end{tikzpicture}
    	\caption{Valor do Bitcoin.}\label{gr:bitcoin_gtrends_12m}
	\end{subfigure}
    \caption{Comparação dos gráfico do valor do Bitcoin e do \textit{Google Trends.}}\label{fg:comparacao_bitcoin_12m}
	\end{figure}
\end{center}
%%%%%%%%%%%%%%%%%%%%%%%%%%%%%%%%%%%%%%%


%%%%%%%%%%%%%%%%%%%%%%%%%%%%%%%%%%%%%%%
\begin{center}
	\begin{figure}[H]


	\begin{subfigure}{}
		
		\begin{tikzpicture}

		\begin{axis}[
		    date coordinates in=x,
		    xticklabel={\day-\month-\year},
		    x tick label style={rotate=45,anchor=north east},
		    date ZERO=2017-05-27,
		    xmin=2017-05-27, xmax=2018-05-20,
		    ymin=170,ymax=1400,
		    height=8cm,
			width=14cm,
			grid=major,
		    xlabel={Data (dia-mês-ano)},
		    ylabel={Valor em dólares},
		]
		\addlegendentry{Preço ETH}
		\addplot [blue, mark=*] table [x=date, y=value, col sep=comma] {ethereum_coindesk_12m.csv};
		\end{axis}
		\end{tikzpicture}
        \caption{Valor do Ethereum.}\label{gr:ethereum_coindesk_12m}
	\end{subfigure}
	%sub fig 2
	\begin{subfigure}{}
		\begin{tikzpicture}

		\begin{axis}[
			title={Valor Ethereum no Google Trends}, 
		    date coordinates in=x,
		    xticklabel={\day-\month-\year},
		    x tick label style={rotate=45,anchor=north east},
		    xmin=2017-05-27, xmax=2018-05-20,
		    ymin=0,ymax=100,
		    height=8cm,
			width=14cm,
			grid=major,
		    xlabel={Data (dia-mês-ano)},
		    ylabel={Valor \%},  
		]
		\addlegendentry{Trends ETH}
		\addplot [red, mark=*] table [x=date, y=value, col sep=comma] {ethereum_gtrends_12m.csv};
		\end{axis}
		\end{tikzpicture}
    	\caption{Valor do Ethereum.}\label{gr:ethereum_gtrends_12m}
	\end{subfigure}
    \caption{Comparação dos gráfico do valor do Ethereum e do \textit{Google Trends}.}\label{fg:comparacao_ethereum_12m}
	\end{figure}
\end{center}
%%%%%%%%%%%%%%%%%%%%%%%%%%%%%%%%%%%%%%%


%%%%%%%%%%%%%%%%%%%%%%%%%%%%%%%%%%%%%%%
\begin{center}
	\begin{figure}[H]


	\begin{subfigure}{}
		
		\begin{tikzpicture}

		\begin{axis}[
		    date coordinates in=x,
		    xticklabel={\day-\month-\year},
		    x tick label style={rotate=45,anchor=north east},
		    date ZERO=2017-05-27,
		    xmin=2017-05-27, xmax=2018-05-20,
		    ymin=20,ymax=320,
		    height=8cm,
			width=14cm,
			grid=major,
		    xlabel={Data (dia-mês-ano)},
		    ylabel={Valor em dólares},
		]
		\addlegendentry{Preço LTC}
		\addplot [blue, mark=*] table [x=date, y=value, col sep=comma] {litecoin_coinmarketcap_12m.csv};
		\end{axis}
		\end{tikzpicture}
        \caption{Valor do Litecoin.}\label{gr:litecoin_coindesk_12m}
	\end{subfigure}
	%sub fig 2
	\begin{subfigure}{}
		\begin{tikzpicture}

		\begin{axis}[
			title={Valor Litecoin no Google Trends}, 
		    date coordinates in=x,
		    xticklabel={\day-\month-\year},
		    x tick label style={rotate=45,anchor=north east},
		    xmin=2017-05-27, xmax=2018-05-20,
		    ymin=0,ymax=100,
		    height=8cm,
			width=14cm,
			grid=major,
		    xlabel={Data (dia-mês-ano)},
		    ylabel={Valor \%},  
		]
		\addlegendentry{Trends LTC}
		\addplot [red, mark=*] table [x=date, y=value, col sep=comma] {litecoin_gtrends_12m.csv};
		\end{axis}
		\end{tikzpicture}
    	\caption{Valor do Ethereum.}\label{gr:litecoin_gtrends_12m}
	\end{subfigure}
    \caption{Comparação dos gráfico do valor do Litecoin e do \textit{Google Trends}.}\label{fg:comparacao_litecoin_12m}
	\end{figure}
\end{center}
%%%%%%%%%%%%%%%%%%%%%%%%%%%%%%%%%%%%%%%
 
 
 %%%%%%%%%%%%%%%%%%%%%%%%%%%%%%%%%%%%%%%
\begin{center}
	\begin{figure}[H]


	\begin{subfigure}{}
		
		\begin{tikzpicture}

		\begin{axis}[
		    date coordinates in=x,
		    xticklabel={\day-\month-\year},
		    x tick label style={rotate=45,anchor=north east},
		    date ZERO=2017-05-27,
		    xmin=2017-05-27, xmax=2018-05-20,
		    ymin=0,ymax=3,
		    height=8cm,
			width=14cm,
			grid=major,
		    xlabel={Data (dia-mês-ano)},
		    ylabel={Valor em dólares},
		]
		\addlegendentry{Preço XRP}
		\addplot [blue, mark=*] table [x=date, y=value, col sep=comma] {ripple_investing_12m.csv};
		\end{axis}
		\end{tikzpicture}
        \caption{Valor do Ripple.}\label{gr:ripple_coindesk_12m}
	\end{subfigure}
	%sub fig 2
	\begin{subfigure}{}
		\begin{tikzpicture}

		\begin{axis}[
			title={Valor Ripple no Google Trends}, 
		    date coordinates in=x,
		    xticklabel={\day-\month-\year},
		    x tick label style={rotate=45,anchor=north east},
		    xmin=2017-05-27, xmax=2018-05-20,
		    ymin=0,ymax=100,
		    height=8cm,
			width=14cm,
			grid=major,
		    xlabel={Data (dia-mês-ano)},
		    ylabel={Valor \%},  
		]
		\addlegendentry{Trends XRP}
		\addplot [red, mark=*] table [x=date, y=value, col sep=comma] {ripple_gtrends_12m.csv};
		\end{axis}
		\end{tikzpicture}
    	\caption{Valor do Ripple.}\label{gr:ripple_gtrends_12m}
	\end{subfigure}
    \caption{Comparação dos gráfico do valor do Ripple e do \textit{Google Trends}.}\label{fg:comparacao_ripple_12m}
	\end{figure}
\end{center}
%%%%%%%%%%%%%%%%%%%%%%%%%%%%%%%%%%%%%%%
 %%%%%%%%%%%%%%%%%%%%%%%%%%%%%%%%%%%%%%%
\begin{center}
	\begin{figure}[H]


	\begin{subfigure}{}
		
		\begin{tikzpicture}

		\begin{axis}[
		    date coordinates in=x,
		    xticklabel={\day-\month-\year},
		    x tick label style={rotate=45,anchor=north east},
		    date ZERO=2017-05-27,
		    xmin=2017-05-27, xmax=2018-05-20,
		    ymin=200,ymax=3000,
		    height=8cm,
			width=14cm,
			grid=major,
		    xlabel={Data (dia-mês-ano)},
		    ylabel={Valor em dólares},
		]
		\addlegendentry{Preço BCH}
		\addplot [blue, mark=*] table [x=date, y=value, col sep=comma] {bcash_investing_12m.csv};
		\end{axis}
		\end{tikzpicture}
        \caption{Valor do Bitcoin Cash.}\label{gr:bcash_coindesk_12m}
	\end{subfigure}
	%sub fig 2
	\begin{subfigure}{}
		\begin{tikzpicture}

		\begin{axis}[
			title={Valor Bitcoin Cash no Google Trends}, 
		    date coordinates in=x,
		    xticklabel={\day-\month-\year},
		    x tick label style={rotate=45,anchor=north east},
		    xmin=2017-05-27, xmax=2018-05-20,
		    ymin=0,ymax=100,
		    height=8cm,
			width=14cm,
			grid=major,
		    xlabel={Data (dia-mês-ano)},
		    ylabel={Valor \%},  
		]
		\addlegendentry{Trends BCH}
		\addplot [red, mark=*] table [x=date, y=value, col sep=comma] {bcash_gtrends_12m.csv};
		\end{axis}
		\end{tikzpicture}
    	\caption{Valor do Bitcoin Cash.}\label{gr:bcash_gtrends_12m}
	\end{subfigure}
    \caption{Comparação dos gráfico do valor do Bitcoin Cash e do \textit{Google Trends}}\label{fg:comparacao_bcash_12m}
	\end{figure}
\end{center}
\subsection{Objetivo}
O objetivo deste trabalho reside na construção de um modelo neural, capaz de modelar relações não lineares, possivelmente presentes na série histórica do Bitcoin. O algoritmo base é o \emph{Backpropagation} \cite{hecht1992theory} que tem a capacidade de treinar as camadas intermediárias. Utilizaremos como fonte de dados para previsão da subida ou descida do valor do preço do Bitcoin os dados do Google Trends e o valor do preço do próprio Bitcoin. É esperado que o algoritmo aumente a acurácia vista na literatura, além de ser disponibilizado como uma API que possa ser usada por outros sistemas. O algoritmo que será proposto deverá categorizar o valor do preço futuro do Bitcoin em duas categorias, as quais serão a saída do algoritmo:
\begin{enumerate}
\item o valor da moeda aumentará, quando comparada ao seu valor anterior (COMPRE);
\item o valor da moeda diminuirá, quando comparada ao seu valor anterior (VENDA).
\end{enumerate}

\subsection{Metodologia}

Este trabalho irá utilizar as conclusões apresentadas em  \cite{matta2015bitcoin, mcnally2016predicting,weigend2018time}, que sugerem a relação entre a quantidade de vezes que o termo ``Bitcoin'' é buscado no Google e o valor do preço do Bitcoin. Os dados de entrada serão os valores da série histórica do Bitcoin, disponíveis no site Investing \cite{Investing}, e os valores referentes a quantidade de vezes que o termo "Bitcoin" é buscado no Google, disponíveis através da ferramenta \textit{Google Trends} \cite{GoogleTrends}. Os dados obtidos serão baixados para serem trabalhados, a princípio, de modo \textit{offline} para elaboração do algoritmo.

Os dados serão divididos em três partes:
\begin{enumerate}
\item base de treinamento: será utilizada para treinamento supervisionado do algoritmo;
\item base de testes: será utilizada pra testar o algoritmo treinado anteriormente;
\item base de previsão: será utilizada para testar o nível de acerto real do algoritmo.
\end{enumerate}

O algoritmo utilizado será o Backpropagation \cite{hecht1992theory}, que tem a capacidade de treinar as camadas intermediárias, através do cálculo o gradiente em relação aos coeficientes das camadas intermediárias. Antes do treinamento será usado o algoritmo do \textbf{PCA} \cite{Jolliffe:1986} (do inglês \emph{Principal Component Analysis}) para compressão dos dados, com objetivo de aumentar a velocidade de aprendizado do algoritmo. A rede neural deverá retornar a classificação da previsão dos valores do preço do Bitcoin em duas classes: o preço irá subir ou descer. Essa informação deverá auxiliar um investidor a tomar a decisão de compra ou de venda da cripto-moeda.


Os dados de treinamento serão utilizados para estimar o número de neurônios na camada escondida e o número de amostrar pregressas, que são variáveis no algoritmo Backpropagation, de forma a otimizar os resultados. 

Utilizaremos outros algoritmos, como o \textbf{RNN} (Rede Neurais recorrentes), \textbf{LSTM}  e \textbf{Análise de Regressão Linear}, para avaliar a capacidade de previsão do algoritmo proposto, onde o objetivo é superar os resultados obtidos por esses algoritmos, em especial em \cite{mcnally2016predicting}.
\section{Redes Neurais}
\subsection{O que são Redes Neurais?}
\label{subsec:oquesaoredesneurais}

As redes neurais artificiais, como conhecidas na literatura \cite{haykin2004comprehensive, haykin2009neural, lecun2015deep}, são estruturas aptas ao aprendizado inspiradas na forma com que os neurônios de cérebros reais funcionam. Um modo estrutural de organização, bem como a distribuição de informação ao longo da arquitetura foram levados em conta na construção deste poderoso modelo matemático que hoje conhecemos como Redes Neurais \cite{haykin2004comprehensive, haykin2009neural, lecun2015deep}. Tais redes são reconhecidamente uma tecnologia bioinspirada, já que são profundamente motivadas pelo aprendizado demonstrado pelo cérebro de seres vivos, notadamente o dos seres humanos. 
\begin{figure}[b]
    \centering
    \includegraphics[scale=0.17]{Figuras/Cap2/neuronio.png}
        \caption{Representação de um neurônio. Figura \ref{fig:neuronio} extraída do site \cite{Pixabay} e editada pelo autor. O site \cite{Pixabay} é uma plataforma gratuita de distribuição de imagens, ou seja, o uso das imagens não ferem direitos autorais.}
        \label{fig:neuronio}
\end{figure}


A Figura \ref{fig:neuronio} representa um neurônio utilizado como componente fundamental das redes neurais artificiais. Para entender como essa representação nos leva ao modelo matemático, devemos primeiro descrever suas quatro partes principais, bem como a função de cada parte do neurônio.

\begin{enumerate}
    \item núcleo: onde é processada a informação;
    \item dendrito: de onde vem a informação a ser processada (ou seja, o \textit{input});
    \item axônio: parte que o neurônio usa para enviar sinais (ou seja, o \textit{output});
    \item terminal do axônio: onde a informação de saída é propagada para os demais neurônios da rede ou utilizada como resposta final da rede.
\end{enumerate}

Uma rede neural artificial é implementada por meio da conexão de diversos neurônios, cujos axônios se conectam ao dendritos de outro, levando o sinal processado pelo núcleo. Os neurônios recebem tal sinal em seus dendritos os quais, por sua vez, processam o sinal em seus núcleos, enviando novamente o resultado (via axônios) para os próximos neurônios, por meio dos terminais dos axônios. Eis a lógica de propagação e processamento de informação que as redes neurais artificiais almejam emular.

Este trabalho tem como motivação o uso de Redes Neurais pelos benefícios que elas trazem para o problema específico da previsão de valores do Bitcoin. A motivação biológica não faz parte das motivações deste trabalho.

A próxima subseção apresentará a representação gráfica e matemática de uma Rede Neural, que será usada no decorrer deste trabalho.

\subsection{Representação de uma Rede Neural}

\subsubsection{Representação de um Neurônio}
\label{subsubsec:representacaoumneuronio}

No objetivo de representar mais formalmente o processamento matemático empreendido por uma Rede Neural, vamos transformar as quatro principais partes do neurônio da subseção \ref{subsec:oquesaoredesneurais} em um diagrama. A Figura \ref{fg:rede_neural_simples} apresenta o diagrama de um neurônio, no qual podemos elencar os componentes seguintes:
\begin{enumerate}
 \item dendrito (\textit{input}): representado pelos círculos azuis. Cada círculo contém o valor referente à sua amostra $x_n$; nesse exemplo, $n = 3$ amostras. A informação de cada uma das $n$ amostras é enviada para o núcleo;
    \item núcleo: representado pelo círculo laranja. No núcleo, as amostras $x_n$ e os pesos $\theta_n$ executam a chamada \textbf{função de ativação}, apresentada na Seção \ref{sec:funcaoativacao};
   
    \item axônio (\textit{output}): consiste na saída do núcleo, a qual armazena o resultado da operação;
    \item terminal do axônio: descreve a transformação da saída num (\textit{input}) de um novo neurônio, geralmente da camada posterior.
    
\end{enumerate}


\begin{figure}
\centering
\large
\begin{tikzpicture}[shorten >=1pt,->,draw=black!70, node distance=\layersep]
    \tikzstyle{every pin edge}=[<-,shorten <=1pt]
    \tikzstyle{neuron}=[circle,fill=black!50,minimum size=05pt,inner sep=15pt]
    \tikzstyle{input neuron}=[neuron, fill=blue!50];
    \tikzstyle{output neuron}=[neuron, fill=orange!50, inner sep=10pt];

    \tikzstyle{annot} = [text width=4em, text centered]

    % Draw the input layer nodes
    \foreach \name / \y in {1,...,3}
    % This is the same as writing \foreach \name / \y in {1/1,2/2,3/3,4/4}
        \node[input neuron, pin=left:] (I-\name) at (0,-2.2*\y){x\y};

    % Draw the output layer node
    \node[output neuron,pin={[pin edge={->}]right:}, right of=I-1] (O) at (0,-4.4) { $h_\theta (x)$};

    % Connect every node in the hidden layer with the output layer
    \foreach \source in {1,...,3}
        \path (I-\source) edge (O);

    % Annotate the layers
   \node[annot,above of=I-1, node distance=1.7cm] (hl) {Entrada};
    \node[annot,right of=hl] {Saída};
    
    
\end{tikzpicture}
\caption{Representação de um neurônio}
\label{fg:rede_neural_simples}
\end{figure}

Os elementos desse neurônio podem ser escritos na forma matricial. Assumindo uma entrada $x$ composta de $n$ elementos, podemos representá-la como:
\begin{equation}
  \mathbf{x}^T \triangleq \begin{bmatrix}x_{1}&x_{2}&x_{3}&...&x_{n-1}&x_{n} \end{bmatrix}
\end{equation}
 com os pesos representados como:
 \begin{equation}
  \mathbf{\theta}^T \triangleq \begin{bmatrix}\theta_{1}&\theta_{2}&\theta_{3}&...&\theta_{n-1}&\theta_{n}\end{bmatrix}
\end{equation}

Frequentemente na literatura \cite{haykin2004comprehensive, haykin2009neural, lecun2015deep} é adicionado em cada neurônio uma amostra extra, chamada de \textit{bias}, ou viés. O \textit{bias} serve para aumentar os graus de liberdade do sistema, levando o resultado do Neurônio para uma determinada direção. Ele permite que restrições muito severas não sejam aplicadas ao sistema. Por exemplo, caso as entradas sejam nulas, o neurônio terá uma saída também nula, o que comumente é uma restrição muito forte. O  \textit{bias}, portanto,  permite uma melhor adaptação do sistema.

%FEITO
%\dbh{Errado, Bernardo! Esse bias é essencial, porque caso contrário a rede neural necessariamente irá jogar uma entrada nula numa saída também nula, o que comumente é uma restrição muito forte!!!}

\subsubsection{Representação de uma Rede Neural}
\label{subsubsec:representacaoumaredeneural}
Nas seção \ref{subsubsec:representacaoumneuronio} foi apresentado a representação de um Neurônio. Nesta seção, será empregada a representação da seção anterior para definir uma Rede Neural, que será usada como modelo no restante deste trabalho. A expansão da representação de \ref{subsubsec:representacaoumneuronio} permite uma descrição global de uma rede neural artificial. A Figura \ref{fg:rede_neural_generica} representa uma Rede Neural. Essa rede possui $n$ componentes de entrada, $m$ neurônios em cada uma das $k$ camadas, onde $n, m, k\in \mathbb{N}$. Cada neurônio aplica uma função não linear nas entradas que recebe; normalmente tal função pode ser a sigmoidal (\ref{subsec:sigmoid}), a tangente hiperbólica (\ref{subsec:htan}) ou a ReLu (\ref{subsec:relu}), que serão abordadas especificamente na seção \ref{sec:funcaoativacao}.

%--

\begin{figure}
\centering
\large
\begin{tikzpicture}[shorten >=1pt,->,draw=black!50, node distance=\layersep]
    \tikzstyle{every pin edge}=[<-,shorten <=1pt]
    \tikzstyle{neuron}=[circle,fill=black!25,minimum size=17pt,inner sep=15pt]
    \tikzstyle{input neuron}=[neuron, fill=blue!50];
    \tikzstyle{output neuron}=[neuron, fill=orange!50,inner sep=13pt];
    \tikzstyle{hidden neuron}=[neuron, fill=red!50,inner sep=13pt];
     \tikzstyle{hidden_n neuron}=[neuron, fill=red!30,inner sep=10pt];
     \tikzstyle{p neuron}=[neuron, fill=red!40,inner sep=17pt];
    \tikzstyle{annot} = [text width=4em, text centered]
    
    \node[input neuron, pin=left:] (I-1) at (-5,-2.2*1) {$x_1$};
    \node[input neuron, pin=left:] (I-2) at (-5,-2.2*2) {$x_2$};
    \node[input neuron, pin=left:] (I-3) at (-5,-2.2*3) {$x_3$};
    \node[input neuron, pin=left:] (I-4) at (-5,-2.2*4) {...};
    \node[input neuron, pin=left:] (I-5) at (-5,-2.2*5) {$x_n$};
    
    \node[hidden neuron] (H-1) at (-2,-2.2*1) {$a_1^2$};
    \node[hidden neuron] (H-2) at (-2,-2.2*2) {$a_2^2$};
    \node[hidden neuron] (H-3) at (-2,-2.2*3) {$a_3^2$};
    \node[hidden neuron] (H-4) at (-2,-2.2*4) { ... };
    \node[hidden neuron] (H-5) at (-2,-2.2*5) {$a_m^2$};
    
    \node[p neuron] (p-1) at (1,-2.2*1) { ... };
    \node[p neuron] (p-2) at (1,-2.2*2) { ... };
    \node[p neuron] (p-3) at (1,-2.2*3) { ... };
    \node[p neuron] (p-4) at (1,-2.2*4) { ... };
    \node[p neuron] (p-5) at (1,-2.2*5) { ... };

    \node[hidden_n neuron] (Hn-1) at (4,-2.2*1) {$a_1^{k-1}$};
    \node[hidden_n neuron] (Hn-2) at (4,-2.2*2) {$a_2^{k-1}$};
    \node[hidden_n neuron] (Hn-3) at (4,-2.2*3) {$a_3^{k-1}$};
    \node[hidden_n neuron] (Hn-4) at (4,-2.2*4) { ... };
    \node[hidden_n neuron] (Hn-5) at (4,-2.2*5) {$a_{m}^{k-1}$};
    
    \node[output neuron,pin={[pin edge={->}]right:}] (O-1) at (7,-2.2*1) { $h_{\theta_1}$};
    \node[output neuron,pin={[pin edge={->}]right:}] (O-2) at (7,-2.2*2) { $h_{\theta_2}$};
    \node[output neuron,pin={[pin edge={->}]right:}] (O-3) at (7,-2.2*3) { $h_{\theta_3}$};
    \node[output neuron,pin={[pin edge={->}]right:}] (O-4) at (7,-2.2*4) { ... };
    \node[output neuron,pin={[pin edge={->}]right:}] (O-5) at (7,-2.2*5) { $h_{\theta_m}$};
    
    \foreach \source in {1,...,5}
        \foreach \dest in {1,...,5}
            \path (I-\source) edge (H-\dest);

    \foreach \source in {1,...,5}
        \foreach \dest in {1,...,5}
            \path (H-\source) edge (p-\dest);
            
    \foreach \source in {1,...,5}
        \foreach \dest in {1,...,5}
            \path (p-\source) edge (Hn-\dest);
    
    \foreach \source in {1,...,5}
        \foreach \dest in {1,...,5}
            \path (Hn-\source) edge (O-\dest);

    \node[annot] at (-5,-0.5) {Entradas};
    \node[annot] at (-2,-0.5){Camada 1};
    \node[annot] at (1, -0.5){...};
    \node[annot] at (4, -0.5){Camada k-1};
    \node[annot] at (7, -0.5){Camada k};
\end{tikzpicture}
\caption{Representação de uma Rede Neural}
\label{fg:rede_neural_generica}
\end{figure}


Para representar a Rede Neural, primeiramente, é preciso  representar $a^{(x)}_y$, que é chamada de \textbf{ativação}. A ativação é individual para cada Neurônio, onde $a^{(x)}_y$ é a ativação referente ao Neurônio da camada $x$, ficando a cargo de $y$ identificar o Neurônio dentro de todos os possíveis Neurônios da camada $x$. A \textbf{ativação} é calculado realizando a soma de todas as entradas de determinado Neurônio, multiplicadas pelos seus respectivos pesos, que servem de entrada para a função de ativação, que é o resultado do Neurônio em questão. Feito esse processo, podemos representar $a^{(x)}_y$ usando a equação \ref{fun:axy}.
\begin{equation}
    a^{(x)}_y=g(\Theta^{(x-1)}_{y0}x{}_0+\Theta^{(x-1)}_{y1}x{}_1+\Theta^{(x-1)}_{y2}x{}_2\Theta^{(x-1)}_{y3}x{}_3+...+\Theta^{(x-1)}_{y(m-1)}x{}_{m-1}+\Theta^{(x-1)}_{ym}x{}_m )
    \label{fun:axy}
\end{equation}


Sabendo como calcular os $a^{(x)}_y$ de todas as camadas, deixando claro que é preciso saber o resultado da camada anteriora para calcular a camada atual, é possível definir uma saída genérica $\mathbf{h}{\Theta}(x)_y$ como na  Eq. \ref{fun:thetanm}. Note que a Eq. \ref{fun:thetanm} é apenas consequência de ir executando a Eq.  \label{fun:axy} até a última camada. Esse procedimento é conhecido como \textit{Forward Propagagtion}, e será abordado melhor na seção \ref{subsec:backpropagation} ao falar de Backpropagation.

\begin{equation}
    \mathbf{h}{\Theta}(x) = a^{(n)}_y = g(\Theta^{n-1}_{y0}a^{n-1}_{0}+\Theta^{n-1}_{y1}a^{n-1}_{1}+\Theta^{n-1}_{y2}a^{n-1}_{2}+...+\Theta^{n-1}_{y(n-1)}a^{n-1}_{n-1}+\Theta^{n-1}_{yn}a^{n-1}_{m})
     \label{fun:thetanm}
\end{equation}


Para simplificar os cálculos, é usual utilizar operações matriciais, que conseguem atualizar todos os seus componentes em uma única operação. Para representar todas as entradas, por exemplo, foi utilizado a Eq. \ref{fun:entrada_generica}.
%FEITO
%\dbh{O texto está muito seco; só definições matemáticas, as quais não apelam à intuição!}


\begin{equation}
    \label{fun:entrada_generica}
  \mathbf{x}^T = \begin{bmatrix}x_{1}&x_{2}&x_{3}&...&x_{n-1}&x_{n}\end{bmatrix}
\end{equation}

E a resposta da Rede Neural Genérica pode ser escrita na forma matricial mediante o emprego de \eqref{fun:h_generica}.

 \begin{equation}
   \label{fun:h_generica}
   \mathbf{h}\theta(x) = 
   \begin{bmatrix}
   g(\Theta^{(n-1)}_{10}a^{(n-1)}_{0}+\Theta^{(n-1)}_{11}a^{(n-1)}_{1}+ ... +\Theta^{(n-1)}_{1(m-1)}a^{(n-1)}_{(m-1)} )\\
   g(\Theta^{(n-1)}_{20}a^{(n-1)}_{0}+\Theta^{(n-1)}_{21}a^{(n-1)}_{1}+ ... +\Theta^{(n-1)}_{2(m-1)}a^{(n-1)}_{(m-1)} )\\
   g(\Theta^{(n-1)}_{30}a^{(n-1)}_{0}+\Theta^{(n-1)}_{31}a^{(n-1)}_{1}+ ... +\Theta^{(n-1)}_{3(m-1)}a^{(n-1)}_{(m-1)} )\\
    \vdots\\
    g(\Theta^{(n-1)}_{(m-1)0}a^{(n-1)}_{0}+\Theta^{(n-1)}_{(m-1)1}a^{(n-1)}_{1}+ ... +\Theta^{(n-1)}_{(m-1)}a^{(n-1)}_{(m-1)} )\\
    
   \end{bmatrix} 
\end{equation}


Essas são as representações gráficas e matemáticas que serão utilizadas no decorrer deste trabalho. Na próxima subseção será abordada a função de ativação.

\subsection{Funções de Ativação}
\label{sec:funcaoativacao}

\cite{activationfun}

\dbh{Motivar o emprego da função não-linear; a não linearidade é muito importante para o poder das redes neurais.}

\bac{motivar mais as funcoes de ativacao **ReLU copiada do guilherme}

A função de ativação é denotada como $h_\Theta (x)$ e define a saída de um neurônio para uma dada entrada $x$. Descreveremos a seguir a função de ativação sigmoide, a mais comum em Redes Neurais \cite{haykin2004comprehensive, haykin2009neural, lecun2015deep}.

\subsubsection{Função Sigmoide}
\label{subsec:sigmoid}
\dbh{Não, mil vezes não! A sigmoide é suave, não é brusca como mostrado na equação (23)! Este capítulo tem que ser muito lapidado. Só tem definições matemáticas, com pouca discussão motivadora! Ler mais!}

A Eq.\eqref{fun:sigmoid} representa a função sigmoide, apresentada graficamente na Fig. \ref{fg:funcao_sigmoide}. 


\begin{equation}
  f(x) =  \frac{\mathrm{1} }{\mathrm{1} + e^{- \mathbf{\theta^T}x}}
  \label{fun:sigmoid}
\end{equation}

\begin{center}
    \begin{figure}
        \centering
        \begin{tikzpicture}
        	\begin{axis}[
        		xlabel=$x$,
        		ylabel={$f(x)$}
        	]
        	\addplot {1 / (1 + e^-x)};
        	\end{axis}
        \end{tikzpicture}
    \caption{Função Sigmoide}\label{fg:funcao_sigmoide}
	\end{figure}
\end{center}

\subsubsection{Função Tangente Hiperbólica}
\label{subsec:htan}
\bac{TANGENTE HIPERBOLCIA}

\begin{equation}
  f(x) =  \frac{\mathrm{1} - e^{- \mathbf{\theta^T}x}}{\mathrm{1} + e^{- \mathbf{\theta^T}x}}
  \label{fun:sigmoid}
\end{equation}

\begin{center}
    \begin{figure}
        \centering
        \begin{tikzpicture}
        	\begin{axis}[
        		xlabel=$x$,
        		ylabel={$f(x)$}
        	]
        	\addplot {(e^x - e^-x) / (e^x + e^-x)};
        	\end{axis}
        \end{tikzpicture}
    \caption{Função Tangente Hiperbólica}\label{fg:funcao_tanh}
	\end{figure}
\end{center}



\subsubsection{Função ReLU}
\label{subsec:relu}
A função ReLU (\textit{rectified linear unit}) realiza a ativação do nó apenas se a entrada estiver acima de um valor predeterminado. O ReLU é representado pela equação $\phi (x) = max(0,x)$. A Figura \ref{fig:relu} mostra o gráfico dessa função. Trata-se de uma função não linear (ainda que linear por partes) que permite implementações com baixo custo computacional.

\begin{figure}[!htb]
    \centering
    
    \begin{tikzpicture}
    \begin{axis}[
        		xlabel=$x$,
        		ylabel={$f(x)$}
        	]
        \addplot+[mark=none,blue,domain=-3:0] {0};
        \addplot+[mark=none,blue,domain=0:4] {x};
    \end{axis}
\end{tikzpicture}
\caption{Função ReLU. \label{fig:relu}}
\end{figure}
\section{Pré-processamento}

\subsection{Base de Dados}
\label{subsec:base_de_dados}
 Com o objetivo de alimentar o modelo proposto foram utilizados dois tipos de dados: o do valor do preço do Bitcoin, fornecidos pela Coinbase \cite{Coinbase}, e o do valor do Google Trends, fornecidos pelo próprio Google. Com esses dados foi possível criar os dados que treinaram e testaram o modelo proposto.
 
Para comparar os resultados obtidos com o o resultados apresentados no trabalho \cite{wilson_yelowitz_2014}, foram utilizados dados no do dia 19 de Agosto de 2013 até 19 de Julho de 2016, que é o mesmo intervalo utilizado pelo trabalho \cite{wilson_yelowitz_2014}.
 
 Os gráficos dos dados utilizados, tanto do valor do preço do Bitcoin como do valor do Google Trends, podem ser vistos na Figura \ref{fg:comparacao_btc}.
 
 \subsection{Normalização}
 \label{subsec:normalizacao}
A normalização dos dados é uma técnica utilizada antes dos dados serem inseridos para a Rede Neural. Essa técnica consiste em subtrair cada amostra original pela média das amostras e dividir pelo desvio padrão das amostras. O objetivo dessa técnica é suavizar os dados, ou seja, remover picos, ou anomalias, dos dados \cite{quackenbush2002microarray}. Ademais, ela é essencial para evitar saturações nos valores de saída dos neurônios das camadas intermediárias.

A equação utilizada para a normalização pode ser vista em \eqref{fun:normalizacao}, onde: $x$ é o valor original; $\overline{m}$ é média da coluna referente a amostra; e $\sigma$ é o desvio padrão da coluna referente a amostra. \dbh{você poderia mostrar o cálculo do desvio padrão}

\begin{equation}
    \label{fun:normalizacao}
 \mathbf{X'} = (\mathbf{X}-\overline{m})/\sigma.
\end{equation}

Para exemplificar o que a técnica de normalização faz com o dados, os gráficos dos dados normalizados, tanto do valor do preço do Bitcoin como do valor do Google Trends, podem ser vistos na Figura \ref{fg:comparacao_btc_n}.

\subsection{PCA}

O PCA, do inglês \textit{Principal Component Analysis}, é um algoritmo que tem como objetivo principal reduzir a dimensão dos dados, de modo a perder o mínimo de informações, aliviando assim a necessidade de poder de processamento para computar a mesma informação. De modo mais genérico, o PCA permite reduzir a dimensão de uma base de dados de $d$ para dimensão $d'$, onde $0<d'<=d$ \cite{Jolliffe}. O PCA empreende uma projeção num subespaço de dimensão inferior, cuja distância quadrática média é mínima (em termos de projeções lineares). O fato de as componentes oriundas da decomposição PCA serem descorrelacionadas também concorre para melhorar o processo de aprendizado da rede neural.

Para realizar a redução na dimensão dos dados mantendo a essência da informação, o PCA busca relacionar os pontos da base de dados de dimensão $d$ com um espaço de dimensão $d'$, representando os dados num espaço de dimensão menor do que a do espaço original. Por exemplo: considere uma base de dados de dimensão igual a $2$ que contém diversos pontos $P_n$. Para reduzir a dimensão desses dados para $1$, o algoritmo do PCA traça a reta (que tem uma dimensão igual a $1$, ou seja, menor que a dimensão original) que minimiza a distância de cada um dos pontos $P_n$ até essa reta. Traçada essa reta, é preciso mapear cada um dos pontos $P_n$ na reta. Para isso, o algoritmo marca na reta o ponto que tiver a menor distância para cada ponto $P_n$, que será a sua nova representação. Ao término desse processo, os dados estarão representados em uma base de dados com uma dimensão inferior à da base de dados original. O mesmo processo é realizado para reduzir uma base de dados de $3$ para $2$ dimensões. Neste caso, o espaço de $3$ dimensões é mapeado em um plano de duas dimensões. Generalizando, o algoritmo PCA mapeia pontos de uma base de dados de dimensão $d$ para dimensão $d'$, onde $0<d'<=d$, calculando a menor distância entre os pontos originais e o espaço com dimensão inferior, sucessivamente até a dimensão $d'$ requerida. 

O algoritmo PCA se baseia na premissa de que os dados estão normalizados, então é preciso executar a normalização apresentada na seção \ref{subsec:normalizacao}. Feito isso, é possível aplicar uma série de operações matemáticas com os dados, apresentadas no algoritmo do PCA. A prova matemática desse algoritmo para o cálculo do PCA foge ao escopo deste trabalho, mas pode ser encontrada em \cite{Jolliffe}.

O Algoritmo abaixo apresenta os passos para o cálculo do PCA, recebendo como entrada a base de dados original, $\mathbf{X}$, e a dimensão desejada, $d'$. O Algoritmo executa os seguintes passos:  
\begin{enumerate}
    \item calcula-se a dimensão de $\mathbf{X}$, $d$;
    \item calcula-se a matriz de covariância, $\Sigma$;
    \item calcula-se o autovalor, $\mathbf{U}$, de $\Sigma$;
    \item seleciona-se as $d'$ primeiras colunas de $\mathbf{U}$, as quais denominaremos de $\mathbf{U}_{\text{reduzido}}$;
    \item calcula-se a resposta, $z$, que é a multiplicação de $\mathbf{U}_{\text{reduzido}}^T$ e a base original ($\mathbf{X}$).
\end{enumerate}

Como é provado matematicamente em \cite{Jolliffe}, com essas operações obtém-se o PCA (vide Algoritmo \ref{alg:PCA1}). 

\begin{center}
 \begin{algorithm}[H]
   \SetAlgoLined
   \Entrada{$\mathbf{X}, d'$} 
   \Saida{$z$}
   \Inicio{
   $d = \text{dim}(\mathbf{X})$\\
   $\Sigma = \frac{1}{d} *  \mathbf{X}^T * \mathbf{X}$\\
   $ \mathbf{U} = \text{autovalores}(\Sigma)$\\
   $  \mathbf{U_{reduzido}} =  \mathbf{U}[:,1:d']$\\
   $  \mathbf{z} = \mathbf{U_{reduzido}}^T * \mathbf{X}$\\
   
   }
   \Retorna{$\mathbf{z}$}
   \label{alg:PCA1}
   \caption{\textsc{Algoritmo do PCA }}
 \end{algorithm}
\end{center}

Para ilustrar o funcionamento do PCA, vamos considerar os dados referentes ao valor do preço do Bitcoin e do Google Trends, apresentados na seção \ref{subsec:base_de_dados}. Essas bases de dados serão concatenadas para formar um gráfico com 3 dimensões. Logo, os dados consistem em uma lista de valores, onde a primeira coluna corresponde a data, a segunda ao valor do preço do Bitcoin e a terceira ao valor do Google Trends. Um gráfico em duas dimensões exibindo as duas listas de valores é apresentada na Figura \ref{gr:2d}. 

Com os dados anteriores, foi executado o algoritmo do PCA com $c=2$, reduzindo a dimensão dos dados para 1. A Figura \ref{gr:1d} apresenta o gráfico resultante do PCA.

Na próxima seção serão apresentados os algoritmos e a metodologia utilizada no trabalho. 

\newpage
 %%%%%%%%%%%%%%%%%%%%%%%%%%%%%%%%%%%%%%%
\begin{center}
	\begin{figure}[H]
    \caption[Comparação dos gráficos do valor do Bitcoin e do \textit{Google Trends}]{Comparação dos gráficos do valor do Bitcoin e do \textit{Google Trends}, no mesmo período utilizado no trabalho \cite{wilson_yelowitz_2014}}\label{fg:comparacao_btc_n}

	\begin{subfigure}{}
		
		\begin{tikzpicture}

		\begin{axis}[
		title={Valor do preço do Bitcoin},
		    date coordinates in=x,
		    xticklabel={\day-\month-\year},
		    x tick label style={rotate=45,anchor=north east},
		    date ZERO=2013-08-19,
		    xmin=2013-08-19, xmax=2016-07-19,
		    ymin=0,ymax=1102,
		    height=8cm,
			width=14cm,
			grid=major,
		    xlabel={Data (dia-mês-ano)},
		    ylabel={Valor em dólares},
		]
		\addlegendentry{Preço BTC}
		\addplot [blue, mark=*] table [x=date, y=value, col sep=comma] {price.csv};
		\end{axis}
		
		\end{tikzpicture}
        \label{gr:btc}
	\end{subfigure}
	%sub fig 2
	\begin{subfigure}{}
		\begin{tikzpicture}

		\begin{axis}[
			title={Valor \textit{Google Trends} do Bitcoin}, 
		    date coordinates in=x,
		    xticklabel={\day-\month-\year},
		    x tick label style={rotate=45,anchor=north east},
		    xmin=2013-08-19, xmax=2016-07-19,
		    ymin=0,ymax=249,
		    height=8cm,
			width=14cm,
			grid=major,
		    xlabel={Data (dia-mês-ano)},
		    ylabel={Valor \%},  
		]
		\addlegendentry{Trends BTC}
		\addplot [red, mark=*] table [x=date, y=bitcoin, col sep=comma] {gt.csv};
		\end{axis}
		\end{tikzpicture}
    	\label{gr:bitcoin_gtrends_mesmoperiodo}
	\end{subfigure}
	\begin{center}
	    Fonte: O autor (2018)
	\end{center}
	\end{figure}
\end{center}
%%%%%%%%%%%%%%%%%%%%%%%%%%%%%%%%%%%%%%%
\newpage
 %%%%%%%%%%%%%%%%%%%%%%%%%%%%%%%%%%%%%%%
\begin{center}
	\begin{figure}[H]
    \caption[Comparação dos gráficos do valor do Bitcoin e do \textit{Google Trends} normalizados]{Comparação dos gráficos do valor do Bitcoin e do \textit{Google Trends} normalizados, no mesmo período utilizado no trabalho
    \cite{wilson_yelowitz_2014}}\label{fg:comparacao_btc_normalizado}

	\begin{subfigure}{}
		
		\begin{tikzpicture}

		\begin{axis}[
		title={Valor do preço do Bitcoin Normalizados},
		    date coordinates in=x,
		    xticklabel={\day-\month-\year},
		    x tick label style={rotate=45,anchor=north east},
		    date ZERO=2013-08-19,
		    xmin=2013-08-19, xmax=2016-07-19,
		    ymin=-2,ymax=3.7,
		    height=8cm,
			width=14cm,
			grid=major,
		    xlabel={Data (dia-mês-ano)},
		    ylabel={Valor em dólares},
		]
		\addlegendentry{Preço BTC}
		\addplot [blue, mark=*] table [x=date, y=value, col sep=comma] {price_n.csv};
		\end{axis}
		\end{tikzpicture}
        \label{gr:btc_normalizado}
	\end{subfigure}
	%sub fig 2
	\begin{subfigure}{}
		\begin{tikzpicture}

		\begin{axis}[
			title={Valor do \textit{Google Trends}  do termor Bitcoin Normalizado}, 
		    date coordinates in=x,
		    xticklabel={\day-\month-\year},
		    x tick label style={rotate=45,anchor=north east},
		    xmin=2013-08-19, xmax=2016-07-19,
		    ymin=-2,ymax=8.4,
		    height=8cm,
			width=14cm,
			grid=major,
		    xlabel={Data (dia-mês-ano)},
		    ylabel={Valor \%},  
		]
		\addlegendentry{Trends BTC}
		\addplot [red, mark=*] table [x=date, y=bitcoin, col sep=comma] {gt_n.csv};
		\end{axis}
		\end{tikzpicture}
    	\label{gr:btc_gtrends_n}
	\end{subfigure}
	\begin{center}
	    Fonte: O autor (2018)
	\end{center}
	\end{figure}
\end{center}
%%%%%%%%%%%%%%%%%%%%%%%%%%%%%%%%%%%%%%%
\newpage
 %%%%%%%%%%%%%%%%%%%%%%%%%%%%%%%%%%%%%%%
\begin{center}
	\begin{figure}[H]
    \caption[Comparação dos gráficos do valor do Bitcoin e do \textit{Google Trends} normalizados concatenados, com o resultado dos mesmos dados após o PCA com $n=2$]{Comparação dos gráficos do valor do Bitcoin e do \textit{Google Trends} normalizados concatenados, com o resultado dos mesmos dados após o PCA com $n=2$, no mesmo período utilizado no trabalho \cite{wilson_yelowitz_2014}}\label{fg:comparacao_btc}

	\begin{subfigure}{}
		
		\begin{tikzpicture}

		\begin{axis}[
		title={Valor do Bitcoin e do \textit{Google Trends} do termo Bitcoin Normalizados, sobrepostos},
		    date coordinates in=x,
		    xticklabel={\day-\month-\year},
		    x tick label style={rotate=45,anchor=north east},
		    date ZERO=2013-08-19,
		    xmin=2013-08-19, xmax=2016-07-19,
		    ymin=-2,ymax=8,
		    height=8cm,
			width=14cm,
			grid=major,
		    xlabel={Data (dia-mês-ano)},
		    ylabel={Valor em dólares},
		]
		\addlegendentry{Trends BTC}
		\addplot [blue, mark=*] table [x=date, y=gtrends, col sep=comma] {2d.csv};
		\addlegendentry{Preço BTC}
		\addplot [red, mark=*] table [x=date, y=bitcoin, col sep=comma] {2d.csv};
		\end{axis}
		\end{tikzpicture}
        \label{gr:2d}
	\end{subfigure}
	%sub fig 2
	\begin{subfigure}{}
		\begin{tikzpicture}

		\begin{axis}[
			title={Valor do Bitcoin e do \textit{Google Trends} fo Bitcoin Normalizados após aplicação do PCA}, 
		    date coordinates in=x,
		    xticklabel={\day-\month-\year},
		    x tick label style={rotate=45,anchor=north east},
		    xmin=2013-08-19, xmax=2016-07-19,
		    ymin=-350,ymax=700,
		    height=8cm,
			width=14cm,
			grid=major,
		    xlabel={Data (dia-mês-ano)},
		    ylabel={Valor},  
		]
		\addlegendentry{PCA com BTC e Trends}
		\addplot [purple, mark=*] table [x=date, y=mix, col sep=comma] {1d.csv};
		\end{axis}
		\end{tikzpicture}
    	\label{gr:1d}
	\end{subfigure}
	\begin{center}
	    Fonte: O autor (2018)
	\end{center}
	\end{figure}
\end{center}
%%%%%%%%%%%%%%%%%%%%%%%%%%%%%%%%%%%%%%%
\section{Algoritmos}
\label{sec:algoritmos}
O foco principal dos algoritmos utilizados neste trabalho, que serão apresentados a seguir, é obter o máximo de informação dos dados, a fim de construir um modelo, uma arquitetura de Rede Neural, capaz de prever o comportamento do valor do preço do Bitcoin. Os algoritmos utilizados são: \emph{Backpropagation} (para adaptação dos parâmetros) e o $K$-Fold (para seleção de modelo), os quais serão apresentados a seguir.
 
\subsection{Backpropagation}
\label{subsec:backpropagation}

 As Redes Neurais estão entre os algoritmos de aprendizado mais poderosos da atualidade. Um dos grandes desafios das Redes Neurais é a escolha dos parâmetros da rede que maximizam o seu desempenho. A escolha dos parâmetros está relacionada diretamente com a \textbf{função custo}.
 
 Para descrever a função custo, vamos considerar o problema de classificação binária, que é o foco deste trabalho. Do ponto de vista da Rede Neural, ela deve possuir apenas um neurônio na última camada para produzir um valor que será encaixado entre 0 e 1, que é alcançado graças a uma função de ativação (nesse trabalho a função sigmoide foi a adotada).
 
 A função custo utilizada neste trabalho foi a MSE. O Backprogation será usado então para minimizar o resultado da função custo, escolhendo os parâmetros da rede que satisfazem essa condição \cite{lehmann2006theory, werbos1974beyond, Rojas1996, werbosBackpropagationthroughtime}. O Backpropagation tem esse nome porque a atualização dos parâmetros se dá num passo final, o qual calcula os erros do passo anterior (ou passo \textit{forward}) de modo retroativo.
 
 Assim, a primeira parte para calcular o Backpropagation consiste na execução o chamado \textit{forward propagation}, etapa que consiste no cômputo da saída $\mathbf{h_\theta}$. Com a saída $\mathbf{h_\theta}$ calculada, devemos calcular o $\mathbf{\delta_k}$, que é o vetor com todos os erros da camada $k$. O cálculo de $\mathbf{\delta_k}$ pode ser visto na Eq. \eqref{eq:deltak}, que pode ser entendido simplesmente como a subtração da saída esperada,  $\mathbf{y}$, pelo resultado obtido, $\mathbf{a_k}$ (é importante notar que $\mathbf{a_k} = \mathbf{h_\theta}$). 
 
\begin{equation}
\label{eq:deltak}
\mathbf{\delta_k} = \mathbf{a_k} - \mathbf{y}
\end{equation}

Com $\mathbf{\delta_k}$ calculado, regredimos uma camada e calculamos $\mathbf{\delta_{k-1}}$, como mostra a Eq. \eqref{eq:deltak-1}, onde $\mathbf{g'(z^k)}$ pode ser calculado usando a Eq. \eqref{eq:glinha}.

\begin{equation}
\label{eq:deltak-1}
\mathbf{\delta_{k-1}} = (\Theta^{k-1})^T \delta^{k}  \mathbf{g'(z^k)}
\end{equation}
  
\begin{equation}
\label{eq:glinha}
\mathbf{g'(z^k)} = \mathbf{a^{k-1}(1-a^{k-1})}
\end{equation}

Esse procedimento é repetido para todas as $k$ camadas, menos a primeira, que é a camada de entrada, para a qual não faz sentido calcular o erro. A Eq. \ref{eq:deltakgenerico} exemplifica o cálculo dos $\mathbf{\delta_{i}}$, os quais serão usados para atualizar os pesos da Rede Neural.  

\begin{equation}
\label{eq:deltakgenerico}
\mathbf{\delta_{i}} = (\Theta^{i})^T \delta^{i+1}  \mathbf{g'(z^i+1)}.
\end{equation}
 
 Essa é a ideia central do Algoritmo Backpropagation, o qual começa por calcular o erro do final (o mais fácil de ser calculado), para então voltar à primeira camada da rede, atualizando os pesos. Com isso, os pesos são atualizados de sorte a melhor se aproximar do resultado desejado. O Algoritmo \ref{alg:BP} descreve o  Backpropagation de modo completo.
 
 \begin{algorithm}[H]
   \SetAlgoLined
   \Entrada{$\mathbf{X, Y}$} 
   \Saida{$\Theta$}
   \Inicio{
    $   \Delta^{(l)}_{ij} \coloneqq \mathbf{0} $\\
       \Para{i = 1 ; m}{
    $         a^{(1)} \coloneqq x^{1}$\\
            Execute forward propagation $a^{(l)}$ para $l=2,3,...,L $\\
    $        \delta^{(L)} \coloneqq a^{(L)}-y^{(i)} $\\
            Calcule todos os $\delta^{(L-1)}$,$\delta^{(L-1)}$,...,$\delta^{(2)} $\\
    $         \Delta^{(l)}_{ij} \coloneqq \Delta^{(l)}_{ij} + a^{(l)}_{j}\Delta^{(l+1)}_{i} $\\
             \Se{j=0}{
    $         \mathbf{D^{(l)}_{ij}} \coloneqq \frac{1}{m}\Delta^{(l)}_{ij} $\\
             }\Se{j$\neq$0}{
    $          \mathbf{D^{(l)}_{ij}} \coloneqq \frac{1}{m} \Delta^{(l)}_{ij}+\lambda\Theta^{(l)}_{ij} $\\
             }
        }
    $    \Theta = \Theta - \mathbf{D}*a $\\
  }
   \Retorna{$\Theta$}
   \label{alg:BP}
   \caption{\textsc{Algoritmo do Backpropagation }}
 \end{algorithm}

Na próxima seção será abordado um dos métodos utilizados para a seleção de modelos, o $K$-Fold. A adoção deste método é imperiosa, para que a arquitetura da rede escolhida seja robusta e menos propensa ao sobreajuste.

\subsection{\textit{K}-Fold}
 
 Um dos problemas encontrados em soluções de Redes Neurais é a estimativa de quantos Neurônios a Rede terá, visto que não existe na literatura um teorema que diga qual o modelo ideal para cada tipo de problema. Ou seja, trata-se essencialmente de uma escolha empírica. Para ajudar na escolha do modelo de rede, foi usado o algoritmo $K$-fold, que é um algoritmo de validação cruzada que tem como objetivo ajudar a estimar parâmetros do modelo final usando os dados de treino disponíveis \cite{bengio2004no, kohavi1995study, rodriguez2010sensitivity, moreno2012study}.
 
  Para executar o $K$-fold, é preciso primeiro selecionar a quantidade de dados que serão reservados para teste e a quantidade de dados que serão reservados para treinamento do modelo, onde usualmente a escolha gira entre algo em torno de 20\% dos dados disponíveis para teste e 80\% dos dados disponíveis para treinamento. Cumpre notar que este conjunto de teste não se confunde com o conjunto de teste final, o qual sempre será separado do conjunto de treinamento, sendo reservado para a avaliação final de desempenho do modelo neural.
  
  Com os dados de treinamento selecionados, é preciso escolher um número $K \in \mathbb{N}$. Nesse projeto, o número 10 foi escolhido pois, como mostrado no estudo realizado pelo trabalho \cite{kohavi1995study}, $K$=10 é a melhor escola para seleção de modelo. Como $K=10$, os dados de treino devem ser divididos em 10 partes iguais, de modo aleatório. Cada uma dessas partes é chamada de um \textit{fold}. Assim, uma parte entre os $K$-folds é selecionada e reservada. A rede é treinada com os $K-1$ folds restantes e validada com a parte escolhida no inicio. O procedimento é repetido para todos os $K$-folds, onde sempre é armazenado o valor da acurácia e do desvio padrão de cada \textit{fold} usado como validação. A ideia central é que o $K$-fold deve ser executado variando o número de neurônios na rede até um valor definido, onde o modelo de rede escolhido será o que obtiver o menor desvio padrão ou a maior acurácia, ao longo de todos os $K$ \textit{folds}.
  
  \begin{figure}
    \centering
    \caption[Exemplo do $K$-fold]{Exemplo do $K$-fold com $K$=10}
    \label{fig:k-fold}

    \begin{tikzpicture}
        \matrix (M) [matrix of nodes,
            nodes={minimum height = 10mm, minimum width = 1.2cm, outer sep=0, anchor=center, draw},
            column 1/.style={nodes={draw=none}, minimum width = 1.2cm},
            row sep=1mm, column sep=-\pgflinewidth, nodes in empty cells,
            e/.style={fill=black!25}
          ]
          {
            Experimento 1  & |[e]| & & & & & & & & & \\
            Experimento 2  & & |[e]| & & & & & & & & \\
            Experimento 3  & & & |[e]| & & & & & & & \\
            Experimento 4  & & & & |[e]| & & & & & & \\
            Experimento 5  & & & & & |[e]| & & & & & \\
            Experimento 6  & & & & & & |[e]| & & & & \\
            Experimento 7  & & & & & & & |[e]| & & & \\
            Experimento 8  & & & & & & & & |[e]| & & \\
            Experimento 9  & & & & & & & & & |[e]| & \\
            Experimento 10 & & & & & & & & & & |[e]| \\
          };
          \draw (M-1-2.north west) ++(0,2mm) coordinate (LT) edge[|<->|, >= latex] node[above]{Número total de dados de treinamento} (LT-|M-1-11.north east);
    \end{tikzpicture}
    \begin{center}
        Fonte: O autor (2018)
    \end{center}
\end{figure}

  A Figura \ref{fig:k-fold} apresenta o modelo do $K$-fold com $K$=10. Cada um dos $K$-folds é representado por um retângulo, onde os retângulos da cor cinza representam o fold de treinamento de cada experimento. Assim, em cada experimento a rede treinará os folds representados pelos retângulos brancos e validará o treinamento com o fold representado pelo retângulo crinza. Note que neste momento, os dados já foram embaralhados, ou seja, cada linha de dados contída dentro de cada fold não está em ordem temporal.
  A próxima seção especificará a metodologia utilizada neste trabalho. Cumpre ressaltar que empregamos o $K$-fold para especificar a quantidade de neurônios no modelo, embora ele também possa ser empregado para outros parâmetros, tais como número de camadas escondidas ou função de ativação a escolher.

\subsection{Metodologia}
\label{sec:metodologia}
Nesta seção de metodologia será explicado o passo a passo de como foi escolhido o modelo da rede a usar: os parâmetros da rede, pesos da rede, pre-processamento dos dados, até o teste final que resultou na acurácia obtida neste trabalho. Para isso, foram utilizadas duas metodologias, as quais serão apresentadas a seguir: 
\begin{enumerate}
\item utilização dos dados de treinamento para selecionar o modelo; 
\item avaliação do resultado para a acurácia obtida nesse trabalho.
\end{enumerate}

Como base para as próximas duas seções utilizaremos 80\% da base de dados como treinamento e 20\% da base de dados como teste, de modo a seguir a mesma proporção do trabalho \cite{mcnally2016predicting}, que será usado como comparação principal neste projeto.

\subsubsection{Tipos de base de dados utilizados}
\label{sec:metobase}
Na análise deste trabalho foram utilizados três tipos de base de dados, baseados no valor do preço do Bitcoin e no valor referentes aos dados do Google Trends, obtidos no período de 19 de Agosto de 2013 até 19 de Julho de 2016, mesmo período utilizado no trabalho \cite{mcnally2016predicting}.

Os três tipos de base de dados utilizados neste trabalho são:
\begin{enumerate}

    \item Dados do valor do preço de Bitcon: foram separadas 14 bases de dados, onde cada base de dados $X_{Bitcoin}^{(n)}$ contém n valores com o preço do Bitcoin por linha, com frequência de um dia, e no final de cada linha o valor 1 se o preço no dia seguinte aumentou e 0 se o preço diminuiu, de modo que cada linha tivesse $n+1$ valores, $n$ valores do preço e 1 valor da previsão;
    
    \item Dados do valor do Google Trends do Bitcoin: foram separadas 14 bases de dados, onde cada base de dados $X_{\text{GTrends}}^{(n)}$ contém n valores do Google Trends do Bitcoin por linha, com frequência de um dia, e no final de cada linha o valor 1 se o preço no dia seguinte aumentou e 0 se o preço diminuiu, de modo que cada linha tivesse $n+1$ valores, $n$ valores do Google Trends e 1 valor da previsão;
    
    \item Dados do valor do preço do Bitcoin concatenado com o valor do Google Trends do Bitcoin: foram separadas 7 bases de dados, onde cada base de dados $X_{\text{Mix}}^{(n)}$ contém $2n+1$ valores, $n$ valores do preço do Bitcoin concatenados com $n$ valores do Google Trends do Bitcoin por linha, com frequência de um dia, e no final de cada linha o valor 1 se o preço no dia seguinte aumentou e 0 se o preço diminuiu. Na base de dado $X_{\text{Mix}}^{(n)}$ foram utilizadas apenas 7 bases de dados para não ultrapassar a quantidade de dados nos dois outros tipos de dados.
    
\end{enumerate}

\subsubsection{Seleção de modelos}
\label{sec:meto0}

 A seleção de modelos é a parte onde será definido o modelo da rede que será usado, utilizando apenas os dados de treinamento. Foi definido um modelo de rede padrão para esse trabalho, pela quantidade de dados disponíveis para treinamento de rede, de três camadas. Em geral, redes com mais camadas só fazem sentido se existirem muitos dados para o treinamento \cite{lecun2015deep,chen2014big, najafabadi2015deep}. Como os dados para esse trabalho se limitaram ao intervalo usado em \cite{mcnally2016predicting}, só existem 1065 linhas de dados do preço do Bitcoin, esse número é considerado pequeno para uma rede com muitas camadas. Na mesma linha de pensamento, foi definida a quantidade máxima de neurônios na segunda camada em 14 neurônios, por não existirem dados suficientes que justifiquem mais neurônios.
 
 O modelo base de rede portanto foi definido assim:
 
 \begin{itemize}
     \item a primeira camada com a mesma dimensionalidade dos dados de entrada;
     \item segunda camada com até 14 Neurônios. A função de ativação utilizada nessa camada será a tangente hiperbólica;
     \item terceira e última camada com um neurônio. A função de ativação utilizada nessa camada será a sigmoide, a qual retorna valores entre 0 e 1.
 \end{itemize}
 

Essa metodologia foi executada para todos os três tipos de base de dados. A seguir a metodologia utilizada é detalhada:
 
 \begin{algorithm}[H]
   \SetAlgoLined
   \Entrada{$\mathbf{X, Y}$} 
    \Saida{Desvio Padrão, Acurácia}
    
    \Inicio{
        \Se{tipo$=X_{Mix}$}{
            $ max_n \coloneqq 7$\\
        }
        \Se{tipo$\neq X_{Mix}$}{
             $ max_n \coloneqq 14$\\
        }
             
       \Para{i = 1 ; $max_n$}{
            \Para{j = 1 ; i}{
                $X_z \coloneqq PCA(X, j)$\\
                $folds = Kfold(n_splits=10)$\\
                \Para{número de Neurônios $= 1$ ; 14}{
                    \Para{treinamento, teste em folds}{
                    
                        crie modelo da Rede Neural 3 camadas, $i$ Neurônio(s) na primeira camada, $n_{\text{neuronios}}$ na segunda camada e $1$ Neurônio na última camada\\
                        treine a rede\\
                        avalie o modelo\\
                        imprima o desvio padrão e a acurácia\\
                        
                    }
                }
            }
        }
  }
  
   \Retorna{$\Theta$}
   \label{alg:selecao_modelo}
   \caption{\textsc{Algoritmo de Seleção de modelo }}
 \end{algorithm}
 
Esse algoritmo foi o algoritmo utilizado para selecionar os melhores modelos. Com Esses dados disponíveis, o algoritmo de teste de modelo usará os melhores modelos para estimar o resultado preciso, evitando que a inicialização da rede influencie muito o resultado.

\subsection{\textit{Ensemble}}

Antes de continuar com a metodologia de teste do modelo, cabe introduzir o conceito de \textit{ensemble}, técnica utilizada nesta metodologia.

O \textit{ensemble} é uma técnica desenvolvida com o objetivo de reduzir a variância de sistemas de decisão, desse modo aumentando a acurácia do sistema \cite{opitz1999popular}. Para isso, o \textit{ensemble} utiliza vários sistemas para resolver o mesmo problema \cite{opitz1999popular, dietterich2000ensemble, tan2003ensemble, kotsiantis2007supervised, dietterich2000experimental, zhang2012ensemble}. A resposta final a ser utilizada é a medina, ou a média, de todos os sistemas, o que reduz a variância, visto que o resultado final usará a reposta que mais aparece cada um dos sistemas do \textit{ensemble}, no caso da mediana em uma classificação binária, o que resulta em um resultado mais confiável e com viés de inicialização reduzido, que pode ocorrer nos sistemas isoladamente.

Apenas para exemplificar o funcionamento do \textit{ensemble}, foi criada a Fig. \\ref{fig:ensemble}. As Figs. \ref{fig:ensemble}a, \ref{fig:ensemble}b e \ref{fig:ensemble}c apresentam 3 modelos que fazem classificação de dados. Cada um desses três modelos classifica os dados e, ao final, suas classificações são somadas (Fig. \ref{fig:ensemble}d). o algoritmo, então, seleciona a média médias (neste trabalho foi utilizado a mediana) da curvas e entrega uma classificação média dos três modelos, que pode ser visto na Fig. \ref{fig:ensemble}e. O resultado final, como mostrado em \cite{opitz1999popular}, reduz variância, consequentemente aumentando a acurácia do modelo final.



\begin{figure}[ht!]
    \centering
    \caption{Exemplo de ensemble}
    \includegraphics[scale=1.45]{Figuras/Cap4/ensemble.png}
    \begin{center}
    Fonte:
        O autor (2018), adaptado de \cite{zhang2012ensemble}.
    \end{center}
         
        \label{fig:ensemble}
\end{figure}

%FEITO
%\dbh{Sua abordagem de métodos ensemble está muito primitiva! Você precisa ler aqui uns 30 artigos, no mínimo, para melhorar este texto!}

\subsubsection{Teste do modelo}
\label{sec:meto1}

O algoritmo de teste de modelo foi pensado para testar os melhores modelos do algoritmo de seleção de modelo. Nesta fase de teste do modelo, foram utilizados os dados de teste, com objetivo de simular o comportamento real do modelo escolhido. cabe ressaltar que esse é o único momento em todo o trabalho em que os dados de teste são utilizado, a fim de simular o comportamento real do modelo.

O algoritmo proposto para o teste do modelo, neste trabalho, realiza dois \textit{ensemble}, executando várias redes (idealmente um número ímpar de redes), com modelos iguais e inicialização dos pesos diferentes.  Ao executar a previsão, as respostas de cada rede são ordenadas e é escolhida a mediana das respostas como resposta final. 
A técnica do \textit{ensemble} foi aplicada em dois contextos no teste do modelo:

\begin{enumerate}
    \item\textit{Ensemble} de modelo: tendo em vista os resultados da seleção de modelo, foram escolhidos os 7 modelos que apresentaram o menor desvio padrão para o \textit{Ensemble};
    \item \textit{Ensemble} de resultado: para cada modelo do \textit{ensemble} de modelo, são criadas 7 Redes Neurais com mesma arquitetura e inicialização diferente. Todas as 7 Redes Neurais são treinadas e testadas, onde o resultado do modelo passa ser a mediana dos resultados de todas as Redes. A mediana é uma medida mais robusta a aberrações do que a média, daí a sua adoção.
\end{enumerate}

Após executar os dois Ensembles, são obtidos os resultados que serão considerados para esse trabalho. No próximo capítulo serão apresentados os resultados do trabalho, obtidos com as metodologias aqui descritas.
\section{Resultados}

Como já foi adiantado no capítulo \ref{sec:algoritmos}, foram realizados 3 grandes avaliações de modelos para os 3 tipos de dados utilizados, que são: dados do valor do preço do Bitcoin, dados do valor do \textit{Google Trends} do termo Bitcoin e os dados do valor do preço do bitcoin e dados do valor do \textit{Google Trends} do termo Bitcoin combinados (concatenados em linha). Esta seção relatará cada um desses experimentos e mostra os resultados obtidos em cada base de dados


\subsection{Comparação dos resultados das três entradas testadas}

Como foi visto na seção \ref{sec:metodologia}, todas as 3 avaliações realizadas neste trabalho, utilizando as metodologias \ref{sec:meto0} e \ref{sec:meto1}, com as 3 bases de dados apresentados na seção \ref{sec:metobase}, serão apresentadas nas próximas seções.

\subsubsection{Resultado com a base de dados do Tipo 1}

Ao executar a metodologia \ref{sec:meto0} com os dados do Tipo 1, que contém apenas o valor do preço do Bitcoin, foram selecionados, ao final, 7 Modelos de Rede Neural a obterem os menores desvio padrão. Os 7 modelos vencedores da metodologia \ref{sec:meto0} são exibidos na Tabela \ref{tab:moto0-1}.

Todos os 7 modelos vencedores utilizaram a base de dados com com 5 amostras e PCA igual a 5, lembrando que o a frequência das amostras é diária. A arquitetura vencedora, que é a arquitetura que ficou na mediana dos valores, apresentou RMS igual a \textbf{24,6557\%}, Desvio padrão de \textbf{5,0190\%} e acurácia média de \textbf{56,6545\%}. Essa arquitetura foi a utilizada na metodologia \ref{sec:meto1}.

\begin{table}[!ht]
\centering

\caption{Tabela de avaliação de modelo dos dados do Tipo 1}
\label{tab:moto0-1}
\begin{tabular}{|M{0.2cm}|M{4cm}|M{1cm}|M{2cm}|M{1.6cm}|M{2cm}|M{1.5cm}|}
\hline
 &\#Neurônios na camada escondida&\#PCA &\#Amostras por linha&RMS (\%)&Desvio padrão (\%)&Acurácia (\%)\\\hline
 1&  1&  5&  5&  24,6357&  3,9080&  57,8748\\ 
 2&  9&  5&  5&  24,6498&  5,0229&  56,7289\\
 3&  7&  5&  5&  24,6515&  4,6648&  56,9145\\
 \textbf{4}&  \textbf{8}&  \textbf{5}&  \textbf{5}&  \textbf{24,6557}&  \textbf{5,0190}&  \textbf{56,6545}\\
 5&  4&  5&  5&  24,6760&  4,7606&  56,6483\\
 6&  5&  5&  5&  24,6773&  4,6635&  56,5674\\
 7&  2&  5&  5&  24,6818&  4,4030&  56,7106\\\hline
\end{tabular}
\begin{center}
	    Fonte: O autor (2018)
	\end{center}
\end{table}

Para executar a metodologia \ref{sec:meto1}, foram realizadas 7 testes da arquitetura vencedora, que podem ser vistos na Tabela \ref{tab:moto1-1}. O resultado final da base de dados do Tipo 1 foi de \textbf{50,7042\%}.

\begin{table}[!ht]
\centering

\caption{Tabela de teste dos dados do Tipo 1}
\label{tab:moto1-1}
\begin{tabular}{|M{0.4cm}|M{2cm}|}
\hline

 &Acurácia (\%)\\\hline
 1&48.826\\
 2&50,2347\\
 3&50,7042\\
 \textbf{4}&\textbf{50,7042}\\
 5&50,7042\\
 6&51.1737\\
 7&52.5822\\\hline
\end{tabular}
\begin{center}
	    Fonte: O autor (2018)
	\end{center}
\end{table}

\subsubsection{Resultado com a base de dados do Tipo 2}

Ao executar a metodologia \ref{sec:meto0} com os dados do Tipo 2, que contém apenas o valor do \textit{Google Trends} do Bitcoin, foram selecionados, ao final, 7 Modelos de Rede Neural a obterem os menores desvio padrão. Os 7 modelos vencedores da metodologia \ref{sec:meto0} são exibidos na Tabela \ref{tab:moto0-2}.

Todos os 7 modelos vencedores utilizaram a base de dados com com x amostras e PCA igual a x, com frequência de amostra diária. A arquitetura vencedora, que é a arquitetura que ficou na mediana dos valores, apresentou RMS igual a \textbf{24,7828\%}, Desvio padrão de \textbf{4,5962\%} e acurácia média de \textbf{52,5286\%}. Essa arquitetura foi a utilizada na metodologia \ref{sec:meto1}.

\begin{table}[!ht]
\centering
\caption{Tabela de avaliação de modelo dos dados do Tipo 2}
\label{tab:moto0-2}
\begin{tabular}{|M{0.2cm}|M{4cm}|M{1cm}|M{2cm}|M{1.6cm}|M{2cm}|M{1.5cm}|}
\hline
 &\#Neurônios na camada escondida&\#PCA &\#Amostras por linha&RMS (\%)&Desvio padrão (\%)&Acurácia (\%)\\\hline
 1&  1&  2&  4&  24,7250&  4,7261&  53,0957\\ 
 2&  2&  2&  4&  24.7503&  4,9895&  52,5204\\
 3&  3&  2&  4&  24,7772&  4,6581&  52,2155\\
 \textbf{4}&  \textbf{4}&  \textbf{2}&  \textbf{4}&  \textbf{24,7828}&  \textbf{4,5962}&  \textbf{52,5286}\\
 5&  10&  2&  4&  24,7891&  4,2942&  52,4477\\
 6&  6&  2&  4&  24,7912&  4,0864&  52,8628\\
 7&  12&  2&  4&  24,7975&  4,1099&  52,3636\\\hline
\end{tabular}
\begin{center}
	    Fonte: O autor (2018)
	\end{center}
\end{table}

Para executar a metodologia \ref{sec:meto1}, foram realizadas 7 testes da arquitetura vencedora, que podem ser vistos na Tabela \ref{tab:moto1-2}. O resultado final da base de dados do Tipo 2 foi de \textbf{49,2958\%}.

\begin{table}[!ht]
\centering

\caption{Tabela de teste dos dados do Tipo 2}
\label{tab:moto1-2}
\begin{tabular}{|M{0.4cm}|M{2cm}|}
\hline
 &Acurácia (\%)\\\hline
 1&48,8263\\
 2&48,8263\\
 3&48,8263\\
\textbf{4}&  \textbf{49,2958}\\
 5&49,2958\\
 6&49,7652\\
 7&49,7652\\\hline
\end{tabular}
\begin{center}
	    Fonte: O autor (2018)
	\end{center}
\end{table}

\subsubsection{Resultado com a base de dados do Tipo 3}

Ao executar a metodologia \ref{sec:meto0} com os dados do Tipo 3, que contém o valor do preço do Bitcoin e do valor do \texttt{Google Trends} do termo Bitcoin concatenados, foram selecionados, ao final, 7 Modelos de Rede Neural a obterem os menores desvio padrão. Os 7 modelos vencedores da metodologia \ref{sec:meto0} são exibidos na Tabela \ref{tab:moto0-3}.

Todos os 7 modelos vencedores utilizaram a base de dados com com 4 amostras e PCA igual a 2, com frequência de amostra diária. A arquitetura vencedora, que é a arquitetura que ficou na mediana dos valores, apresentou RMS igual a \textbf{24,7828\%}, Desvio padrão de \textbf{4,5962\%} e acurácia média de \textbf{52,5286\%}. Essa arquitetura foi a utilizada na metodologia \ref{sec:meto1}.

\begin{table}[!ht]
\centering

\caption{Tabela de avaliação de modelo dos dados do Tipo 3}
\label{tab:moto0-3}
\begin{tabular}{|M{0.2cm}|M{4cm}|M{1cm}|M{2cm}|M{1.6cm}|M{2cm}|M{1.5cm}|}
\hline
 &\#Neurônios na camada escondida&\#PCA &\#Amostras por linha&RMS (\%)&Desvio padrão (\%)&Acurácia (\%)\\\hline
 1&  3&  6&  6&  24,5493&  4,7034&  57,7263\\ 
 2&  2&  6&  6&  24,5499&  5,2139&  57,5297\\
 3&  13&  6&  6&  24,6325&  4,9916&  56,9421\\
 \textbf{4}&  \textbf{8}&  \textbf{6}&  \textbf{6}&  \textbf{24,6357}&  \textbf{5,1420}&  \textbf{56,5317}\\
 5& 10&  6&  6&  24,6402&  5,0183&  56,8247\\
 6&  6&  6&  6&  24,6405&  5,2531&  56,7097\\
 7&  9&  6&  6&  24,6409&  5,0773&  56,5769\\\hline
\end{tabular}
\begin{center}
	    Fonte: O autor (2018)
	\end{center}
\end{table}

Para executar a metodologia \ref{sec:meto1}, foram realizadas 7 testes da arquitetura vencedora, que podem ser vistos na Tabela \ref{tab:moto1-3}. O resultado final da base de dados do Tipo 3 foi de \textbf{53,9906\%}.

\begin{table}[!ht]

\centering

\caption{Tabela de teste dos dados do Tipo 3}
\label{tab:moto1-3}
\begin{tabular}{|M{0.4cm}|M{2cm}|}
\hline
 &Acurácia (\%)\\\hline
 4&53,0516\\
 1&53,5211\\
 2&53,0516\\
 3&553,9906\\
 \textbf{4}&\textbf{53,9906}\\
 5&54,4601\\
 6&54,9296\\
 7&55,8685\\\hline
\end{tabular}
\begin{center}
	    Fonte: O autor (2018)
	\end{center}
\end{table}

\subsection{Comparação do resultado com trabalhos anteriores}

O foco principal deste trabalho foi a comparação com trabalhos anteriores. Até a data deste trabalho, o único trabalho que havia apresentado uma metodologia e resultado para este problema, de predição dos valores futuros do Bitcoin, foi o trabalho \cite{mcnally2016predicting}. Por esse motivo este trabalho utilizou o mesmo intervalo de dados de \cite{mcnally2016predicting}, com objetivo de propor que os dados provenientes do \textit{Google Trends} pode ajudar aumentar a acurácia de um modelo de predição do valor do preço do Bitcoin. O resultado da comparação entre este trabalho e o \cite{mcnally2016predicting} pode ser encontrado na Tabela \ref{tab:compamatta2015}.

\begin{table}[!ht]
\centering

\caption[Tabela de comparação do resultado]{Tabela de comparação do resultado deste trabalho com o resultado do trabalho \cite{mcnally2016predicting}}
\label{tab:compamatta2015}
\begin{tabular}{|M{0.8cm}|M{2.4cm}|M{0.8cm}|M{2.6cm}|M{2.5cm}|}
\hline
 Tipo&Trabalho \cite{mcnally2016predicting} Acurácia (\%)&Tipo&Este trabalho Acurácia (\%)&Diferença (\%)\\\hline
 1&52,78&1&50,70&-2,08\\
 1&52,78&2&49,30&-3,48\\
 \textbf{1}&\textbf{52,78}&\textbf{3}& \textbf{53,99}&\textbf{+1,21}\\\hline
 
\end{tabular}
\begin{center}
	    Fonte: O autor (2018)
	\end{center}
\end{table}

\subsection{Comparação com o serviço Amazon Machine Learning}

Para fazer avaliações deste trabalho com outras metodolodias e técnicas de classificação binária, foi utilizado um serviço da Amazon, especificamente da AWS (\textit{Amazon Web Services}), que é o maior provedor de computação em nuvem do mundo, e oferece diversos serviços na area de Inteligência Artificial e \textit{Machine Learning}. Dentre os serviços disponíveis que permitem realizar a classificação binária, está o serviço chamado \textbf{Amazon Machine Learning}  \cite{amazonmachinelearning}, que foi utilizado com o objetivo de comparação de resultado com o modelo proposto neste trabalho.

O Amazon Machina Learning realiza basicamente 3 tipos de tarefas:

\begin{enumerate}
    \item Classificação binária: utilizando o algoritmo de Regressão logística, que usa a \textit{logistic loss function} somado com o gradiente descendente estocástico;
    \item classificação multi-classe: utilizando o alforitmo de Regressão logística multinomial, que usa a \textit{multinomial logistic loss} somado com o gradiente descendente estocástico;
    \item regressão: utilizando o algoritmo de Regressão, que usa a \textit{squared loss function} somado com o o gradiente descendente estocástico.
\end{enumerate}

Como este este trabalho se propõem a estudar um problema de classificação binária, foi utilizado a classificação binária Amazon Machine Learning. 

\subsubsection{Experimentos com o Amazon Machine Learning}

Para os experimentos com o Amazon Machine Learning, foram utilizados as mesmas bases de dados utilizadas na \ref{sec:meto1}, que foram as bases dos modelos vencedores provenientes da metodologia \ref{sec:meto0}. Foram utilizadas as configurações padrões e recomendadas pelo do serviço da Amazon, que utilizam 70\% dos dados para treinamento e 30\% para teste, separando os dados de maneira sequencial.

Para os dados vencedores do Tipo 1, que é a base de dados com apenas o valor do preço do Bitcoin com 5 amostras por linha e frequência de 1 dia. Essa configuração gerou resultado de 47,17\% 

Para os dados vencedores do Tipo 2, que é a base de dados com apenas o valor \textit{Google Trends} do Bitcoin com x amostras por linha e frequência de 1 dia. Essa configuração gerou resultado de 50,17\%.

Para os dados vencedores do Tipo 3, que é a base de dados com a concatenação de amostras o valor do preço do Bitcoin e do valor do \texttt{Google Trends} do termo Bitcoin, com 6, onde 3 são do valor do preço do Bitcoin e 3 \texttt{Google Trends} do termo Bitcoin amostras por linha e frequência de 1 dia, ou seja, essa base de dados representa dados de 3 dias. Essa configuração gerou resultado de 51,64\%.

A tabela \ref{tab:compamazon} apresenta o resultado da comparação 

\begin{table}[!ht]
\centering

\caption{Tabela Comparação do resultado da Amazon Machine Learning (AML) e da metodologia proposta neste trabalho}
\label{tab:compamazon}
\begin{tabular}{|M{0.8cm}|M{2.4cm}|M{2.6cm}|M{2.5cm}|}
\hline
 Tipo&AML Acurácia (\%)&Este trabalho Acurácia (\%)&Diferença (\%)\\\hline
 1&47,17&50,70&+3,53\\
 2&50,17&49,30&-0,87\\
 3&51,64&53,99&+2,35\\\hline
\end{tabular}
\begin{center}
	    Fonte: O autor (2018)
	\end{center}
\end{table}

\subsection{Teoria do mercado eficiente}

%Feito
%\dbh{eu esperava uma discussão muito mais profunda aqui, recheada de referências!}

A Teoria de mercado eficiente, proposta por \cite{malkiel1970efficient} em 1969, é uma Teoria que define uma mercado eficiente como aquele que sempre consegue refletir o preço do ativo de acordo quantidade de informações disponíveis desse ativo. De acordo com essa Teoria, é impossível alguém conseguir vencer o mercado, pois todas as informações disponíveis para precificar o ativo já estão disponíveis para todos os investidores. 

Na Teoria de mercado eficiente, como explica \cite{junior2004mercados}, sempre que uma informação nova fica disponível, automaticamente o mercado se adapta para ajustar o valor do preço do ativo, ou seja, o mercado sempre tende ao preço real de um ativo, ou valor intrínseco, com raras exceções \cite{jensen1978some}.

Dentro da teoria de mercado eficientes, como explicado em \cite{basu1977investment}, diversos estudo empíricos apontam que o preço refletem conjunto particular de informações  disponíveis sobre determinado ativo, sendo assim possível classificar 3 diferentes formas de eficiência:

\begin{itemize}
    \item Forma fraca: onde apenas informações do preço histórico dos ativos é levado em conta;
    \item forma semi-forte: onde a preocupação e se os preços vão se ajustar eficientemente a novas informações que surgirem, essas informações têm que ser, obviamente, públicas;
    \item forma forte: onde a preocupação é se mesmo que alguns investidores ou grupos (e.g., administradores de fundos de investimento) tenham informações privilegiadas relevantes para o preço, se o preço vai se adaptar a essas novas informações.
\end{itemize}

Essas classificações servem para identificar o montante de informação para cada uma das formas e conseguir distinguir quando elas falham. Cabe ressaltar os termos \textbf{preço real} ou \textbf{valor intrínseco} são aproximações, e não valores perfeitamente exatos. 

A teoria de mercado eficiente também é abordada em \cite{timmermann2004efficient} para a previsão de valores, mostrando que \textit{traders} (nome designado a investidores que compram e vendem ações em curto período de tempo, em geral no mesmo dia) tentam explorar o \textit{delay} que leva entre uma nova informação sobre um determinado ativo ficar disponível e o mercado atualizar o preço deste ativo.

Essa Teoria pode explicar a dificuldade encontrada tanto em \cite{mcnally2016predicting}, quanto neste trabalho em conseguir prever os valores do Bitcoin, mesmo com o uso do \textit{Google Trends}, que ajuda a encontrar comportamentos de manada dos investidores. O trabalho \cite{bartos2015does} avalia o ativo Bitcoin como um mercado eficiente, chegando na conclusão que os eventos disponíveis, tanto positivos, quanto negativos, influenciam diretamente o valor do Bitcoin, de modo extremamente rápido. Isso significa que o preço fica mais baixo durante eventos negativos, e permanecendo em alta durante eventos positivos. Também mostra o Bitcoin segue a formar padrão de oferta e demanda, podendo alterar bastante o preço do ativo.
\section{Conclusões}\label{conclusao}


Toda a implementação, junto com a documentação de como usar o códigos estão disponíveis no GitHub em \cite{repositorio}.

%\nocite{NBR6023,NBR6024,NBR6027,NBR6028,NBR10520,NBR14724,IBGE,CEFET_2007,CEFET_2014}



%@manual{CEFET_2014,
%address={Petr�polis (RJ)},
%organization={CELSO FEDERAL DE EDUCA��O TECNOL�GICO CELSO SUCKOW DA FONSECA. Campus Petr�polis. Coordena��o de %Trabalho %de Conclus�o de Curso},
%title={Manual para elabora��o de Trabalho de Conclus�o de Curso (TCC)},
%year={2014}
%}


%%% Bibliografia %%%
%\addcontentsline{toc}{section}{Refer�ncias}
%\bibliographystyle{apateste}

%\bibliographystyle{abnt-alf}
{
\thispagestyle{empty}
\renewcommand{\refname}{REFERÊNCIAS}
\bibliography{bibliografia}
}

\newpage

\begin{appendices}
%\renewcommand{\chaptername}{Ap�ndice}
\section{Título do Apêndice} \label{ap:defesa}

\paragraph{}Elemento que consiste em um texto ou documento elaborado pelo autor, com o intuito de complementar sua, sem prejuízo do trabalho. São identificados por letras maiúsculas consecutivas e pelos respectivos títulos.
\include{apendiceC_TCC}
%\include{apendiceD}
\end{appendices}


\end{document}