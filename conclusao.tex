\section{Conclusões}\label{conclusao}

Esse Trabalho apresentou uma nova abordagem para o problema de predição do valor do Bitcoin, anteriormente explorado em \cite{mcnally2016predicting} apenas com o valor do preço dessa cripto-moeda, trazendo novos dados, o valor do \textit{Google Trends} do termo Bitcoin, com objetivo de aumentar a acurácia encontrada em \cite{mcnally2016predicting} que foi de 52,78\%. Nesse sentido é possível dizer que o trabalho obteve sucesso, visto que obteve acurácia de 53,99\%, aumento de 1,21\% em relação a \cite{mcnally2016predicting}, utilizando os dados do valor do preço do Bitcoin e o valor do \textit{Google Trends} do termo Bitcoin.

Entretanto, o acurácia ainda está muito abaixo de uma valor no qual seria possível obter lucros superiores ao mercado. Essa dificuldade pode sfer explicada parcialmente pela teoria de mercado eficiente, onde \cite{bartos2015does} apresenta um estudo empírico que conclui o Bitcoin se comporta como um mercado eficiente, o que significa que com as informações pública em relação ao Bitcoin, sendo as tendências de busca uma dessas informações públicas, não seria possível obter lucros acima da média. O único jeito então de superar o mercado seria por sorte ou por uso informações privilegiadas, que levariam um certo tempo até serem absorvidas por toda a rede de investidores.

Um aspecto interessante do Bitcoin, e de ripto-moedas em geral, é que por não serem reguladas por entidades governamentais, eles não estão vinculadores exclusivamente a uma bolsa, ou seja, podem ser negociadas 24 horas por dia, 7 dias na semana, sem feriados, ou qualquer outro evento normal que faria a bolsa ficar fora de operação, logo, a correção do valor ocorreria  rapidamente quando uma nova informação relevante fosse divulgada.

Para trabalhos futuros, é proposto uma extensão de como o valor do Bitcoin influência as outras cripto-moedas, como o Litecoin, Ripple, Ethereum e Bitcoin Cash, onde em \cite{baccao2018information} foi apresentado um estudo neste sentido, e utilizar o \textit{delay} proveniente dessa influência para obter lucros superiores ao mercado. Outra possível extensão deste trabalho é seguir na linha de identificações de tendencias, utilizando algum tipo Inteligência Artificial para classificação de noticias e comentários na internet, e utiliza-los para se antecipar a movimentos do mercado, visto que como mostrado em \cite{bartos2015does} existe um \textit{delay} para que novas informações seja atualizei o preço do Bitcoin.

Toda a implementação, junto com a documentação de como usar o códigos, estão disponíveis no GitHub em \cite{repositorio}.