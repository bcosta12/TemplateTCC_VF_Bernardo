\section{Resultados}

Como já foi adiantado no capítulo \ref{sec:algoritmos}, foram realizados 3 grandes avaliações de modelos para os 3 tipos de dados utilizados, que são: dados do valor do preço do Bitcoin, dados do valor do \textit{Google Trends} do termo Bitcoin e os dados do valor do preço do bitcoin e dados do valor do \textit{Google Trends} do termo Bitcoin combinados (concatenados em linha). Esta seção relatará cada um desses experimentos e mostra os resultados obtidos em cada base de dados


\subsection{Comparação dos resultados das três entradas testadas}

Como foi visto na seção \ref{sec:metodologia}, todas as 3 avaliações realizadas neste trabalho, utilizando as metodologias \ref{sec:meto0} e \ref{sec:meto1}, com as 3 bases de dados apresentados na seção \ref{sec:metobase}, serão apresentadas nas próximas seções.

\subsubsection{Resultado com a base de dados do Tipo 1}

Ao executar a metodologia \ref{sec:meto0} com os dados do Tipo 1, que contém apenas o valor do preço do Bitcoin, foram selecionados, ao final, 7 Modelos de Rede Neural a obterem os menores desvio padrão. Os 7 modelos vencedores da metodologia \ref{sec:meto0} são exibidos na Tabela \ref{tab:moto0-1}.

Todos os 7 modelos vencedores utilizaram a base de dados com com 5 amostras e PCA igual a 5, lembrando que o a frequência das amostras é diária. A arquitetura vencedora, que é a arquitetura que ficou na mediana dos valores, apresentou RMS igual a \textbf{24,6557\%}, Desvio padrão de \textbf{5,0190\%} e acurácia média de \textbf{56,6545\%}. Essa arquitetura foi a utilizada na metodologia \ref{sec:meto1}.

\begin{table}[!ht]
\centering

\caption{Tabela de avaliação de modelo dos dados do Tipo 1}
\label{tab:moto0-1}
\begin{tabular}{|M{0.2cm}|M{4cm}|M{1cm}|M{2cm}|M{1.6cm}|M{2cm}|M{1.5cm}|}
\hline
 &\#Neurônios na camada escondida&\#PCA &\#Amostras por linha&RMS (\%)&Desvio padrão (\%)&Acurácia (\%)\\\hline
 1&  1&  5&  5&  24,6357&  3,9080&  57,8748\\ 
 2&  9&  5&  5&  24,6498&  5,0229&  56,7289\\
 3&  7&  5&  5&  24,6515&  4,6648&  56,9145\\
 \textbf{4}&  \textbf{8}&  \textbf{5}&  \textbf{5}&  \textbf{24,6557}&  \textbf{5,0190}&  \textbf{56,6545}\\
 5&  4&  5&  5&  24,6760&  4,7606&  56,6483\\
 6&  5&  5&  5&  24,6773&  4,6635&  56,5674\\
 7&  2&  5&  5&  24,6818&  4,4030&  56,7106\\\hline
\end{tabular}
\begin{center}
	    Fonte: O autor (2018)
	\end{center}
\end{table}

Para executar a metodologia \ref{sec:meto1}, foram realizadas 7 testes da arquitetura vencedora, que podem ser vistos na Tabela \ref{tab:moto1-1}. O resultado final da base de dados do Tipo 1 foi de \textbf{50,7042\%}.

\begin{table}[!ht]
\centering

\caption{Tabela de teste dos dados do Tipo 1}
\label{tab:moto1-1}
\begin{tabular}{|M{0.4cm}|M{2cm}|}
\hline

 &Acurácia (\%)\\\hline
 1&48.826\\
 2&50,2347\\
 3&50,7042\\
 \textbf{4}&\textbf{50,7042}\\
 5&50,7042\\
 6&51.1737\\
 7&52.5822\\\hline
\end{tabular}
\begin{center}
	    Fonte: O autor (2018)
	\end{center}
\end{table}

\subsubsection{Resultado com a base de dados do Tipo 2}

Ao executar a metodologia \ref{sec:meto0} com os dados do Tipo 2, que contém apenas o valor do \textit{Google Trends} do Bitcoin, foram selecionados, ao final, 7 Modelos de Rede Neural a obterem os menores desvio padrão. Os 7 modelos vencedores da metodologia \ref{sec:meto0} são exibidos na Tabela \ref{tab:moto0-2}.

Todos os 7 modelos vencedores utilizaram a base de dados com com x amostras e PCA igual a x, com frequência de amostra diária. A arquitetura vencedora, que é a arquitetura que ficou na mediana dos valores, apresentou RMS igual a \textbf{24,7828\%}, Desvio padrão de \textbf{4,5962\%} e acurácia média de \textbf{52,5286\%}. Essa arquitetura foi a utilizada na metodologia \ref{sec:meto1}.

\begin{table}[!ht]
\centering
\caption{Tabela de avaliação de modelo dos dados do Tipo 2}
\label{tab:moto0-2}
\begin{tabular}{|M{0.2cm}|M{4cm}|M{1cm}|M{2cm}|M{1.6cm}|M{2cm}|M{1.5cm}|}
\hline
 &\#Neurônios na camada escondida&\#PCA &\#Amostras por linha&RMS (\%)&Desvio padrão (\%)&Acurácia (\%)\\\hline
 1&  1&  2&  4&  24,7250&  4,7261&  53,0957\\ 
 2&  2&  2&  4&  24.7503&  4,9895&  52,5204\\
 3&  3&  2&  4&  24,7772&  4,6581&  52,2155\\
 \textbf{4}&  \textbf{4}&  \textbf{2}&  \textbf{4}&  \textbf{24,7828}&  \textbf{4,5962}&  \textbf{52,5286}\\
 5&  10&  2&  4&  24,7891&  4,2942&  52,4477\\
 6&  6&  2&  4&  24,7912&  4,0864&  52,8628\\
 7&  12&  2&  4&  24,7975&  4,1099&  52,3636\\\hline
\end{tabular}
\begin{center}
	    Fonte: O autor (2018)
	\end{center}
\end{table}

Para executar a metodologia \ref{sec:meto1}, foram realizadas 7 testes da arquitetura vencedora, que podem ser vistos na Tabela \ref{tab:moto1-2}. O resultado final da base de dados do Tipo 2 foi de \textbf{49,2958\%}.

\begin{table}[!ht]
\centering

\caption{Tabela de teste dos dados do Tipo 2}
\label{tab:moto1-2}
\begin{tabular}{|M{0.4cm}|M{2cm}|}
\hline
 &Acurácia (\%)\\\hline
 1&48,8263\\
 2&48,8263\\
 3&48,8263\\
\textbf{4}&  \textbf{49,2958}\\
 5&49,2958\\
 6&49,7652\\
 7&49,7652\\\hline
\end{tabular}
\begin{center}
	    Fonte: O autor (2018)
	\end{center}
\end{table}

\subsubsection{Resultado com a base de dados do Tipo 3}

Ao executar a metodologia \ref{sec:meto0} com os dados do Tipo 3, que contém o valor do preço do Bitcoin e do valor do \texttt{Google Trends} do termo Bitcoin concatenados, foram selecionados, ao final, 7 Modelos de Rede Neural a obterem os menores desvio padrão. Os 7 modelos vencedores da metodologia \ref{sec:meto0} são exibidos na Tabela \ref{tab:moto0-3}.

Todos os 7 modelos vencedores utilizaram a base de dados com com 4 amostras e PCA igual a 2, com frequência de amostra diária. A arquitetura vencedora, que é a arquitetura que ficou na mediana dos valores, apresentou RMS igual a \textbf{24,7828\%}, Desvio padrão de \textbf{4,5962\%} e acurácia média de \textbf{52,5286\%}. Essa arquitetura foi a utilizada na metodologia \ref{sec:meto1}.

\begin{table}[!ht]
\centering

\caption{Tabela de avaliação de modelo dos dados do Tipo 3}
\label{tab:moto0-3}
\begin{tabular}{|M{0.2cm}|M{4cm}|M{1cm}|M{2cm}|M{1.6cm}|M{2cm}|M{1.5cm}|}
\hline
 &\#Neurônios na camada escondida&\#PCA &\#Amostras por linha&RMS (\%)&Desvio padrão (\%)&Acurácia (\%)\\\hline
 1&  3&  6&  6&  24,5493&  4,7034&  57,7263\\ 
 2&  2&  6&  6&  24,5499&  5,2139&  57,5297\\
 3&  13&  6&  6&  24,6325&  4,9916&  56,9421\\
 \textbf{4}&  \textbf{8}&  \textbf{6}&  \textbf{6}&  \textbf{24,6357}&  \textbf{5,1420}&  \textbf{56,5317}\\
 5& 10&  6&  6&  24,6402&  5,0183&  56,8247\\
 6&  6&  6&  6&  24,6405&  5,2531&  56,7097\\
 7&  9&  6&  6&  24,6409&  5,0773&  56,5769\\\hline
\end{tabular}
\begin{center}
	    Fonte: O autor (2018)
	\end{center}
\end{table}

Para executar a metodologia \ref{sec:meto1}, foram realizadas 7 testes da arquitetura vencedora, que podem ser vistos na Tabela \ref{tab:moto1-3}. O resultado final da base de dados do Tipo 3 foi de \textbf{53,9906\%}.

\begin{table}[!ht]

\centering

\caption{Tabela de teste dos dados do Tipo 3}
\label{tab:moto1-3}
\begin{tabular}{|M{0.4cm}|M{2cm}|}
\hline
 &Acurácia (\%)\\\hline
 4&53,0516\\
 1&53,5211\\
 2&53,0516\\
 3&553,9906\\
 \textbf{4}&\textbf{53,9906}\\
 5&54,4601\\
 6&54,9296\\
 7&55,8685\\\hline
\end{tabular}
\begin{center}
	    Fonte: O autor (2018)
	\end{center}
\end{table}

\subsection{Comparação do resultado com trabalhos anteriores}

O foco principal deste trabalho foi a comparação com trabalhos anteriores. Até a data deste trabalho, o único trabalho que havia apresentado uma metodologia e resultado para este problema, de predição dos valores futuros do Bitcoin, foi o trabalho \cite{mcnally2016predicting}. Por esse motivo este trabalho utilizou o mesmo intervalo de dados de \cite{mcnally2016predicting}, com objetivo de propor que os dados provenientes do \textit{Google Trends} pode ajudar aumentar a acurácia de um modelo de predição do valor do preço do Bitcoin. O resultado da comparação entre este trabalho e o \cite{mcnally2016predicting} pode ser encontrado na Tabela \ref{tab:compamatta2015}.

\begin{table}[!ht]
\centering

\caption[Tabela de comparação do resultado]{Tabela de comparação do resultado deste trabalho com o resultado do trabalho \cite{mcnally2016predicting}}
\label{tab:compamatta2015}
\begin{tabular}{|M{0.8cm}|M{2.4cm}|M{0.8cm}|M{2.6cm}|M{2.5cm}|}
\hline
 Tipo&Trabalho \cite{mcnally2016predicting} Acurácia (\%)&Tipo&Este trabalho Acurácia (\%)&Diferença (\%)\\\hline
 1&52,78&1&50,70&-2,08\\
 1&52,78&2&49,30&-3,48\\
 \textbf{1}&\textbf{52,78}&\textbf{3}& \textbf{53,99}&\textbf{+1,21}\\\hline
 
\end{tabular}
\begin{center}
	    Fonte: O autor (2018)
	\end{center}
\end{table}

\subsection{Comparação com o serviço Amazon Machine Learning}

Para fazer avaliações deste trabalho com outras metodolodias e técnicas de classificação binária, foi utilizado um serviço da Amazon, especificamente da AWS (\textit{Amazon Web Services}), que é o maior provedor de computação em nuvem do mundo, e oferece diversos serviços na area de Inteligência Artificial e \textit{Machine Learning}. Dentre os serviços disponíveis que permitem realizar a classificação binária, está o serviço chamado \textbf{Amazon Machine Learning}  \cite{amazonmachinelearning}, que foi utilizado com o objetivo de comparação de resultado com o modelo proposto neste trabalho.

O Amazon Machina Learning realiza basicamente 3 tipos de tarefas:

\begin{enumerate}
    \item Classificação binária: utilizando o algoritmo de Regressão logística, que usa a \textit{logistic loss function} somado com o gradiente descendente estocástico;
    \item classificação multi-classe: utilizando o alforitmo de Regressão logística multinomial, que usa a \textit{multinomial logistic loss} somado com o gradiente descendente estocástico;
    \item regressão: utilizando o algoritmo de Regressão, que usa a \textit{squared loss function} somado com o o gradiente descendente estocástico.
\end{enumerate}

Como este este trabalho se propõem a estudar um problema de classificação binária, foi utilizado a classificação binária Amazon Machine Learning. 

\subsubsection{Experimentos com o Amazon Machine Learning}

Para os experimentos com o Amazon Machine Learning, foram utilizados as mesmas bases de dados utilizadas na \ref{sec:meto1}, que foram as bases dos modelos vencedores provenientes da metodologia \ref{sec:meto0}. Foram utilizadas as configurações padrões e recomendadas pelo do serviço da Amazon, que utilizam 70\% dos dados para treinamento e 30\% para teste, separando os dados de maneira sequencial.

Para os dados vencedores do Tipo 1, que é a base de dados com apenas o valor do preço do Bitcoin com 5 amostras por linha e frequência de 1 dia. Essa configuração gerou resultado de 47,17\% 

Para os dados vencedores do Tipo 2, que é a base de dados com apenas o valor \textit{Google Trends} do Bitcoin com x amostras por linha e frequência de 1 dia. Essa configuração gerou resultado de 50,17\%.

Para os dados vencedores do Tipo 3, que é a base de dados com a concatenação de amostras o valor do preço do Bitcoin e do valor do \texttt{Google Trends} do termo Bitcoin, com 6, onde 3 são do valor do preço do Bitcoin e 3 \texttt{Google Trends} do termo Bitcoin amostras por linha e frequência de 1 dia, ou seja, essa base de dados representa dados de 3 dias. Essa configuração gerou resultado de 51,64\%.

A tabela \ref{tab:compamazon} apresenta o resultado da comparação 

\begin{table}[!ht]
\centering

\caption{Tabela Comparação do resultado da Amazon Machine Learning (AML) e da metodologia proposta neste trabalho}
\label{tab:compamazon}
\begin{tabular}{|M{0.8cm}|M{2.4cm}|M{2.6cm}|M{2.5cm}|}
\hline
 Tipo&AML Acurácia (\%)&Este trabalho Acurácia (\%)&Diferença (\%)\\\hline
 1&47,17&50,70&+3,53\\
 2&50,17&49,30&-0,87\\
 3&51,64&53,99&+2,35\\\hline
\end{tabular}
\begin{center}
	    Fonte: O autor (2018)
	\end{center}
\end{table}

\subsection{Teoria do mercado eficiente}

%Feito
%\dbh{eu esperava uma discussão muito mais profunda aqui, recheada de referências!}

A Teoria de mercado eficiente, proposta por \cite{malkiel1970efficient} em 1969, é uma Teoria que define uma mercado eficiente como aquele que sempre consegue refletir o preço do ativo de acordo quantidade de informações disponíveis desse ativo. De acordo com essa Teoria, é impossível alguém conseguir vencer o mercado, pois todas as informações disponíveis para precificar o ativo já estão disponíveis para todos os investidores. 

Na Teoria de mercado eficiente, como explica \cite{junior2004mercados}, sempre que uma informação nova fica disponível, automaticamente o mercado se adapta para ajustar o valor do preço do ativo, ou seja, o mercado sempre tende ao preço real de um ativo, ou valor intrínseco, com raras exceções \cite{jensen1978some}.

Dentro da teoria de mercado eficientes, como explicado em \cite{basu1977investment}, diversos estudo empíricos apontam que o preço refletem conjunto particular de informações  disponíveis sobre determinado ativo, sendo assim possível classificar 3 diferentes formas de eficiência:

\begin{itemize}
    \item Forma fraca: onde apenas informações do preço histórico dos ativos é levado em conta;
    \item forma semi-forte: onde a preocupação e se os preços vão se ajustar eficientemente a novas informações que surgirem, essas informações têm que ser, obviamente, públicas;
    \item forma forte: onde a preocupação é se mesmo que alguns investidores ou grupos (e.g., administradores de fundos de investimento) tenham informações privilegiadas relevantes para o preço, se o preço vai se adaptar a essas novas informações.
\end{itemize}

Essas classificações servem para identificar o montante de informação para cada uma das formas e conseguir distinguir quando elas falham. Cabe ressaltar os termos \textbf{preço real} ou \textbf{valor intrínseco} são aproximações, e não valores perfeitamente exatos. 

A teoria de mercado eficiente também é abordada em \cite{timmermann2004efficient} para a previsão de valores, mostrando que \textit{traders} (nome designado a investidores que compram e vendem ações em curto período de tempo, em geral no mesmo dia) tentam explorar o \textit{delay} que leva entre uma nova informação sobre um determinado ativo ficar disponível e o mercado atualizar o preço deste ativo.

Essa Teoria pode explicar a dificuldade encontrada tanto em \cite{mcnally2016predicting}, quanto neste trabalho em conseguir prever os valores do Bitcoin, mesmo com o uso do \textit{Google Trends}, que ajuda a encontrar comportamentos de manada dos investidores. O trabalho \cite{bartos2015does} avalia o ativo Bitcoin como um mercado eficiente, chegando na conclusão que os eventos disponíveis, tanto positivos, quanto negativos, influenciam diretamente o valor do Bitcoin, de modo extremamente rápido. Isso significa que o preço fica mais baixo durante eventos negativos, e permanecendo em alta durante eventos positivos. Também mostra o Bitcoin segue a formar padrão de oferta e demanda, podendo alterar bastante o preço do ativo.