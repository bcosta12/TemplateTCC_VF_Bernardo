Download


Source

PDF
Actions

   Copy Project
   Publish as Template
   Word Count
Sync

   Dropbox
   GitHub
   Mendeley
Settings

Compiler

Main document

Spell check

Auto-complete

Auto-close Brackets

Code check

Editor theme

Overall theme

Keybindings

Font Size

Font Family

Line Height

PDF Viewer

Hotkeys

   Show Hotkeys
Menu
TemplateTCC_VF_Bernardo
B
Review
Share
Submit
History
Chat
Browsing project as of 4th Nov 2018, 2:44 am
 Label this version Compare to another version Restore to before these changes
Show all of the project history or only labelled versions.All historyLabels
Figs
TCC.bcf
TCC.run.xml
TCC.lof
TCC.pdf
TCC.lot
Siglas.tex
TCC.tex
abnt-alf.bst
abnt.bst
abntex2.cls
abntex2abrev.sty
abntex2cite.sty
apateste.bst
apateste.bst.tex
apendiceA_TCC.tex
apendiceC_TCC.tex
apendiceNormas.tex
biblio.bib
bibliografia.bib
cap1.tex
cap3.tex
capaNew.tex
elsarticle-harv.bst
elsarticle.cls
model3-num-names.bst
model3-num-names11.bst.tex
model3-num-namesold.bst.tex
tocloft.sty
conclusao.tex
csv
Figuras
cap5.tex
cap4.tex
cap2.tex
Today
Edited cap1.tex
Edited cap2.tex
Edited cap4.tex
4:07 am • You
Edited cap1.tex
Edited cap2.tex
3:59 am • You
Edited bibliografia.bib
Edited cap1.tex
3:54 am • You
Edited bibliografia.bib
Edited cap2.tex
3:47 am • You
Edited bibliografia.bib
Edited cap1.tex
3:42 am • You
Edited cap1.tex
3:34 am • You
Edited cap1.tex
3:29 am • You
Edited cap2.tex
3:22 am • You
Edited cap2.tex
3:17 am • You
Edited cap2.tex
3:08 am • You
Edited cap2.tex
3:02 am • You
Edited bibliografia.bib
Edited cap4.tex
2:44 am • You
Edited bibliografia.bib
Edited cap3.tex
2:27 am • You
Edited bibliografia.bib
Edited cap4.tex
2:16 am • You
Edited bibliografia.bib
Edited cap2.tex
2:00 am • You
Edited cap2.tex
1:48 am • You
Edited cap2.tex
Edited cap4.tex
1:42 am • You
Edited cap2.tex
1:37 am • You
Edited cap2.tex
1:32 am • You
Edited cap2.tex
1:27 am • You
Yesterday
Edited cap4.tex
8:40 pm • karineribeiroc
Edited cap4.tex
8:34 pm • karineribeiroc
Edited cap4.tex
8:29 pm • karineribeiroc
Edited cap4.tex
8:20 pm • karineribeiroc
Edited cap4.tex
8:15 pm • karineribeiroc
Edited cap4.tex
8:06 pm • karineribeiroc
Edited cap4.tex
7:58 pm • karineribeiroc
Edited cap3.tex
7:52 pm • karineribeiroc
Edited cap2.tex
12:55 am • You
Edited cap2.tex
12:50 am • You
Edited cap1.tex
Edited cap2.tex
12:44 am • You
Edited biblio.bib
Edited bibliografia.bib
Edited cap1.tex
12:39 am • You
Edited biblio.bib
12:33 am • You
Deletedcap6.tex
12:24 am • You
Edited TCC.tex
Edited capaNew.tex
12:24 am • You
Edited capaNew.tex
12:18 am • You
Edited cap4.tex
Edited capaNew.tex
12:13 am • You
Edited cap4.tex
12:08 am • You
Yesterday
Edited cap4.tex
11:49 pm • You
Edited cap4.tex
11:43 pm • You
Edited cap4.tex
11:34 pm • You
Edited cap4.tex
11:27 pm • You
Edited cap4.tex
8:17 pm • You
Thu, 1st Nov 18
Edited cap3.tex
9:06 pm • karineribeiroc
Edited cap3.tex
6:14 pm • karineribeiroc
Edited cap2.tex
Edited cap3.tex
6:09 pm • karineribeiroc
Edited cap2.tex
6:04 pm • karineribeiroc
Edited cap2.tex
5:37 pm • karineribeiroc
Edited cap1.tex
4:20 pm • karineribeiroc
Edited cap1.tex
4:08 pm • karineribeiroc
Wed, 31st Oct 18
Edited cap1.tex
8:18 pm • karineribeiroc
Edited cap1.tex
8:10 pm • karineribeiroc
Tue, 30th Oct 18
Edited cap4.tex
11:55 pm • You
Edited cap4.tex
11:49 pm • You
Edited cap4.tex
11:35 pm • You
Edited cap4.tex
11:29 pm • You
Edited cap3.tex
Edited cap4.tex
11:22 pm • You
Edited cap4.tex
11:15 pm • You
Edited cap4.tex
11:08 pm • You
Edited cap4.tex
11:03 pm • You
Edited cap4.tex
12:15 am • You
Mon, 29th Oct 18
Edited cap4.tex
11:56 pm • You
Edited cap4.tex
11:50 pm • You
Edited cap4.tex
11:44 pm • You
Edited cap4.tex
11:38 pm • You
Edited cap4.tex
11:30 pm • You
Edited cap4.tex
11:25 pm • You
Edited cap4.tex
11:11 pm • You
Edited cap4.tex
11:06 pm • You
Edited TCC.tex
Edited cap4.tex
11:01 pm • You
Edited cap4.tex
10:55 pm • You
Edited cap4.tex
10:49 pm • You
Edited cap4.tex
10:40 pm • You
Edited cap4.tex
10:33 pm • You
Edited cap4.tex
10:28 pm • You
Edited cap4.tex
10:23 pm • You
Edited cap4.tex
10:17 pm • You
Edited cap4.tex
9:48 pm • You
Edited cap4.tex
9:39 pm • You
Edited TCC.tex
Edited cap4.tex
9:24 pm • You
Edited cap4.tex
9:02 pm • You
Edited cap4.tex
8:56 pm • You
Edited cap4.tex
8:51 pm • You
Edited TCC.tex
Edited cap4.tex
8:35 pm • You
RenamedCap2.tex → cap2.tex
RenamedCap4.tex → cap4.tex
RenamedCap5.tex → cap5.tex
RenamedCap6.tex → cap6.tex
8:30 pm • You
Edited TCC.tex
Edited biblio.bib
Edited bibliografia.bib
Edited capaNew.tex
8:30 pm • You
Edited TCC.tex
8:24 pm • You
Deletedteste.png
Deletedrsz_tom_cruise.png
Deletedrsz_image.png
Deletedlstm_chain.png
DeletedLSTM3-SimpleRNN.png
Deletedgrafico.png
DeletedFigure_1.png
Deletedcodigo_treinar.py
CreatedFiguras/Cap2/neuronio.png
Createdcsv/bitcoin_coindesk_12m.csv
Createdcsv/ethereum_coindesk_12m.csv
Createdcsv/bcash_gtrends_12m.csv
Createdcsv/bcash_investing_12m.csv
Createdcsv/ripple_gtrends_12m.csv
Createdcsv/1d.csv
Createdcsv/gt.csv
Createdcsv/price_n.csv
Createdcsv/bitcoin_gtrends_12m.csv
Createdcsv/ethereum_gtrends_12m.csv
Createdcsv/litecoin_coinmarketcap_12m.csv
Createdcsv/price.csv
Createdcsv/2d.csv
Createdcsv/gt_n.csv
Createdcsv/ripple_investing_12m.csv
Createdcsv/litecoin_gtrends_12m.csv
DeletedFiguras/csv/bitcoin_coindesk_12m.csv
DeletedFiguras/csv/ethereum_coindesk_12m.csv
DeletedFiguras/csv/bcash_investing_12m.csv
DeletedFiguras/csv/ripple_gtrends_12m.csv
DeletedFiguras/csv/gt.csv
DeletedFiguras/csv/price_n.csv
DeletedFiguras/csv/bitcoin_gtrends_12m.csv
DeletedFiguras/csv/ethereum_gtrends_12m.csv
DeletedFiguras/csv/litecoin_coinmarketcap_12m.csv
DeletedFiguras/csv/price.csv
DeletedFiguras/csv/gt_n.csv
DeletedFiguras/csv/ripple_investing_12m.csv
DeletedFiguras/csv/litecoin_gtrends_12m.csv
DeletedFiguras/csv/neuronio.png
DeletedFiguras/csv/bcash_gtrends_12m.csv
DeletedFiguras/csv/2d.csv
DeletedFiguras/csv/1d.csv
RenamedFiguras/bitcoin_coindesk_12m.csv → Figuras/csv/bitcoin_coindesk_12m.csv
RenamedFiguras/ethereum_coindesk_12m.csv → Figuras/csv/ethereum_coindesk_12m.csv
RenamedFiguras/bcash_investing_12m.csv → Figuras/csv/bcash_investing_12m.csv
RenamedFiguras/ripple_gtrends_12m.csv → Figuras/csv/ripple_gtrends_12m.csv
RenamedFiguras/gt.csv → Figuras/csv/gt.csv
RenamedFiguras/price_n.csv → Figuras/csv/price_n.csv
RenamedFiguras/bitcoin_gtrends_12m.csv → Figuras/csv/bitcoin_gtrends_12m.csv
RenamedFiguras/ethereum_gtrends_12m.csv → Figuras/csv/ethereum_gtrends_12m.csv
RenamedFiguras/litecoin_coinmarketcap_12m.csv → Figuras/csv/litecoin_coinmarketcap_12m.csv
RenamedFiguras/price.csv → Figuras/csv/price.csv
RenamedFiguras/gt_n.csv → Figuras/csv/gt_n.csv
RenamedFiguras/ripple_investing_12m.csv → Figuras/csv/ripple_investing_12m.csv
RenamedFiguras/litecoin_gtrends_12m.csv → Figuras/csv/litecoin_gtrends_12m.csv
RenamedFiguras/neuronio.png → Figuras/csv/neuronio.png
RenamedFiguras/bcash_gtrends_12m.csv → Figuras/csv/bcash_gtrends_12m.csv
RenamedFiguras/2d.csv → Figuras/csv/2d.csv
RenamedFiguras/1d.csv → Figuras/csv/1d.csv
CreatedFiguras/bitcoin_coindesk_12m.csv
CreatedFiguras/ethereum_coindesk_12m.csv
CreatedFiguras/bcash_gtrends_12m.csv
CreatedFiguras/bcash_investing_12m.csv
CreatedFiguras/ripple_gtrends_12m.csv
CreatedFiguras/1d.csv
CreatedFiguras/gt.csv
CreatedFiguras/price_n.csv
CreatedFiguras/bitcoin_gtrends_12m.csv
CreatedFiguras/ethereum_gtrends_12m.csv
CreatedFiguras/litecoin_coinmarketcap_12m.csv
CreatedFiguras/price.csv
CreatedFiguras/2d.csv
CreatedFiguras/gt_n.csv
CreatedFiguras/ripple_investing_12m.csv
CreatedFiguras/litecoin_gtrends_12m.csv
CreatedFiguras/neuronio.png
8:21 pm • You
Deletedarticle_cite/Capturar1.PNG
Deletedarticle_cite/Capturar.PNG
Deletedarticle_cite/Capturar4.PNG
Deletedarticle_cite/Capturar3.PNG
Deletedarticle_cite/Capturar2.PNG
8:15 pm • You
Edited Cap2.tex
Edited Cap4.tex
Edited Cap5.tex
Edited Cap6.tex
Edited cap1.tex
Edited cap3.tex
8:14 pm • You
Createdconclusao.tex
CreatedCap6.tex
CreatedCap5.tex
CreatedCap4.tex
CreatedCap2.tex
Createdtocloft.sty
Createdmodel3-num-namesold.bst.tex
Createdmodel3-num-names11.bst.tex
Createdmodel3-num-names.bst
Createdelsarticle.cls
Createdelsarticle-harv.bst
CreatedcapaNew.tex
Createdcap3.tex
Createdcap1.tex
Createdbibliografia.bib
Createdbiblio.bib
CreatedapendiceNormas.tex
CreatedapendiceC_TCC.tex
CreatedapendiceA_TCC.tex
Createdapateste.bst.tex
Createdapateste.bst
Createdabntex2cite.sty
Createdabntex2abrev.sty
Createdabntex2.cls
Createdabnt.bst
Createdabnt-alf.bst
CreatedTCC.tex
CreatedSiglas.tex
CreatedLSTM3-SimpleRNN.png
Createdlstm_chain.png
CreatedTCC.lot
CreatedTCC.pdf
Createdcodigo_treinar.py
Createdrsz_tom_cruise.png
Createdrsz_image.png
Createdteste.png
Createdgrafico.png
CreatedTCC.lof
CreatedTCC.run.xml
CreatedFigure_1.png
CreatedTCC.bcf
Createdarticle_cite/Capturar1.PNG
Createdarticle_cite/Capturar.PNG
Createdarticle_cite/Capturar4.PNG
Createdarticle_cite/Capturar3.PNG
Createdarticle_cite/Capturar2.PNG
CreatedFigs/logoEngComp.png
CreatedFigs/logoCefetCampusPetropolis-eps-converted-to.pdf
CreatedFigs/biblioteca.jpeg
CreatedFigs/logoEngComp.eps
CreatedFigs/republica.eps
CreatedFigs/republica.png
CreatedFigs/pedestres.jpg
CreatedFigs/logoCefetCampusPetropolis.eps
CreatedFigs/biblioteca_engcomp.jpeg
CreatedFigs/logoCefetRio.eps
CreatedFigs/logoCefetCampusPetropolis.jpg
CreatedFigs/logo.PNG
CreatedFigs/logoCefetRio-eps-converted-to.pdf
8:13 pm • You

1
2
3
4
5
6
7
8
9
10
11
12
13
14
15
16
17
18
19
20
21
22
23
\section{Introduçao}
 \subsection{Séries Temporais}
 \label{sec:SeriesTemporais}
Séries temporais são definidas como ``qualquer conjunto de observações ordenadas no tempo'' 
    \cite{morettin2006analise}. Elas são observadas em problemas envolvendo diversas áreas de 
    conhecimento, tais como: meteorologia \cite{7982030}, processamento de imagens 
    \cite{7729869,7164182, 6723283}, processamento de vídeos \cite{6469509}, marca d'água 
    digital\footnote{\textit{Digital watermarking} - ou, em português, marca d'água digital é 
    uma técnica esteganográfica de ocultação de informação.} \cite{7024611}, agricultura 
    \cite{6723610}, medicina \cite{6707296} economia \cite{4810671}, entre outros.
 
 \subsection{Tipos de Séries Temporais}
 Existem séries temporais \textit{discretas} e \textit{contínuas}. As discretas costumam ser 
     obtidas através de amostragem de uma série contínua a intervalos de tempo regulares 
     \cite{morettin2006analise,hyndman2018forecasting}. Por exemplo, para analisar a série do 
     valor da temperatura de uma cidade ao longo de um ano, será preciso amostrá-la a 
     intervalos, obtendo uma lista de valores. Esse processo converte uma série contínua em uma 
     série discreta.
Os estudos de séries temporais estão geralmente relacionados a dois domínios: o domínio do 
    tempo e o domínio da frequência. Ambos os enfoques estão interessados em construir modelos 
    para séries. Os modelos no domínio do tempo em geral são paramétricos, ou seja, têm um 
    número finito de parâmetros \cite{conover1981rank}. Já os modelos no domínio da frequência 
    são classificados como não-paramétricos \cite{hollander1999nonparametric}. Em termos mais 
    rigorosos, modelos paramétricos são aqueles que assumem um conjunto finito de parâmetros 
    que cumpre estimar, o que limita a complexidade do modelo mesmo quando a informação contida 
    nos dados é arbitrariamente grande. Já os modelos não paramétricos assumem que a 
    distribuição dos dados depende de um conjunto de parâmetros de dimensionalidade infinita. 
    Estes modelos capturam mais informação dos dados à medida que a disponibilidade de 
    informação aumenta, o que os torna mais flexíveis \cite{ChenNeural2001}.
 
\subsection{Notação}
Este trabalho interpretará uma série temporal como um vetor, cujos elementos consistem de 
    amostras no tempo $t$. Admitindo um total de $r$ elementos, podemos definir o vetor 
    $\boldsymbol{z}(t) \in \mathbb{R}^n$ que contém uma série temporal como
\begin{equation}
\boldsymbol{z}(t) \triangleq \begin{bmatrix}z(t_1) & z(t_2) & \ldots & z(t_n)\end{bmatrix}.
\end{equation}
%\dbh{Prefiro a seguinte notação: vetores em negrito e minúsculas, escalares em fonte normal e 
    matrizes em negrito e em maiúsculas}.
 
 \subsection{Estacionariedade}
 
 Uma série temporal é dita estacionária de primeira ordem quando se desenvolve no tempo 
     aleatoriamente ao redor de uma média constante. A maioria dos procedimentos de análise 
     estatística de séries temporais supõe que estas sejam estacionárias. Caso tal hipótese não 
     se confirme, será necessário efetuar alguma transformação nos dados originais de sorte a 
     realçar a estacionariedade da série transformada \cite{grenander1957statistical}.
 A transformação mais comum consiste em tomar diferenças sucessivas da série original. Esta 
     técnica é conhecida como diferenciação. Seu objetivo consiste em ajudar a estabilizar a 
     média de uma série temporal, removendo as alterações bruscas no nível de uma série 
     temporal e, portanto, eliminando (ou reduzindo) a tendência e a sazonalidade 
     \cite{morettin2006analise, hyndman2018forecasting}.
