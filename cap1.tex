\section{Introduçao}

 \subsection{Séries Temporais}
 \label{sec:SeriesTemporais}
Séries temporais são definidas como ``qualquer conjunto de observações ordenadas no tempo'' \cite{morettin2006analise}. Elas são observadas em problemas envolvendo diversas áreas de conhecimento, tais como: meteorologia \cite{7982030}, processamento de imagens \cite{7729869,7164182, 6723283}, processamento de vídeos \cite{6469509}, marca d'água digital\footnote{\textit{Digital watermarking} - ou, em português, marca d'água digital é uma técnica esteganográfica de ocultação de informação.} \cite{7024611}, agricultura \cite{6723610}, medicina \cite{6707296} economia \cite{4810671}, entre outros.
 
 \subsection{Tipos de Séries Temporais}
 Existem séries temporais \textit{discretas} e \textit{contínuas}. As discretas costumam ser obtidas através de amostragem de uma série contínua a intervalos de tempo regulares \cite{morettin2006analise,hyndman2018forecasting}. Por exemplo, para analisar a série do valor da temperatura de uma cidade ao longo de um ano, será preciso amostrá-la a intervalos, obtendo uma lista de valores. Esse processo converte uma série contínua em uma série discreta.

Os estudos de séries temporais estão geralmente relacionados a dois domínios: o domínio do tempo e o domínio da frequência. Ambos os enfoques estão interessados em construir modelos para séries. Os modelos no domínio do tempo em geral são paramétricos, ou seja, têm um número finito de parâmetros \cite{conover1981rank}. Já os modelos no domínio da frequência são classificados como não-paramétricos \cite{hollander1999nonparametric}. Em termos mais rigorosos, modelos paramétricos são aqueles que assumem um conjunto finito de parâmetros que cumpre estimar, o que limita a complexidade do modelo mesmo quando a informação contida nos dados é arbitrariamente grande. Já os modelos não paramétricos assumem que a distribuição dos dados depende de um conjunto de parâmetros de dimensionalidade infinita. Estes modelos capturam mais informação dos dados à medida que a disponibilidade de informação aumenta, o que os torna mais flexíveis \cite{ChenNeural2001}.
 
\subsection{Notação}
Este trabalho interpretará uma série temporal como um vetor, cujos elementos consistem de amostras no tempo $t$. Admitindo um total de $r$ elementos, podemos definir o vetor $\boldsymbol{z}(t) \in \mathbb{R}^n$ que contém uma série temporal como
\begin{equation}
\boldsymbol{z}(t) \triangleq \begin{bmatrix}z(t_1) & z(t_2) & \ldots & z(t_n)\end{bmatrix}.
\end{equation}
%\dbh{Prefiro a seguinte notação: vetores em negrito e minúsculas, escalares em fonte normal e matrizes em negrito e em maiúsculas}.
 
 \subsection{Estacionariedade}
 
 Uma série temporal é dita estacionária de primeira ordem quando se desenvolve no tempo aleatoriamente ao redor de uma média constante. A maioria dos procedimentos de análise estatística de séries temporais supõe que estas sejam estacionárias. Caso tal hipótese não se confirme, será necessário efetuar alguma transformação nos dados originais de sorte a realçar a estacionariedade da série transformada \cite{grenander1957statistical}.

 A transformação mais comum consiste em tomar diferenças sucessivas da série original. Esta técnica é conhecida como diferenciação. Seu objetivo consiste em ajudar a estabilizar a média de uma série temporal, removendo as alterações bruscas no nível de uma série temporal e, portanto, eliminando (ou reduzindo) a tendência e a sazonalidade \cite{morettin2006analise, hyndman2018forecasting}.
 

 
 A primeira diferença é definida por:
\begin{equation} 
\delta_1(t) \triangleq z(t) - z(t - 1),
\end{equation}
para $t \in \{2,\ldots,r\}$. A sequência obtida pela segunda diferença pode ser descrita como
\begin{equation}
\delta_2(t) = \delta_1\{\delta_1(t)\} = \delta_1[z(t) - z(t - 1)],
\end{equation}
ou, equivalentemente:
\begin{equation}
\delta_2(t) \triangleq z(t) - 2z(t - 1) + z(t - 2).
\end{equation}
De modo geral, a $n$-ésima diferença de $\boldsymbol{z}(t)$ será:
\begin{equation}
\delta_n\boldsymbol{z}(t) = \delta_1[\delta_{n-1}\boldsymbol{z}(t)].
\end{equation}
Em situações normais, será necessário efetuar uma ou duas diferenças para que a série se torne estacionária \cite{morettin2006analise}.

\subsection{Objetivo da análise de séries temporais}

O processamento de séries temporais, em geral, é empregado para realizarmos procedimentos de previsão, eliminação de ruído, predição e análise de valores \cite{morettin2006analise, hyndman2018forecasting}. Nesse escopo, os principais objetivos da análise de séries temporais são:

\begin{itemize}
\item investigar o mecanismo gerador da série; por exemplo, desejamos entender como os valores de uma determinada série foram gerados;
\item verificação de comportamento; neste caso desejamos identificar a existência de tendências, ciclos e variações sazonais;
\item procurar periodicidades relevantes nos dados;
\item previsão (foco deste trabalho); por exemplo, desejamos usar os valores anteriores para prever qual o próximo valor da série temporal.
\end{itemize}

\subsection{Previsão em Séries Temporais}

O problema de previsão é um dos principais objetivos da análise de séries temporais. Prever o(s) próximo(s) valor(es) de uma série utilizando valores pregressos é um problema que pode ser usado em diversas aplicações, sendo amplamente explorado na literatura \cite{weigend2018time}. Tal técnica é de grande interesse nas mais distintas áreas, com resultados que auxiliam o processo de tomada de decisão \cite{morettin2006analise, hyndman2018forecasting}. Neste trabalho estaremos interessados no problema de previsão de valores de uma cripto-moeda, o Bitcoin \cite{nakamoto2008bitcoin}, cujo problema será explicado nas próximas subseções.

O problema de previsão de séries temporais pode ser descrito como encontrar o valor da série temporal $z(t)$ no instante $t + h$, onde $t$ é o instante de origem e $h$ o \textit{horizonte}. Por exemplo, considere que a série temporal $z(t)$ representa os valores do real frente ao dólar e que suas amostras estão espaçadas no período de um dia. Se quisermos prever o valor do próximo dia do Real frente ao Dólar, teremos $h = 1$ e calcularemos $z(t + 1)$ . Se quisermos prever o valor do real frente ao Dólar na próxima semana, teremos $h = 7$ e estimaremos $z(t + 7)$; e assim por diante. 

Por questão de simplicidade nas notações, iremos denotar a estimativa de $z(t+h)$ como $\hat{z}(h)$. Por exemplo, se a série temporal $z(t)$ representa os valores do real frente ao dólar, para representar o valor desta série na próxima semana (ou seja, $z(t + 7)$, escreveremos simplesmente $\hat{z}(7)$. 

Levando em conta essas notações, o erro quadrático médio\footnote{MSE, do inglês \emph{mean-squared error}.} da previsão pode ser descrito como:
\begin{equation}
\text{MSE} \triangleq \mathbb{E}\left\{\left[z(t + h) - \hat{z}(t+h)\right]^2\right\},
\end{equation}
onde $\mathbb{E}[\cdot]$ é o operador de esperança estatística.

\subsection{Aplicações de previsões de séries temporais}
O estado da arte no tratamento de séries temporais, como foi dito na Seção \ref{sec:SeriesTemporais}, contempla diversas áreas do conhecimento e, consequentemente, diversos conhecimentos específicos de cada área são empregados como suporte ao modelo de previsão proposto. 

O trabalho \cite{8412670}, por exemplo, efetua uma previsão para o pico de carga em circuitos de distribuição. Para fazer previsão da carga de pico são utilizadas árvores de regressão aditiva Bayesiana (BART). O método é comparado com o método de regressão linear múltipla e o método de regressão vetorial de suporte com base na área residencial com dados de carga pública do utilitário do Texas.

Já o trabalho \cite{8412036} implementa uma rede neural para previsão de curto prazo da produção de energia em uma usina fotovoltaica. A estrutura dessas redes neurais é otimizada com um algoritmo genético que seleciona os valores para os principais parâmetros da rede neural e as variáveis usadas como entradas. Essas variáveis de entrada são selecionadas entre um conjunto de variáveis que inclui variáveis meteorológicas, cronológicas, astronômicas e previstas relacionadas à localização da usina.

O trabalho \cite{8413116} explora o problema de otimização do ajuste hierárquico de carga, onde a carga elétrica é organizada hierarquicamente com base na geografia, o que requer previsões hierárquicas que abrangem todos os níveis das operações do sistema de energia. Uma maneira trivial de implementar previsões hierárquicas consiste em gerar previsões de carga em cada nível de forma independente. No entanto, essas previsões geradas de forma independente podem não satisfazer estruturas hierárquicas, ou seja, a soma das previsões de nível inferior não pode ser adicionado exatamente nas previsões de nível superior. Para resolver esse o problema, o trabalho \cite{8413116} apresenta um modelo de programação quadrática (QP) para ajustar otimamente as previsões de carga geradas independentemente em cada nível de uma hierarquia.

As tecnologias para \textit{Smart Grids} também podem ser amparadas mediante o concurso de modelos de previsão de séries temporais. Classicamente, um modelo de \emph{Smart Grid} permite que diferentes fontes de energia participem da rede \cite{8408964}. Além disso, a demanda de carga também varia continuamente, por isso a previsão de carga é de importância crucial para a operação, manutenção e planejamento adequados do sistema de energia elétrica. A referência \cite{8408964} apresenta uma abordagem para esse problema, utilizando Sistemas Adaptativos de Inferência Neuro-difusa.

O trabalho \cite{8410563} investiga como a previsão dos efeitos de ressonância pode impactar um pneu, onde a intensidade de vibração do pneu está relacionada com a taxa de atrito com dependência de longo alcance. As características dinâmicas dos pneus permitem prever a tendência dos sinais de vibração.

Aprendizado de máquina e algoritmos de regressão são usados no trabalho \cite{8403331} para prever cargas de aquecimento e resfriamento em uma rede \textit{district heating and cooling systems}. Os dados são modelados para prever a carga de consumo de aquecimento e resfriamento em uma rede DHC.

O trabalho \cite{8408773} analisa uma série de algoritmos de inteligência artificial, como redes neurais, para prever a velocidade do vento, o que é importante para a indústria no planejamento da operação de sistemas de energia renovável.

A previsão de preços de eletricidade é de grande importância, especialmente na construção do mercado de energia em todo o mundo. O trabalho \cite{8403372} propôs uma abordagem de previsão combinada paralela com o modelo ARIMA \cite{box1970time}, que consiste em ajustar modelos auto-regressivos integrados de médias móveis, redes neurais, entre outros, com base na classificação sazonal para previsão de preço de eletricidade (DAEPF) no dia seguinte. 

A previsão de séries temporais associadas a preço de ações também é um problema recorrente na previsão de séries temporais, explorado por exemplo em \cite{8410278, 7310722}. Eis um problema clássico para a previsão de séries temporais, onde em geral é utilizado o valor do preço de uma ação para estimar quanto será o preço futuro do ativo ou ainda classificar se o ativo irá se valorizar ou não. Abordagens com redes neurais são comuns e foram aplicadas em \cite{8410278}. Já o trabalho \cite{7310722} utiliza algoritmos tradicionais de clusterização para a previsão.

\subsection{Série Temporal: O Bitcoin}
 
 O Bitcoin \cite{nakamoto2008bitcoin} é uma versão \emph{peer-to-peer} de dinheiro eletrônico que permite pagamentos \emph{online} que são enviados diretamente de uma parte para outra sem necessidade de intermédio de uma instituição financeira. 
 
 Para realizar a validação das transações é necessário o emprego de assinaturas digitais que evitam que o mesmo montante de Bitcoins seja enviado para duas pessoas ao mesmo tempo, fenômeno conhecido como \textit{double-spending}. Para  resolver o problema de \textit{ double-spending}, o Bitcoin usa uma rede \emph{peer-to-peer} que salva o horário das transações, calculando o seu HASH (atualmente consoante o padrão Double-SHA256 \cite{bitcoinwikihashcash}), e salvando em uma estrutura de dados que hoje é conhecida como \textit{Blockchain}.
 
O Blockchain, como o próprio nome diz, é uma corrente de blocos, na qual o bloco atual é ligado ao bloco anterior por meio do cálculo de HASH do bloco anterior. 

Para validar o HASH de cada bloco, o Bitcoin usa um \textit{proof-of-work} que consiste em calcular o Double-HASH-256 de cada bloco. Ou seja, o nó deverá calcular qual frase gerará o HASH recebido.

Para executar a rede do Bitcoin os seguintes passos são necessários:
\begin{enumerate}
\item Novas transações são enviadas em \textit{broadcast} para todos os nós da rede;
\item Cada nó transforma a nova transação em um bloco;
\item Cada nó trabalha achando a \textit{proof-of-work} daquele bloco;
\item Quando o nó achar o \textit{proof-of-work}, o resultado é enviado em \textit{broadcast} para todos os nós dentro do novo bloco;
\item Os nós aceitam o bloco apenas se todas as transações nele forem válidas e não foram feitas ainda;
\item Os nós mostram que aceitaram o novo bloco criando o próximo bloco da sua ``corrente'', usando o HASH do bloco que foi aceito.
\end{enumerate}

Nesse trabalho empregaremos a Série Temporal histórica do Bitcoin. O Bitcoin começou a ser negociado em meados de 2009 sem valor comercial, passou a barreira de US\$ 1,00 no começo de 2011, chegando a incríveis US\$ 19.990,00 no final de 2017.

O foco deste trabalho consiste na análise e predição das séries temporais do valor do Bitcoin, possivelmente correlacionadas com a série de tendência de busca do termo ``Bitcoin'' nos mecanismos de busca da internet. Trabalhos anteriores mostraram a forte relação entre essas duas séries. O trabalho \cite{matta2015bitcoin} chegou a mostrar uma  correlação linear cruzada de 64\% entre a série do \textit{Google Trends} e a série do valor do preço do Bitcoin. Além disso, o trabalho apresentado em \cite{wilson_yelowitz_2014} também realizou análises de correlação cruzada que sugeriram uma forte dependência estatística entre as duas séries.

Usando também essa relação, \cite{mcnally2016predicting} obteve acurácia de 52,78\% na previsão do valor do Bitcoin, com o algoritmo \textit{Long Short Term Memory} (LSTM), utilizando dados do valor do preço de fechamento do Bitcoin entre 19 de Agosto de 2013 e 19 de Julho de 2016. Embora esta taxa pareça baixa, cabe ressaltar que a obtenção de uma taxa de acerto ligeiramente acima de 50\% pode, ao longo do tempo, implicar grande retorno financeiro. Esta acurácia de 52,78\% também explicita o quão desafiadora é a tarefa enfocada neste trabalho.


\subsection{Tema}

O tema do trabalho é o desenvolvimento de um algoritmo de inteligência artificial para auxiliar na tomada de decisão de compra ou de venda do Bitcoin, a partir de dados da série temporal do histórico do preço do Bitcoin \cite{nakamoto2008bitcoin} e da quantidade de vezes que o termo foi buscado no Google, utilizando a ferramenta \textit{Google Trends}. O problema a ser resolvido consiste em prever se o preço futuro do Bitcoin irá subir ou descer em determinado espaço de tempo.

O algoritmo proposto, baseado numa estrutura neural treinada pelo concurso do algoritmo \emph{Backpropagation} \cite{hecht1992theory}, pretende superar a acurácia obtida por \cite{mcnally2016predicting}. Cumpre ressaltar que tal algoritmo apresenta a capacidade de treinar as camadas intermediárias da rede neural, calculando o gradiente que será usado para ajustar os pesos das camadas escondidas. Tal procedimento é indicado para identificação de relações não lineares, como as que a série histórica do Bitcoin provavelmente apresenta. Afinal de contas, há diversos fatores que concorrem para aumentar ou reduzir a cotação desta cripto-moeda e seria demasiado ingênuo supor que todos estes fatores possam ser descritos por operadores puramente lineares. 


\subsection{Delimitação}

O objeto de estudo é a Série Temporal do histórico do preço do Bitcoin e como a tendência das buscas do termo "Bitcoin", capturada pelo \textit{Google Trends}, pode treinar uma rede neural classificadora para que ela determine se o valor do preço Bitcoin irá subir ou diminuir.

Existem dezenas de cripto-moedas no mercado atualmente. Esse trabalho focará, em primeiro momento, apenas no Bitcoin, pois busca explorar a ligação que o valor do Bitcoin tem com as tendências de busca do termo "Bitcoin", como mostrado em \cite{matta2015bitcoin}. 

As Figuras \ref{fg:comparacao_bitcoin_12m}, \ref{fg:comparacao_ethereum_12m}, \ref{fg:comparacao_litecoin_12m}, \ref{fg:comparacao_ripple_12m}, \ref{fg:comparacao_bcash_12m} apresentam a comparação entre a série do preço das cripto-moedas Bitcoin \cite{nakamoto2008bitcoin}, Ethereum \cite{Ethereum}, Litecoin \cite{Litecoin}, Ripple \cite{Ripple} e Bitcoin Cash \cite{BitcoinCash}, e os dados do \textit{Google Trends} referentes ao nome das cripto-moedas, nos últimos 12 meses, com frequência de 7 dias, onde o valor é medido em Dólares e o valor do \textit{Google Trends} é medido em percentagem, ou seja, 0\% se no período a busca relativa ao termo foi mínima e 100\% se foi máxima. Essas são as 5 das cripto-moedas mais relevantes da atualidade. 
Analisando os dois gráficos de cada cripto-moeda é possível perceber como os dois gráficos apresentam um formato parecido, especialmente no Bitcoin, que é a cripto-moeda mais famosa. Essa semelhança mostra uma possível relação entre os dois valores, relação esta que será explorada nesse trabalho, por meio das estatísticas oriundas do \textit{Google Trends} para prever o valor da cripto-moeda. É possível perceber que os outros dois gráficos das cripto-moedas restantes não têm uma correlação tão forte como a apresentada pela série do Bitcoin (o que coincide com \cite{matta2015bitcoin}). O Bitcoin, por ser a cripto-moeda mais famosa, também é a que tem mais relação com a quantidade de buscas das pessoas, levando a crer que as outras cripto-moedas são negociadas por nichos mais específicos de compradores, ou seja, a pessoa que vai comprar uma cripto-moeda mais específica, como a Ripple, é uma pessoa com mais conhecimento técnico, não precisando buscar tanto o termo antes de comprar. Já o Bitcoin, como é mais conhecido, oferece diversos meios de compra, de forma mais simples, o que favorece a compra das pessoas mais leigas, colaborando com a premissa de que a quantidade de vezes que o termo ``Bitcoin'' é buscado tem relação com o valor do Bitcoin. Porém, deve ser deixado claro que o modelo proposto pode ser posteriormente usado com qualquer cripto-moeda.

%%%%%%%%%%%%%%%%%%%%%%%%%%%%%%%%%%%%%%%
\begin{center}
	\begin{figure}[H]


	\begin{subfigure}{}
		
		\begin{tikzpicture}

		\begin{axis}[
		    date coordinates in=x,
		    xticklabel={\day-\month-\year},
		    x tick label style={rotate=45,anchor=north east},
		    date ZERO=2017-05-27,
		    xmin=2017-05-27, xmax=2018-05-20,
		    ymin=1000,ymax=21000,
		    height=8cm,
			width=14cm,
			grid=major,
		    xlabel={Data (dia-mês-ano)},
		    ylabel={Valor em dólares},
		]
		\addlegendentry{Preço BTC}
		\addplot [blue, mark=*] table [x=date, y=value, col sep=comma] {bitcoin_coindesk_12m.csv};
		\end{axis}
		\end{tikzpicture}
        \caption{Valor do Bitcoin.}\label{gr:bitcoin_coindesk_12m}
	\end{subfigure}
	%sub fig 2
	\begin{subfigure}{}
		\begin{tikzpicture}

		\begin{axis}[
			title={Valor Bitcoin no Google Trends}, 
		    date coordinates in=x,
		    xticklabel={\day-\month-\year},
		    x tick label style={rotate=45,anchor=north east},
		    xmin=2017-05-27, xmax=2018-05-20,
		    ymin=0,ymax=100,
		    height=8cm,
			width=14cm,
			grid=major,
		    xlabel={Data (dia-mês-ano)},
		    ylabel={Valor \%},  
		]
		\addlegendentry{Trends BTC}
		\addplot [red, mark=*] table [x=date, y=value, col sep=comma] {bitcoin_gtrends_12m.csv};
		\end{axis}
		\end{tikzpicture}
    	\caption{Valor do Bitcoin.}\label{gr:bitcoin_gtrends_12m}
	\end{subfigure}
    \caption{Comparação dos gráfico do valor do Bitcoin e do \textit{Google Trends.}}\label{fg:comparacao_bitcoin_12m}
	\end{figure}
\end{center}
%%%%%%%%%%%%%%%%%%%%%%%%%%%%%%%%%%%%%%%


%%%%%%%%%%%%%%%%%%%%%%%%%%%%%%%%%%%%%%%
\begin{center}
	\begin{figure}[H]


	\begin{subfigure}{}
		
		\begin{tikzpicture}

		\begin{axis}[
		    date coordinates in=x,
		    xticklabel={\day-\month-\year},
		    x tick label style={rotate=45,anchor=north east},
		    date ZERO=2017-05-27,
		    xmin=2017-05-27, xmax=2018-05-20,
		    ymin=170,ymax=1400,
		    height=8cm,
			width=14cm,
			grid=major,
		    xlabel={Data (dia-mês-ano)},
		    ylabel={Valor em dólares},
		]
		\addlegendentry{Preço ETH}
		\addplot [blue, mark=*] table [x=date, y=value, col sep=comma] {ethereum_coindesk_12m.csv};
		\end{axis}
		\end{tikzpicture}
        \caption{Valor do Ethereum.}\label{gr:ethereum_coindesk_12m}
	\end{subfigure}
	%sub fig 2
	\begin{subfigure}{}
		\begin{tikzpicture}

		\begin{axis}[
			title={Valor Ethereum no Google Trends}, 
		    date coordinates in=x,
		    xticklabel={\day-\month-\year},
		    x tick label style={rotate=45,anchor=north east},
		    xmin=2017-05-27, xmax=2018-05-20,
		    ymin=0,ymax=100,
		    height=8cm,
			width=14cm,
			grid=major,
		    xlabel={Data (dia-mês-ano)},
		    ylabel={Valor \%},  
		]
		\addlegendentry{Trends ETH}
		\addplot [red, mark=*] table [x=date, y=value, col sep=comma] {ethereum_gtrends_12m.csv};
		\end{axis}
		\end{tikzpicture}
    	\caption{Valor do Ethereum.}\label{gr:ethereum_gtrends_12m}
	\end{subfigure}
    \caption{Comparação dos gráfico do valor do Ethereum e do \textit{Google Trends}.}\label{fg:comparacao_ethereum_12m}
	\end{figure}
\end{center}
%%%%%%%%%%%%%%%%%%%%%%%%%%%%%%%%%%%%%%%


%%%%%%%%%%%%%%%%%%%%%%%%%%%%%%%%%%%%%%%
\begin{center}
	\begin{figure}[H]


	\begin{subfigure}{}
		
		\begin{tikzpicture}

		\begin{axis}[
		    date coordinates in=x,
		    xticklabel={\day-\month-\year},
		    x tick label style={rotate=45,anchor=north east},
		    date ZERO=2017-05-27,
		    xmin=2017-05-27, xmax=2018-05-20,
		    ymin=20,ymax=320,
		    height=8cm,
			width=14cm,
			grid=major,
		    xlabel={Data (dia-mês-ano)},
		    ylabel={Valor em dólares},
		]
		\addlegendentry{Preço LTC}
		\addplot [blue, mark=*] table [x=date, y=value, col sep=comma] {litecoin_coinmarketcap_12m.csv};
		\end{axis}
		\end{tikzpicture}
        \caption{Valor do Litecoin.}\label{gr:litecoin_coindesk_12m}
	\end{subfigure}
	%sub fig 2
	\begin{subfigure}{}
		\begin{tikzpicture}

		\begin{axis}[
			title={Valor Litecoin no Google Trends}, 
		    date coordinates in=x,
		    xticklabel={\day-\month-\year},
		    x tick label style={rotate=45,anchor=north east},
		    xmin=2017-05-27, xmax=2018-05-20,
		    ymin=0,ymax=100,
		    height=8cm,
			width=14cm,
			grid=major,
		    xlabel={Data (dia-mês-ano)},
		    ylabel={Valor \%},  
		]
		\addlegendentry{Trends LTC}
		\addplot [red, mark=*] table [x=date, y=value, col sep=comma] {litecoin_gtrends_12m.csv};
		\end{axis}
		\end{tikzpicture}
    	\caption{Valor do Ethereum.}\label{gr:litecoin_gtrends_12m}
	\end{subfigure}
    \caption{Comparação dos gráfico do valor do Litecoin e do \textit{Google Trends}.}\label{fg:comparacao_litecoin_12m}
	\end{figure}
\end{center}
%%%%%%%%%%%%%%%%%%%%%%%%%%%%%%%%%%%%%%%
 
 
 %%%%%%%%%%%%%%%%%%%%%%%%%%%%%%%%%%%%%%%
\begin{center}
	\begin{figure}[H]


	\begin{subfigure}{}
		
		\begin{tikzpicture}

		\begin{axis}[
		    date coordinates in=x,
		    xticklabel={\day-\month-\year},
		    x tick label style={rotate=45,anchor=north east},
		    date ZERO=2017-05-27,
		    xmin=2017-05-27, xmax=2018-05-20,
		    ymin=0,ymax=3,
		    height=8cm,
			width=14cm,
			grid=major,
		    xlabel={Data (dia-mês-ano)},
		    ylabel={Valor em dólares},
		]
		\addlegendentry{Preço XRP}
		\addplot [blue, mark=*] table [x=date, y=value, col sep=comma] {ripple_investing_12m.csv};
		\end{axis}
		\end{tikzpicture}
        \caption{Valor do Ripple.}\label{gr:ripple_coindesk_12m}
	\end{subfigure}
	%sub fig 2
	\begin{subfigure}{}
		\begin{tikzpicture}

		\begin{axis}[
			title={Valor Ripple no Google Trends}, 
		    date coordinates in=x,
		    xticklabel={\day-\month-\year},
		    x tick label style={rotate=45,anchor=north east},
		    xmin=2017-05-27, xmax=2018-05-20,
		    ymin=0,ymax=100,
		    height=8cm,
			width=14cm,
			grid=major,
		    xlabel={Data (dia-mês-ano)},
		    ylabel={Valor \%},  
		]
		\addlegendentry{Trends XRP}
		\addplot [red, mark=*] table [x=date, y=value, col sep=comma] {ripple_gtrends_12m.csv};
		\end{axis}
		\end{tikzpicture}
    	\caption{Valor do Ripple.}\label{gr:ripple_gtrends_12m}
	\end{subfigure}
    \caption{Comparação dos gráfico do valor do Ripple e do \textit{Google Trends}.}\label{fg:comparacao_ripple_12m}
	\end{figure}
\end{center}
%%%%%%%%%%%%%%%%%%%%%%%%%%%%%%%%%%%%%%%
 %%%%%%%%%%%%%%%%%%%%%%%%%%%%%%%%%%%%%%%
\begin{center}
	\begin{figure}[H]


	\begin{subfigure}{}
		
		\begin{tikzpicture}

		\begin{axis}[
		    date coordinates in=x,
		    xticklabel={\day-\month-\year},
		    x tick label style={rotate=45,anchor=north east},
		    date ZERO=2017-05-27,
		    xmin=2017-05-27, xmax=2018-05-20,
		    ymin=200,ymax=3000,
		    height=8cm,
			width=14cm,
			grid=major,
		    xlabel={Data (dia-mês-ano)},
		    ylabel={Valor em dólares},
		]
		\addlegendentry{Preço BCH}
		\addplot [blue, mark=*] table [x=date, y=value, col sep=comma] {bcash_investing_12m.csv};
		\end{axis}
		\end{tikzpicture}
        \caption{Valor do Bitcoin Cash.}\label{gr:bcash_coindesk_12m}
	\end{subfigure}
	%sub fig 2
	\begin{subfigure}{}
		\begin{tikzpicture}

		\begin{axis}[
			title={Valor Bitcoin Cash no Google Trends}, 
		    date coordinates in=x,
		    xticklabel={\day-\month-\year},
		    x tick label style={rotate=45,anchor=north east},
		    xmin=2017-05-27, xmax=2018-05-20,
		    ymin=0,ymax=100,
		    height=8cm,
			width=14cm,
			grid=major,
		    xlabel={Data (dia-mês-ano)},
		    ylabel={Valor \%},  
		]
		\addlegendentry{Trends BCH}
		\addplot [red, mark=*] table [x=date, y=value, col sep=comma] {bcash_gtrends_12m.csv};
		\end{axis}
		\end{tikzpicture}
    	\caption{Valor do Bitcoin Cash.}\label{gr:bcash_gtrends_12m}
	\end{subfigure}
    \caption{Comparação dos gráfico do valor do Bitcoin Cash e do \textit{Google Trends}}\label{fg:comparacao_bcash_12m}
	\end{figure}
\end{center}
\subsection{Objetivo}
O objetivo deste trabalho reside na construção de um modelo neural, capaz de modelar relações não lineares, possivelmente presentes na série histórica do Bitcoin. O algoritmo base é o \emph{Backpropagation} \cite{hecht1992theory} que tem a capacidade de treinar as camadas intermediárias. Utilizaremos como fonte de dados para previsão da subida ou descida do valor do preço do Bitcoin os dados do Google Trends e o valor do preço do próprio Bitcoin. É esperado que o algoritmo aumente a acurácia vista na literatura, além de ser disponibilizado como uma API que possa ser usada por outros sistemas. O algoritmo que será proposto deverá categorizar o valor do preço futuro do Bitcoin em duas categorias, as quais serão a saída do algoritmo:
\begin{enumerate}
\item o valor da moeda aumentará, quando comparada ao seu valor anterior (COMPRE);
\item o valor da moeda diminuirá, quando comparada ao seu valor anterior (VENDA).
\end{enumerate}

\subsection{Metodologia}

Este trabalho irá utilizar as conclusões apresentadas em  \cite{matta2015bitcoin, mcnally2016predicting,weigend2018time}, que sugerem a relação entre a quantidade de vezes que o termo ``Bitcoin'' é buscado no Google e o valor do preço do Bitcoin. Os dados de entrada serão os valores da série histórica do Bitcoin, disponíveis no site Investing \cite{Investing}, e os valores referentes a quantidade de vezes que o termo "Bitcoin" é buscado no Google, disponíveis através da ferramenta \textit{Google Trends} \cite{GoogleTrends}. Os dados obtidos serão baixados para serem trabalhados, a princípio, de modo \textit{offline} para elaboração do algoritmo.

Os dados serão divididos em três partes:
\begin{enumerate}
\item base de treinamento: será utilizada para treinamento supervisionado do algoritmo;
\item base de testes: será utilizada pra testar o algoritmo treinado anteriormente;
\item base de previsão: será utilizada para testar o nível de acerto real do algoritmo.
\end{enumerate}

O algoritmo utilizado será o Backpropagation \cite{hecht1992theory}, que tem a capacidade de treinar as camadas intermediárias, através do cálculo o gradiente em relação aos coeficientes das camadas intermediárias. Antes do treinamento será usado o algoritmo do \textbf{PCA} \cite{Jolliffe:1986} (do inglês \emph{Principal Component Analysis}) para compressão dos dados, com objetivo de aumentar a velocidade de aprendizado do algoritmo. A rede neural deverá retornar a classificação da previsão dos valores do preço do Bitcoin em duas classes: o preço irá subir ou descer. Essa informação deverá auxiliar um investidor a tomar a decisão de compra ou de venda da cripto-moeda.


Os dados de treinamento serão utilizados para estimar o número de neurônios na camada escondida e o número de amostrar pregressas, que são variáveis no algoritmo Backpropagation, de forma a otimizar os resultados. 

Utilizaremos outros algoritmos, como o \textbf{RNN} (Rede Neurais recorrentes), \textbf{LSTM}  e \textbf{Análise de Regressão Linear}, para avaliar a capacidade de previsão do algoritmo proposto, onde o objetivo é superar os resultados obtidos por esses algoritmos, em especial em \cite{mcnally2016predicting}.